\documentclass{book}%This template is for use with HHL chapters and
            %requires a Bibtex file (here call BiBfile.bib)
                    %and requires named.sty, named.bst, elsevierg.sty
                    %to be available when LaTeXing the file.
\usepackage{makeidx,named,elsevierg,amsmath,latexsym,amssymb,rotate}
\usepackage[dvips]{graphicx}
\newcommand{\game}{\rotate[u]{\rotate[f]{$\Game$}}}
\newcommand{\breals}{{^{\omega}}\omega}
\newcommand{\creals}{{^{\omega}}2}
\def\underTilde#1{{\baselineskip=0pt\vtop{\hbox{$#1$}\hbox{$\sim$}}}{}}
\def\undertilde#1{{\baselineskip=0pt\vtop
  {\hbox{$#1$}\hbox{$\scriptscriptstyle\sim$}}}{}}
\newcommand{\delot}{\undertilde{\delta}^{1}_{2}}
\newcommand{\uTPi}{\underTilde{\Pi}}
\newcommand{\uTSigma}{\underTilde{\Sigma}}
\newcommand{\uTDelta}{\underTilde{\Delta}}
\newcommand{\utdelta}{\undertilde{\delta}}
\newcommand{\restrict}{\mathord{\upharpoonright}}
\newcommand{\less}{\mathord{<}}
\newcommand{\thing}{\mathord{-}}
\newcommand{\cut}{\mathord{||}}
\newcommand{\pmax}{\mathbb{P}{max}}
\newcommand{\dom}{{\rm dom}}

\makeindex
\collection
\begin{document}
\paper{A brief history of determinacy}{Paul B. Larson}\thanks{
The author is supported in part by NSF grant DMS-0801009.}

%Body of chapter here
%\tableofcontents


%\pagebreak

\section{Introduction}

Determinacy axioms are statements to the effect that certain games
are \emph{determined}, in that each player in the game has an
optimal strategy.
\footnote{That is, either one of the players has a strategy that guarantees a win, or both players
have strategies which guarantee at least a tie. Some of the games considered in this chapter allow
ties, but most do not.}
%%%%%SOLOVAY1: The footnote above was rewritten. The orginal is commented out below.
%\footnote{For most of this chapter, an optimal strategy will be a
%strategy that guarantees a win, though we will occasionally consider games in which ties are
%possible.}
The commonly accepted axioms for mathematics, the
Zermelo-Fraenkel axioms with the Axiom of Choice (ZFC; see
\cite{Jech:settheory, Kunen}), imply the determinacy of many games
that people actually play.
%via von Neumann's \emph{minimax} theorem \cite{vonNeumann}, for instance.
This applies in particular to many \emph{games of perfect
information}, games in which the players alternate moves which are known to both players, and the
outcome of the game depends only on this list of moves, and not on chance or other external factors. Games of perfect
information which must end in finitely many moves are determined. This follows from the
work of Ernst Zermelo\index{Zermelo, E.} \shortcite{Zermelo:1913},
D\'{e}nes K\H{o}nig\index{Konig, D.@K\H{o}nig, D.} \shortcite{Konig:1927} and L\'{a}szl\'{o} K\'{a}lmar\index{K\'{a}lmar, L.} \shortcite{Kalmar:1928/29}, and also from the independent work  of John von Neumann\index{von Neumann, J.} and Oskar Morgenstern\index{Morgenstern, O.} (in their 1944 book, reprinted as \shortcite{vNMo}).

As pointed out by Stanis{\l}aw Ulam\index{Ulam, S.} \shortcite{Ulam:1960},
determinacy for games of
perfect information of a fixed finite length is essentially a
theorem of logic. If we let
$x_{1}$,$y_{1}$,$x_{2}$,$y_{2}$,$\ldots$,$x_{n}$,$y_{n}$ be
variables standing for the moves made by players $I$
(who plays $x_{1}$,$\ldots$,$x_{n}$) and $II$ (who plays $y_{1}$,$\ldots$,$y_{n}$), and
$A$ (consisting of sequences of length $2n$) is the set of runs of
the game for which $I$ wins, the statement $$\exists x_{1} \forall
y_{1} \ldots \exists x_{n} \forall y_{n} \langle
x_{1},y_{1},\ldots,x_{n},y_{n} \rangle \in A$$ essentially asserts
that the first player has a winning strategy in the game, and its
negation, $$\forall x_{1} \exists y_{1} \ldots \forall x_{n} \exists
y_{n} \langle x_{1},y_{1},\ldots,x_{n},y_{n}\rangle \not\in A$$
essentially asserts that the second player has a winning strategy.\footnote{
If there exists a way of choosing a member from each nonempty set of moves of the game, then
these statements are actually equivalent to the assertions that the
corresponding strategies exist. Otherwise, in the absence of the Axiom of Choice
the statements above can hold without the
corresponding strategy existing.}

We let $\omega$ denote the set of natural numbers $0,1,2,\ldots$; for brevity we will often refer to the members of this set as ``integers''.
Given sets $X$ and $Y$, ${^{X}}Y$ denotes the set of functions
from $X$ to $Y$. The \emph{Baire space}\index{Baire space} is the space ${^{\omega}}\omega$, with the product topology.
The Baire space is homeomorphic to the space of irrational real numbers (see \cite[p.~9]{Moschovakis:DST09}, for instance),
and we will often refer to its members as ``reals" (though in various contexts
the Cantor space $\creals$, the set of subsets of $\omega$ ($\mathcal{P}(\omega)$) and the set of infinite subsets of $\omega$ ($[\omega]^{\omega}$) are all referred to as ``the reals").

Given $A \subseteq\breals$, we let
$G_{\omega}(A)$\index{$G_{\omega}(A)$} denote the game of perfect
information of length $\omega$ in which the two players collaborate to define an element $f$ of $\breals$ (with $I$ choosing
$f(0)$, $II$ choosing $f(1)$, $I$ choosing $f(2)$, and so on),
with $I$ winning a run of the game if and only if $f$ is an element of $A$. A game of this type is called
an \emph{integer game}, \index{integer game} and the set $A$ is called the \emph{payoff set}.\index{payoff set}
A \emph{strategy}\index{strategy in an integer game} in such a game for player $I$ ($II$) is a function
$\Sigma$ with domain the set of sequences of integers of even (odd) length such
that for each $a \in \dom(\Sigma)$, $\Sigma(a)$ is in $\omega$.
A run of the game (partial or complete) is said to be
\emph{according to} a strategy $\Sigma$ for player $I$ ($II$) if every
initial segment of the run of odd (nonzero even) length is of the form
$a^{\frown}\langle \Sigma(a) \rangle$ for some sequence $a$. A
strategy $\Sigma$ for player $I$ ($II$) is a \emph{winning strategy}\index{winning strategy}
if every
complete run of the game according to $\Sigma$ is in (out of) $A$.
We say that a set $A \subseteq{^{\omega}}\omega$ is \emph{determined}\index{determined set} (or  the corresponding game
$G_{\omega}(A)$ is determined) if there exists a winning strategy for one of the players. These notions
generalize naturally for games in which players play objects other than integers (for instance, \emph{real games},\index{real
game} in which they play elements of $\breals$) or games which run for more than $\omega$ many rounds (in which case player $I$ typically
plays at limit stages).

%More generally, for essentially all of this chapter it suffices to think of a
%\emph{game} as a set $T$ of finite sequences closed under initial
%segments (i.e., a \emph{tree}) and a set $A$ of sequences (finite or
%infinite) whose finite initial segments are all in $T$. We think of $T$ as a game between players
%$I$ and $II$, with sequences of even length corresponding to
%positions in the game where it is $I$'s turn to move, and positions
%of odd length corresponding to positions where it is $II$'s turn to
%move. The set $A$ is called the \emph{payoff} set; when a run of the
%game reaches a sequence with no proper extensions in $T$, $I$ is
%declared the winner if the sequence is in $A$, and $II$ is declared
%the winner otherwise. A \emph{strategy} for player $I$ is a function
%$\Sigma$ with domain the set of sequences in $T$ of even length such
%that for each $a \in \dom(\Sigma)$, $a^{\frown}\langle
%\Sigma(a)\rangle$ (the extension of $a$ by $\Sigma(a)$) is in $T$.
%Likewise, a \emph{strategy} for player $II$ is a function $\Sigma$
%with domain the set sequences in $T$ of odd length such that for
%each $a \in \dom(\Sigma)$, $a^{\frown}\langle \Sigma(a)\rangle$ is in
%$T$. A run of the game (partial or complete) is said to be
%\emph{according to} a strategy $\Sigma$ for player $I$ if every
%initial segment of the run of odd length is of the form
%$a^{\frown}\langle \Sigma(a) \rangle$ for some sequence $a \in T$ of
%even length. Likewise, a run of the game is said to be according to
%a strategy $\Sigma$ for player $II$ if every nonempty initial
%segment of the run of even length is of the form $a^{\frown}\langle
%\Sigma(a) \rangle$ for some sequence $a \in T$ of odd length. A
%strategy $\Sigma$ for player $I$ is a \emph{winning strategy} if every
%sequence according to $\Sigma$ whose finite initial segments are all
%in $T$ and which has no proper extensions in $T$ is in $A$. A
%strategy $\Sigma$ for player $II$ is a winning strategy if every
%sequence according to $\Sigma$ whose finite initial segments are all
%in $T$ and which has no proper extensions in $T$ is not in $A$. The
%Axiom of Determinacy then says that for every tree $T$ of finite
%sequences of integers and every payoff set for $T$ there exists
%either a winning strategy for player $I$ or a winning strategy for
%player $II$.

%A game in which the players play integers will be called an \emph{integer game}\index{integer game}.
%A game in which the players play real numbers will be called a \emph{real game}\index{real game}.

The study of determinacy axioms
concerns games whose determinacy is neither proved nor refuted by
the Zermelo-Fraenkel axioms ZF (without the Axiom of Choice). Typically such games are
infinite. Axioms
stating that infinite games of various types are determined were studied by Stanis{\l}aw Mazur\index{Mazur, S.}, Stefan Banach\index{Banach, S.} and Ulam\index{Ulam, S.} in
the late 1920's and early 1930's; were reintroduced by David Gale\index{Gale, D.} and Frank Stewart\index{Stewart, F.} \shortcite{GaleStewart} in the 1950's and again by Jan Mycielski\index{Mycielski, J.} and Hugo Steinhaus\index{Steinhaus, H.} \shortcite{MycielskiSteinhaus} in the early 1960's; gained interest with the work of
David Blackwell\index{Blackwell, D.} \shortcite{Blackwell:1967} and Robert Solovay\index{Solovay, R.} in the late 1960's; and attained increasing importance in the 1970's and 1980's, finally coming to a central position in
contemporary set theory.

Mycielski\index{Mycielski, J.} and Steinhaus\index{Steinhaus, H.} introduced the
Axiom of Determinacy (AD),\index{Axiom of Determinacy (AD)}\index{AD}
which asserts
the determinacy of $G_{\omega}(A)$ for all $A \subseteq \breals$. Work of
Banach\index{Banach, S.} in the 1930's shows that AD implies that all sets of reals satisfy the property
of Baire. In the 1960's, Mycielski\index{Mycielski, J.} and \'{S}wierczkowski\index{Swierczkowski, S.@\'{S}wierczkowski, S.} proved that AD implies
that all sets of reals are Lebesgue measurable, and Mycielski\index{Mycielski, J.} showed that AD implies
countable choice for reals. Together, these results show that determinacy provides a natural
context for certain areas of mathematics, notably analysis, free of the paradoxes induced
by the Axiom of Choice.

Unaware of the work of Banach,\index{Banach, S.} Gale\index{Gale, D.} and Stewart\index{Stewart, F.} \shortcite{GaleStewart}
had shown that the AD contradicts ZFC. However, the
proof used a wellordering of the reals given by the Axiom of Choice,
and therefore did not give a nondetermined game of this type with
definable payoff set. Starting with Banach's\index{Banach, S.} work, many simply definable
payoff sets were shown to induce determined games, culminating in
D. Anthony Martin's\index{Martin, D. A.} celebrated 1974 result \shortcite{Martin:1975} that all games
with Borel  payoff set are determined. This result came after Martin\index{Martin, D. A.} had used measurable
cardinals to prove the determinacy of games whose payoff set is an analytic sets of reals.

%%%%%SOLOVAY2: "games whose payoff set is an" inserted into the line above.

The study of determinacy gained interest from two theorems in 1967, the first due to
Solovay\index{Solovay, R.} and the second to Blackwell\index{Blackwell, D.}. Solovay\index{Solovay, R.} proved that under AD,
the first uncountable cardinal $\omega_{1}$ is a measurable cardinal, setting off a study
of strong Ramsey properties on the ordinals implied by determinacy axioms. Blackwell\index{Blackwell, D.} used
open determinacy (proved by Gale\index{Gale, D.} and Stewart\index{Stewart, F.}) to reprove a classical theorem of Kazimierz Kuratowski\index{Kuratowski, K.}. This also
led to the application, by John Addison\index{Addison, J.}, Martin\index{Martin, D. A.}, Yiannis Moschovakis\index{Moschovakis, Y.} and others, of stronger determinacy axioms
to produce structural properties for definable sets of reals. These axioms included the
determinacy of $\underTilde{\Delta}^{1}_{n}$ sets of reals, for $n \geq 2$, statements which would not be proved
consistent relative to large cardinals until the 1980's.

The large cardinal hierarchy was developed over the same period, and came to be seen as a method
for calibrating consistency strength. In the 1970's, various special cases of $\uTDelta^{1}_{2}$ determinacy were located
%%%%%SOLOVAY3: "special cases of $\uTDelta^{1}_{2}$" was "forms of analytic" in the line above
on this scale, in terms of the large cardinals needed to prove them. Determining the consistency (relative to large cardinals)
of forms of determinacy at the level of $\underTilde{\Delta}^{1}_{2}$ and beyond would take the introduction of new large cardinal concepts.
Martin\index{Martin, D. A.} (in 1978) and W. Hugh Woodin\index{Woodin, W. H.} (in 1984) would prove $\underTilde{\Pi}^{1}_{2}$-determinacy
and AD$^{L(\mathbb{R})}$ respectively, using hypotheses near the very top of the large cardinal hierarchy.
In a dramatic development, the hypotheses for these results would be significantly reduced through work of Woodin,\index{Woodin, W. H.} Martin\index{Martin, D. A.} and John Steel.\index{Steel, J.}
The initial impetus for this development was a seminal result of Matthew Foreman,\index{Foreman, M.} Menachem Magidor\index{Magidor, M.} and Saharon Shelah\index{Shelah, S.} which showed, assuming the existence of a
supercompact cardinal, that there exists a generic elementary embedding with well-founded ranged and critical point $\omega_{1}$.
%%%%%SOLOVAY4: The two preceeding sentences replace the three following commented lines.
%In a dramatic development, the hypotheses needed for these results would be significantly reduced, through
%work of Matthew Foreman,\index{Foreman, M.} Menachem Magidor\index{Magidor, M.} and Saharon Shelah,\index{Shelah, S.} who showed, assuming the existence of a supercompact
%cardinal, that there exists a generic elementary embedding with well-founded range and critical point $\omega_{1}$.
Combined with work of Woodin\index{Woodin, W. H.}, this yielded the Lebesgue measurability of all sets in the inner model
$L(\mathbb{R})$ from this hypothesis. Shelah\index{Shelah, S.} and Woodin\index{Woodin, W. H.} would reduce the hypothesis for this result further,
to the assumption that there exist infinitely many Woodin cardinals below a measurable cardinal.

Woodin cardinals would turn out to be the central large cardinal concept for the study of determinacy.
Through the study of tree representations for sets of reals, Martin\index{Martin, D. A.} and Steel\index{Steel, J.} would show that $\underTilde{\Pi}^{1}_{n+1}$-determinacy
follows from the existence of $n$ Woodin cardinals below a measurable cardinal, and that this hypothesis was not sufficient
to prove stronger determinacy results for the projective hierarchy. Woodin\index{Woodin, W. H.} would then show that the existence of infinitely
many Woodin cardinals below a measurable cardinal implies AD$^{L(\mathbb{R})}$, and he would locate the exact consistency strengths
of $\underTilde{\Delta}^{1}_{2}$-determinacy and AD$^{L(\mathbb{R})}$ at one Woodin cardinal and $\omega$ Woodin cardinals respectively.





%The Axiom of Determinacy is the statement that all two-person
%length-$\omega$ integer games of perfect information are determined.

In the aftermath of these results, many new directions were developed, and we give only the briefest indication here.
Using techniques from inner model theory, tight bounds were given for establishing the exact consistency strength
of many determinacy hypotheses. Using similar techniques, it has been shown that almost every natural statement (i.e.,
not invented specifically to be a counterexample) implies directly those determinacy hypotheses of lesser
consistency strength. For instance, by G\"{o}del's Second Incompleteness Theorem, ZFC cannot prove that the AD holds in $L(\mathbb{R})$, as the latter
implies the consistency of the former. Empirically, however,
every natural extension of ZFC without this limitation (i.e., not proved consistent by AD) does appear to imply that AD holds in $L(\mathbb{R})$. This sort of
phenomenon is taken by some as evidence that the statement that AD holds in $L(\mathbb{R})$, and other determinacy axioms,
should be counted among the true statements extending ZFC.


The history presented here relies heavily on those given by Jackson\index{Jackson, S.}
\shortcite{Jackson:handbook}, Kanamori\index{Kanamori, A.} \shortcite{Kanamori:1995, Kanamori}, Moschovakis\index{Moschovakis, Y.} \shortcite{Moschovakis:DST09}, Neeman\index{Neeman, I.}
\shortcite{Neeman:DLG} and Steel\index{Steel, J.} \shortcite{Steel:gamesscales}. As the title suggests, this is a selective and abbreviated account
of the history of determinacy. We have omitted many interesting topics, including, for instance,
Blackwell games \cite{Blackwell:1969, Martin:1998, MartinNeemanVervoort} and proving determinacy in second-order arithmetic \cite{LouveauSaintRaymond:1987, LouveauSaintRaymond:1988, KoellnerWoodin:handbook}.

\section{Early developments}

%The paper \cite{SchwalbeWalker:2001} is a very useful guide to untangling the early
%history of determinacy.

The first published paper in mathematical game theory appears to be
Zermelo's\index{Zermelo, E.} paper \shortcite{Zermelo:1913} on chess. Although he noted
%Zermelo's \cite{Zermelo:1913} studies chess, and although he notes
that his arguments apply to all games of reason not involving chance, Zermelo\index{Zermelo, E.}
worked under two additional chess-specific assumptions. The
first was that the game in question has only finitely many states, and the second
was that an infinite run of the game was to be considered a draw.
%Zermelo's primary interest in this paper was showing that if one
%player has a winning strategy in such a game which guarantees a win
%within a fixed number of moves, then he has a strategy that
%guarantees that he will win within in at most $t$ moves, where $t+1$
%is the number of possible states for the game. His proof of this
%statement was found by K\H{o}nig to contain an error, which was
%later corrected Zermelo in an appendix to \cite{Konig:1927}, using a
%result of K\H{o}nig's.
%In \cite{Zermelo:1913},
Zermelo\index{Zermelo, E.} specified a condition which is equivalent
to having a
winning strategy in such a game guaranteeing a win within a fixed number of moves,
as well as another condition equivalent to having a strategy guaranteeing that one
will not lose within a given fixed number of moves. His analysis implicitly introduced
the notions of \emph{game tree},
%(in the sense of Footnote \ref{treefoot}),
\emph{subtree} of a game tree, and \emph{quasi-strategy}.\footnote{As defined above, a strategy for a given player specifies a move in each
relevant position; a quasi-strategy merely specifies a set of acceptable moves.
The distinction is important when the Axiom of Choice fails, but is less important in the context of Zermelo's\index{Zermelo, E.} paper.}
%His notion of
%strategy differs from the modern notion - it is closer to the
%modern notion of quasi-strategy, in that it gives a set of
%acceptable moves at each point of the game accessible by the
%strategy, instead of prescribing a particular move.
%It has the
%advantage that the union of strategies of this type is a strategy.
%It also has the disadvantage that it will not work for infinite
%games, which were not his concern.

The paper states indirectly, but does not quite prove,
%(even aside from the error found by K\H{o}nig)
or even define, the statement
that in any game of perfect information with finitely many possible
positions such that infinite runs of the game are draws, either one
player has a strategy that guarantees a win, or both players have
strategies that guarantee at least a draw. A special case of this fact is
determinacy for games of perfect information of a fixed finite length,
which is sometimes called Zermelo's\index{Zermelo, E.} Theorem.%%\index{Zermelo's Theorem}

%This special case also follows from von Neumann's \emph{minimax} theorem \cite{vonNeumann},
%proved independently.
%fixed-finite length determinacy fact later proved independently by
%von Neumann \cite{vonNeumann}: if at a given point in the
%game one player does not have a strategy guaranteeing that he will
%not lose in the next $n$ moves (for a given integer $n$) then at
%point the other player has a strategy guaranteeing that he will win
%in the next $n$ moves.

%analyzed games of perfect information with finitely many possible
%positions (i.e., states of the game) and thus with finitely many
%possible moves for each player at each point of the game. Unlike von
%Neumann and Morgenstern, who considered games of a fixed finite
%length, Zermelo allowed for infinitely long runs of the game, but
%considered such a run to be a draw. Zermelo introduced the notion of
%\emph{winning position} in a game, and implicitly introduced the
%notions of \emph{winning strategy} and \emph{determined game}, and
%he outlined a proof of the following theorem (following many
%authors, we will call this fact \emph{Zermelo's
%Theorem}\index{Zermelo's Theorem}).

%\begin{theorem}[Zermelo \cite{Zermelo:1913}] Given a game of
%perfect information with finitely many possible positions such that
%infinite runs of the game are draws, either one player has a
%strategy that guarantees a win, or both players have strategies that
%guarantee at least a draw.
%\end{theorem}

%His primary concern is showing that if a player
%has a winning strategy in such a game, then he can bound in advance
%the number of moves required to win.

K\H{o}nig\index{Konig, D.@K\H{o}nig, D.} \shortcite{Konig:1927} applied the fundamental fact now
known as \emph{K\H{o}nig's Lemma}\index{Konig's Lemma@K\H{o}nig's Lemma} to the study of
games, among other topics. While
K\H{o}nig's formulation was somewhat different, his Lemma is
equivalent to the assertion that every infinite finitely branching
tree with a single root has an infinite path (a path can be found by
iteratively choosing any successor node such that the tree above
that node is infinite). Extending Zermelo's\index{Zermelo, E.} analysis to games in
which infinitely many positions are possible while retaining the
condition that each player has only finitely many options at each
point, K\H{o}nig used the statement above to prove that in such a
game, if one player has a strategy (from a given point in the game)
guaranteeing a win, then he can guarantee victory within a fixed number of moves.
The application of K\H{o}nig's Lemma to
the study of games was suggested by von Neumann.\index{von Neumann, J.}
%In order to phrase
%his result, K\H{o}nig needed a more general notion of winning
%strategy than the one used in Zermelo's paper. The appendix to
%\cite{Konig:1927} contains an argument by Zermelo, using K\H{o}nig's
%argument, proving correctly that if a player has a winning strategy,
%then he has one that guarantees a win in at most $t$ moves, as above
%(provided that $t$ as above exists).

%extend Zermelo's Theorem to games with infinitely many possible
%positions but only finitely many options for each player at each
%point.

K\'{a}lmar\index{K\'{a}lmar, L.} \shortcite{Kalmar:1928/29} took the analysis a step further by
proving Zermelo's\index{Zermelo, E.} Theorem for games with infinitely many possible
moves in each round. His arguments proceeded by assigning transfinite ordinals
to nodes in the game tree, a method which remains an important tool in modern set theory.
K\'{a}lmar\index{K\'{a}lmar, L.} explicitly introduced the notion of
a winning strategy for a game, though his strategies were also quasi-strategies
as above. In his analysis,
K\'{a}lmar\index{K\'{a}lmar, L.} introduced a number of other important technical notions,
including the notion of a \emph{subgame} (essentially a subtree of
the original game tree), and classifying strategies into those which
depend only on the current position in the game and those which use
the history of the game so far.\footnote{See \cite{SchwalbeWalker:2001} for
much more on these papers of Zermelo,
K\H{o}nig  and K\'{a}lmar.}

%\begin{theorem}[K\'{a}lmar \cite{Kalmar:1928/29}] If $G$ is a game of
%perfect information such that and infinite runs of the game are
%draws, then either one player has a strategy that guarantees a win,
%or both players have strategies that guarantee at least a draw.
%\end{theorem}


Games of perfect information for which the set of infinite runs is
divided into winning sets for each player appear in a question by Mazur\index{Mazur, S.} in the Scottish
Book, answered by Banach\index{Banach, S.} in an entry dated August 4, 1935   (see \cite[p.~113]{Mauldin:Scottish}).
%According to Ulam\index{Ulam, S.}
%\shortcite[p.~23]{Ulam:1960}, Banach\index{Banach, S.} proposed this game around 1928; according to
%Steinhaus\index{Steinhaus, H.} \shortcite[p.~464]{Steinhaus:1965} it was Banach\index{Banach, S.} and Mazur\index{Mazur, S.} together in 1925, just
%after the publication of Steinhaus's\index{Steinhaus, H.} \shortcite{Steinhaus:1925} (translated in \cite{DimDim}).
Following up on Mazur's\index{Mazur, S.} question (still in the Book), Ulam\index{Ulam, S.} asked about games where two
players collaborate to build an infinite sequence of 0's and 1's by
alternately deciding each member of the sequence, with the winner
determined by whether the infinite sequence constructed falls inside
some predetermined set $E$.
%(called the \emph{payoff set}\index{payoff set} in contemporary usage).
Essentially raising
the issue of determinacy for arbitrary $G_{\omega}(E)$, Ulam\index{Ulam, S.} asked:
for which sets $E$ does the first player (alternately, the second
player) have a winning strategy? (Section \ref{regularitysection} below has more on
the Banach-Mazur game.)

%John von Neumann's \emph{minimax} theorem in his 1928 paper \cite{vonNeumann:1928}
%proves the existence of optimal strategies in a class of games in which chance
%may have a role. The restriction of this result to games of perfect information
%is essentially determinacy for games of a finite fixed length.
Games of perfect information were formally defined
in 1944 by von Neumann\index{von Neumann, J.} and Morgenstern\index{Morgenstern, O.} \shortcite{vNMo}. Their book also contains a proof that
games of perfect information of a fixed finite length are determined
(page 123).

%Games of perfect information had appeared in earlier work of
%Zermelo, as well as in the Scottish Book (see
%\cite{Mauldin:Scottish}).

%\section{Determinacy in ZFC I}


Infinite games of perfect information were reintroduced by Gale\index{Gale, D.} and
Stewart\index{Stewart, F.} \shortcite{GaleStewart}, who were unaware of the work of Mazur\index{Mazur, S.},
Banach\index{Banach, S.} and Ulam\index{Ulam, S.} (Gale, personal communication). They showed that a nondetermined game can be
constructed using the Axiom of Choice (more specifically, from a
wellordering of the set of real numbers).\footnote{Given a set $Y$, we let AC$_{Y}$\index{AC$_{Y}$}
denote the statement that whenever $\{ X_{a} \mid a \in Y \}$
is a collection of nonempty sets, there is a
function $f$ with domain $Y$ such that $f(a) \in X_{a}$ for all $a
\in Y$. Zermelo's\index{Zermelo, E.}
\emph{Axiom of Choice}\index{Axiom of Choice (AC)} \shortcite{Zermelo:1904} is
equivalent to the statement that AC$_{Y}$ holds for all sets $Y$. A linear ordering $\leq$ of a set $X$ is a
\emph{wellordering}\index{well-ordering} if every nonempty subset of
$X$ has a $\leq$-least element. The Axiom of Choice is equivalent to
the statement that there exist wellorderings of every set.
K\H{o}nig's Lemma is a weak form of the Axiom of Choice and cannot
be proved in ZF (see \cite{Levy:settheory}, Exercise IX.2.18).}
They also noted that the proof from the Axiom of Choice does not
give a definable undetermined game, and raised the issue of whether
determinacy might hold for all games with a suitably definable
payoff set. Towards this end, they introduced a topological
classification of infinite games of perfect information, defining a
game (or the set of runs of the game which are winning for the first
player) to be \emph{open}\index{open game} if all winning runs for
the first player are won at some finite stage (i.e., if, whenever
$\langle x_{0},x_{1},x_{2},\ldots\rangle$ is a winning run of the
game for the first player, there is some $n$ such that the first
player wins all runs of the game extending $\langle
x_{0},\ldots,x_{n}\rangle$). Using this framework, they proved a
number of fundamental facts, including the determinacy of all games
whose payoff set is a Boolean combination of open sets (i.e., in the
class generated from the open sets by the operations of finite
union, finite intersection and complementation). The determinacy of
open games would become the basis for proofs of many of the strongest
determinacy hypotheses. Gale\index{Gale, D.} and Stewart\index{Stewart, F.} also asked a number of important
questions, including the question of whether all Borel games are
determined (to be answered positively by Martin\index{Martin, D. A.} \shortcite{Martin:1975}
in 1974).\footnote{The \emph{Borel} sets\index{Borel set} are the members of the
smallest class containing the open sets and closed under the operations of
complementation and countable union. The collection of Borel sets is generated in $\omega_{1}$
many stages from these two operations. A natural process assigns a measure to each Borel set (see, for instance,
\cite{Halmos:measuretheory}).}
%%%%%SOLOVAY5:The previous sentence replaces "This process naturally assigns a measure to each Borel set", which
%%%%%SOLOVAY5:Solovay pointed out is not quite correct. Halmos has been added to the biliography.
Classifying games by the definability of their payoffs sets would
be an essential tool in the study of determinacy.


%They considered games of a fixed finite length and proved their
%determinacy. While our focus is on infinite games, which are not
%discussed in \cite{}, the proof of the determinacy of all finite
%games of perfect information, given by \cite{}, uses the important
%technique of ranking positions in a games. Another important move is
%to consider the set of runs of a game as a topological space with
%the initial segment topology, and to classify games in terms of the
%topological complexity of the payoff set. In conjunction with
%K\H{o}nig's lemma, this technique is used in \cite{} to prove the
%determinacy of closed games, and thus closed ones as well. Gale and
%Stewart \cite{GS} proved that the Axiom of Choice implies that there
%are undetermined games.



\subsection{Regularity properties}\label{regularitysection}

Early motivation for the study of determinacy was given by its
implications for regularity properties for sets of reals. In
particular, determinacy of certain games of perfect information was
shown to imply that every set of reals has the property of Baire and
the perfect set property, and is Lebesgue measurable.\footnote{A set
of reals $X$ has the \emph{property of Baire}\index{Baire property}
if $X \bigtriangleup O$ is meager for some open set $O$,
where the \emph{symmetric difference}\index{symmetric difference} $A
\bigtriangleup B$ of two sets $A$ and $B$ is the set $(A \setminus
B) \cup (B \setminus A)$, where $A \setminus B = \{ x \in A \mid x
\not\in B\}$\index{$A\setminus B$}. A set of reals $X$ has the
\emph{perfect set property}\index{perfect set property} if it is
countable or contains a perfect set (an uncountable closed set
without isolated points). A set of reals $X$ is \emph{Lebesgue
measurable}\index{Lebesgue measure} if there is a Borel set
$B$ such that $X \bigtriangleup B$ is a subset of a Borel measure $0$ set. See
\cite{Oxtoby}.} These three facts themselves each contradict the Axiom
of Choice. We will refer to Lebesgue measurability, the property of
Baire and the perfect set property as the \emph{regularity
properties}\index{regularity property}, the fact that there are
other regularity properties notwithstanding.

Question 43 of the Scottish Book, posed by Mazur,\index{Mazur, S.} asks about games
where two players alternately select the members of a shrinking
sequence of intervals of real numbers, with the first player the
winner if the intersection of the sequence intersects a set given in
advance. Banach posted an answer in 1935, showing that such games
are determined if and only if the given set is either meager (in which case the
second player wins) or comeager relative to some interval (in which
case the first player wins), i.e., if and only if the given set has the Baire property (see \cite[pp.~27-30]{Oxtoby}, \cite[pp.~373-374]{Kanamori}).
This game has come to be known as the
Banach-Mazur game. Using an enumeration of the rationals,
one can code intervals with rational
endpoints with integers, getting a game on integers.

Morton Davis\index{Davis, M.} \shortcite{Davis64} studied a game, suggested by Dubins, where
the first player plays arbitrarily long finite strings of 0's and
1's and the second player plays individual 0's and 1's, with the
payoff set a subset of the set of infinite binary sequences as
before. Davis\index{Davis, M.} proved that the first player has a
winning strategy in such a game if and only if the payoff set
contains a perfect set,
%(an uncountable closed set with no isolated points)
and the second player has a winning strategy if and only if the payoff set is
finite or countably infinite. The determinacy of all such games then
implies that every uncountable set of reals contains a perfect set
(asymmetric games of this type can be coded by integer
games of perfect information). It follows that under AD
there is no set of reals whose cardinality falls strictly between
$\aleph_0$ and $2^{\boldsymbol\aleph_0}$.\footnote{i.e., for every
set $X$, if there exist injections $f \colon \omega \to X$ and $g
\colon X \to 2^{\omega}$, then either $X$ is countable or there
exists a bijection between $X$ and $2^{\omega}$.}

Mycielski\index{Mycielski, J.} and Stanis{\l}aw \'{S}wierczkowski\index{Swierczkowski, S.@\'{S}wierczkowski, S.} \shortcite{MycielskiSwierczkowski} showed
that the determinacy of certain integer games of perfect information
implies that every subset of the real line is Lebesgue measurable. Simpler proofs of this fact
were later given by Leo Harrington\index{Harrington, L.} (see \cite[pp.~375-377]{Kanamori})
and Martin\index{Martin, D. A.} \shortcite{Martin:2003}.

By way of contrast, an argument of Vitali \shortcite{Vitali:1905} shows that under ZFC there are sets of reals
which are not Lebesgue measurable. Banach and Tarski (\shortcite{BanachTarski:1924}, see also
\cite{Wagon:BTP}), building on work of Hausdorff \shortcite{Hausdorff:1914}, showed that under ZFC the unit ball can be partitioned into five pieces
which can be rearranged to make two copies of the same sphere, again violating Lebesgue measurability as well as physical
intuition. As with the undetermined game given by Gale\index{Mycielski, J.} and Stewart,\index{Stewart, F.} the constructions of Vitali and Banach-Tarski
use the Axiom of Choice and do not give definable examples of nonmeasurable sets. Via the Mycielski-\'{S}wierczkowski theorem,
determinacy results would rule out the existence of definable examples, for various notions of definability.


\subsection{Definability}\label{defsec}

As discussed above, ZFC implies that open sets are determined, and implies also that there exists a nondetermined set. The study of determinacy
was to merge naturally with the study of sets of reals in terms of their definability (i.e., descriptive set theory), which can be taken
as a measure of their complexity. In this section we briefly introduce some important definability classes for sets of reals. Standard references
include \cite{Moschovakis:DST, Kechris:CDST}. While we do mention some important results in this section, much of the section can be skipped on a first reading and used for later reference.

A \emph{Polish space}\index{Polish space} is a topological space which is separable and completely metrizable. Common examples include the integers $\omega$,
the reals $\mathbb{R}$, the open interval $(0,1)$, the Baire space $\breals$, the Cantor
space $\creals$ and their finite and countable products. Uncountable Polish spaces without isolated points are a natural setting
for studying definable sets of reals. For the most part we will concentrate on the Baire space and its finite powers.

Following notation introduced by Addison
\index{Addison, J.}\shortcite{Addison:1958},\footnote{The papers \cite{Addison:1958} and
\cite{Addison:1959} appear in the same volume of \emph{Fundamenta
Mathematicae}. The front page of the volume gives the date 1958-1959.
The individual papers have the dates 1958 and 1959 on them,
respectively.}
%and Shoenfield\index{Shoenfield, J.} \cite{Shoenfield:1961},
open subsets of a Polish space are called
$\uTSigma^{0}_{1}$, complements of
$\uTSigma^{0}_{n}$%%\index{$\uTSigma^{0}_{n}$}\index{$\uTPi^{0}_{n}$}
sets are $\uTPi^{0}_{n}$, and countable unions of
$\uTPi^{0}_{n}$ sets are $\uTSigma^{0}_{n+1}$. More generally, given
a positive $\alpha < \omega_{1}$, $\uTSigma^{0}_{\alpha}$ consists of all
countable unions of members of $\bigcup_{\beta < \alpha}\uTPi^{0}_{\beta}$, and
$\uTPi^{0}_{\alpha}$ consists of all complements of members of $\uTSigma^{0}_{\alpha}$.
%%%%%SOLOVAY5: ", and $\uTPi^{0}_{\alpha} consists...." was added to the end of the previous sentence.
The Borel sets\index{Borel set} are the members of
$\bigcup_{\alpha <\omega_{1}}\uTSigma^{0}_{\alpha}$.

A \emph{pointclass}\index{pointclass} is a collection of subsets of Polish spaces.
%for $\breals$ is a
%collection of sets, each a subset of $(\breals)^{k}$ for some fixed
%positive integer $k$.
Given a pointclass $\Gamma \subseteq \mathcal{P}(\breals)$, we let Det($\Gamma$)\index{Det($\Gamma$)} and
$\Gamma$-\emph{determinacy}\index{$\Gamma$-determinacy} each denote
the statement that $G_{\omega}(A)$ is determined for all $A \in \Gamma$.
%all length-$\omega$ integer games of perfect information
%where the payoff set is in $\Gamma$
%are determined.
%(i.e., determinacy for games where the payoff set is a countable intersection of open
%sets ($G_{\delta}$)) in ZFC.
Philip Wolfe\index{Wolfe, P.}
\shortcite{Wolfe55} proved $\uTSigma^{0}_{2}$-determinacy in ZFC. Davis\index{Davis, M.} \shortcite{Davis64} followed by proving $\uTPi^{0}_{3}$-determinacy. Jeffrey Paris\index{Paris, J.} \shortcite{Paris:72} would prove $\uTSigma^{0}_{4}$-determinacy.
However, this result was proved after Martin\index{Martin, D. A.} had used a measurable cardinal to prove analytic determinacy
(see Section \ref{Borelsub}).
%(determinacy for games where the
%payoff set is a countable intersection of countable unions of closed
%sets ($F_{\sigma\delta}$)).





Continuous images of
$\uTPi^{0}_{1}$ sets are said to be
$\uTSigma^{1}_{1}$, complements of
$\uTSigma^{1}_{n}$ sets are
$\uTPi^{1}_{n}$,%%\index{$\uTSigma^{1}_{n}$}%%\index{$\uTPi^{1}_{n}$}
and continuous images of $\uTPi^{1}_{n}$ sets are
$\uTSigma^{1}_{n+1}$. For each $i \in \{0,1\}$ and $n \in
\omega$, the pointclass $\underTilde{\Delta}^{i}_{n}$ is the intersection of
$\uTSigma^{i}_{n}$ and $\uTPi^{i}_{n}$. The \emph{boldface projective
pointclasses}\index{boldface projective pointclass}\index{pointclass!boldface projective}
$\uTSigma^{1}_{n}$,
$\uTPi^{1}_{n}$, $\uTDelta^{1}_{n}$ were introduced independently by
Nikolai Luzin\index{Lusin, N.} \shortcite{Luzin:1925a, Luzin:1925b, Luzin:1925c} and Wac{\l}aw Sierpi\'{n}ski\index{Sierpi\'{n}ski, W.}
\shortcite{Sierpinski:1925}. The notion of a boldface pointclass in general (i.e., possibly non-projective) is used in
various ways in the literature. We will say that a pointclass $\Gamma$ is \emph{boldface}\index{boldface pointclass}\index{pointclass!boldface}
(or \emph{closed under continuous preimages}
%%\index{closure under continuous preimages}
or \emph{continuously closed}\index{pointclass!continuously closed})
 if $f^{-1}[A] \in \Gamma$ for all $A \in \Gamma$ and
all continuous functions $f$ between Polish spaces (where $A$ is a subset of the codomain).
%Consider for example the continuous function from $X$ to $X \times X$ (for some Polish space
%$X$ and some $x \in X$) defined by $y \mapsto (x,y)$.
The classes $\uTSigma^{0}_{n}$,
$\uTPi^{0}_{n}$, $\uTDelta^{0}_{n}$ are also boldface in this sense.


The pointclass $\uTSigma^{1}_{1}$ is also known as the class of
\emph{analytic sets},\index{analytic set} and was given an
%%%%%SOLOVAY7: "an" inserted in the previous line.
independent characterization by Mikhail Suslin\index{Suslin, M.} \shortcite{Suslin:1917}: a set of reals $A$ is
analytic if and only if there exists a family of closed sets $D_{s}$ (for each finite sequence $s$ consisting of integers)
such that $A$ is the set of reals $x$ for which there is an $\omega$-sequence $S$ of integers such that $x \in \bigcap_{n \in \omega}D_{S \restrict n}$.\footnote{For $S$ a function with domain $\omega$, and $n \in \omega$, $S \restrict n = \langle S(0),\ldots, S(n-1)\rangle$.}
Suslin showed that
there exist non-Borel analytic sets, and that
the Borel sets are exactly the $\uTDelta^{1}_{1}$ sets.


We let $\exists^{0}$ and $\exists^{1}$
denote existential quantification over the integers and reals,
respectively, and $\forall^{0}$ and $\forall^{1}$ the analogous
forms of universal quantification. Given a set $A \subseteq(\breals)^{k+1}$, for some positive integer $k$, $\exists^{1}A$ is
the set of $(x_{1},\ldots,x_{k}) \in (\breals)^{k}$
such that for some $x \in \breals$, $(x, x_{1},\ldots,x_{k}
) \in A$, and $\forall^{1}A$ is the set of $(x_{1},\ldots,x_{k}) \in (\breals)^{k}$ such that for all $x
\in \breals$, $(x, x_{1},\ldots,x_{k}) \in A$. Given a
pointclass $\Gamma$, $\exists^{1}\Gamma$ consists of $\exists^{1}A$
for all $A \in \Gamma$, and $\forall^{1}\Gamma$ consists of
$\forall^{1}A$ for all $A \in \Gamma$. It follows easily that for each
positive integer $n$, $\exists^{1}\uTPi^{1}_{n} = \uTSigma^{1}_{n+1}$ and
$\forall^{1}\uTSigma^{1}_{n} = \uTPi^{1}_{n+1}$.

Given a pointclass $\Gamma$, $\neg \Gamma$\index{$\neg\Gamma$} is the set of complements of members of $\Gamma$,
and $\Delta_{\Gamma}$ is the pointclass $\Gamma \cap \neg\Gamma$;\index{$\Delta_{\Gamma}$}
$\Gamma$ is said to be
\emph{selfdual}\index{selfdual pointclass}\index{pointclass!selfdual}
if $\Delta_{\Gamma} = \Gamma$.
A set $A \in \Gamma$ is said to be $\Gamma$-complete if
every member of $\Gamma$ is a continuous preimage of $A$.
If $\Gamma$ is closed under continuous preimages
and $\Gamma$-determinacy holds, then $\neg\Gamma$-determinacy holds.
Each of the regularity properties for a set of reals $A$ are given
by the determinacy of games with payoff set simply definable from
$A$ (indeed, continuous preimages of $A$), but not necessarily with payoff $A$
itself. It follows that when $\Gamma$ is a boldface pointclass,
$\Gamma$-determinacy implies the regularity properties for sets of
reals in $\Gamma$.

A simple application of Fubini's theorem shows that if $\Gamma$ is a boldface pointclass
and there exists in $\Gamma$ a wellordering of a set of reals of positive Lebesgue measure,
then there is a non-Lebesgue measurable set in $\Gamma$.
%Each boldface
%pointclass $\Gamma$ is also closed under
%quantification over the integers, from which it follows that if
%there is a wellordering of the reals in $\Gamma$, then there is a
%non-Lebesgue measurable set of reals in $\Gamma$.
%It follows that
%$\Pi^{1}_{1}$-determinacy fails in $L$.
Skipping ahead for a moment, in the early 1970's
Alexander Kechris\index{Kechris, A.} and Martin\index{Martin, D. A.}, using a technique of Solovay\index{Solovay, R.}
called \emph{unfolding}, proved that for
each integer $n$, $\uTPi^{1}_{n}$-determinacy plus \emph{countable
choice for sets of reals}\footnote{The statement that whenever $X_{n}$ $(n
\in \omega$) are nonempty sets of reals, there is a function $f
\colon \omega \to \mathbb{R}$ such that $f(n) \in X_{n}$ for each
$n$.\index{countable choice for
sets of reals} Countable choice for sets of reals is a consequence of AD,
as shown by Mycielski\index{Mycielski, J.} \shortcite{Mycielski:6364} (see Section \ref{adsubsec}).}
implies that all $\uTSigma^{1}_{n+1}$ sets of reals are
Lebesgue measurable, have the Baire property and have the perfect
set property (see \cite[pp.~380-381]{Kanamori}).






As developed by Stephen Kleene\index{Kleene, S.}, the \emph{effective}
(or \emph{lightface}\index{pointclass!effective}\index{lightface pointclass}\index{pointclass!lightface}) pointclasses
$\Sigma^{0}_{n}$, $\Pi^{0}_{n}$, $\Delta^{0}_{n}$ \shortcite{Kleene:1943} and $\Sigma^{1}_{n}$,
$\Pi^{1}_{n}$, $\Delta^{1}_{n}$ \shortcite{Kleene:1955ajm, Kleene:1955bams, Kleene:1955tams} are formed in the same way as their
boldface counterparts, starting instead from $\Sigma^{0}_{1}$, the
collection of open sets $O$ such that the set of indices for basic
open sets contained in $O$ (under a certain natural enumeration of the
basic open sets) is recursive (see \cite{Moschovakis:DST09}, for
instance). Sets in $\Sigma^{0}_{1}$ are called
\emph{semirecursive}\index{semirecursive set of reals}, and sets in
$\Delta^{0}_{1}$ are called \emph{recursive}\index{recursive
set of reals}. Given $a \in \breals$, $\Sigma^{0}_{1}(a)$ is the
collection of open sets $O$ such that the set of indices for basic
open sets contained in $O$ is recursive in $a$, and the
\emph{relativized lightface projective pointclasses}\index{pointclass!relativized lightface
projective} $\Sigma^{0}_{n}(a)$, $\Pi^{0}_{n}(a)$,
$\Delta^{0}_{n}(a)$, $\Sigma^{1}_{n}(a)$, $\Pi^{1}_{n}(a)$,
$\Delta^{1}_{n}(a)$ are built from $\Sigma^{0}_{n}$ in the manner
above. It follows that each boldface pointclass is the union of the
corresponding relativized lightface classes (relativizing over each
member of $\breals$).

Following \cite{Moschovakis:DST09}, a pointclass is
\emph{adequate}\index{adequate pointclass}\index{pointclass!adequate}
if it contains all recursive sets
and is closed under finite unions and intersections, bounded
universal and existential integer quantification (see \cite[p.~119]{Moschovakis:DST09}) and preimages by
recursive functions.\footnote{A function
$f$ from a Polish space $X$ to a Polish space $Y$ is said to be
\emph{recursive}\index{recursive function on a Polish space} if the set of pairs
$x \in X$, $n \in \omega$ such that $f(x)$ is in the $n$-th basic open neighborhood of
$Y$ is semi-recursive.} The arithmetic and projective pointclasses are
adequate (see \cite[pp.~118-120]{Moschovakis:DST09}).
%A pointset is said to be \emph{semi-recursive}\index{semi-recursive pointset} if it is the
%union of a recursively listed sequence of basic open intervals,
%and \emph{recursive}\index{recursive pointset} it both it and its complement are recursive.


Given a Polish space $X$, an integer $k$, a set $A \subseteq X^{k+1}$ and $x \in X$,
$A_{x}$\index{$A_{x}$} is the set of $(x_{1},\ldots,x_{k})$ such
that $(x,x_{1},\ldots,x_{k}) \in A$. A set $A \subseteq X^{k+1}$ in
a pointclass $\Gamma$ is said to be \emph{universal}\index{pointclass!universal member of a}
for $\Gamma$ if each
subset of $X^{k}$ in $\Gamma$ has the form $A_{x}$ for some $x \in X$. Pointclasses of the form
$\Sigma^{1}_{n}$, $\Pi^{1}_{n}$ have universal members. Those of the form $\Delta^{1}_{n}$ do not. Each member of each
boldface pointclass is of the form $A_{x}$ for $A$ a member of the
corresponding effective class. Conversely, as each member of each
lightface projective pointclass listed above is definable, each member of each
corresponding boldface pointclass is definable from a real number as
a parameter.

A set of reals is said to be $\Sigma^{2}_{1}$ ($\Pi^{2}_{1}$) if is definable by a formula of the
form $\exists X \subseteq \mathbb{R}\, \phi$ ($\forall X \subseteq \mathbb{R}\, \phi$),
where all quantifiers in $\phi$ range over the reals or the integers.


In the \emph{L\'{e}vy hierarchy}\index{Levy hierarchy}\index{Levy, A.} \cite{Levy:1965},   a
formula $\phi$ in the language of set theory is $\Delta_{0}$ (equivalently $\Sigma_{0}$, $\Pi_{0}$) if
all quantifiers appearing in $\phi$ are bounded (see \cite[Chapter 13]{Jech:settheory}); $\Sigma_{n+1}$ if it has the
form $\exists x \psi$ for some $\Pi_{n}$ formula $\psi$; and $\Pi_{n+1}$ if it has the form $\forall x \psi$ for some $\Sigma_{n}$
formula $\psi$. A set is {\it $\Sigma_{n}$-definable} if it can be defined by a $\Sigma_{n}$ formula (and similarly for
$\Pi_{n}$). We say that a model $M$ is $\Gamma$-\emph{correct},\index{$\Gamma$-correct model} for a class of
formulas $\Gamma$, if for all $\phi$ in $\Gamma$ and $x \in M$, $M \models \phi(x)$ if and only if $V \models \phi(x)$.
If $M$ is a model of ZF, we say that a set in $M$ is $\Sigma^{M}_{n}$ if it is definable by a $\Sigma_{n}$
formula relativized to $M$ (and similarly for other classes of formulas).



G\"{o}del's inner model $L$\index{$L$} is the smallest transitive model of
ZFC containing the ordinals. For any set $A$, G\"{o}del's constructible universe $L$ generalizes to two inner models $L(A)$\index{$L(A)$} and $L[A]$,\index{$L[A]$} developed respectively by Andr\'{a}s Hajnal\index{Hajnal, A.} \shortcite{Hajnal:1956, Hajnal:1961} and Azriel L\'{e}vy\index{Levy, A.} \shortcite{Levy:1957, Levy:1960} (see \cite[Chapter 13]{Jech:settheory} or \cite[p.~34]{Kanamori}). Given a set $A$, $L(A)$\index{$L(A)$}
is the smallest transitive model of ZF containing $A$ and the ordinals and
$L[A]$\index{$L[A]$} is the smallest transitive model of ZF containing the
%%%%%SOLOVAY8: "transitive" inserted three times in the preceedings lines of this paragraph.
ordinals and closed under the function $X \mapsto A \cap X$.  Alternately, $L(A)$ is constructed in the same manner as $L$, but introducing the members of the transitive closure of the set $\{A\}$ at the first level, and $L[A]$ is constructed as $L$, but by adding a predicate for membership in $A$ to the language.\footnote{A set $x$ is \emph{transitive}\index{transitive set} if $z \in x$ whenever
$y \in x$ and $z \in y$. The \emph{transitive closure}\index{transitive closure} of a set $x$
is the smallest transitive set containing $x$.}  When $A$ is contained in $L$, $L(A)$ and
$L[A]$ are the same. While $L[A]$ is always a model of AC, $L(A)$ need not be.
Indeed, $L(\mathbb{R})$ is a model of AD in the presence of suitably large cardinals, and is thus a natural example of a ``smaller universum'' as described in the quote from \cite{MycielskiSteinhaus} in Section \ref{adsubsec}.


Though it can be formulated in other ways, we will view the set
$0^{\#}$\index{$0^\#$} (``zero sharp'')
as the theory of a certain class of ordinals which are indiscernibles over the inner model $L$. This notion was independently isolated by Solovay\index{Solovay, R.} \shortcite{Solovay:1967}
%%%%%SOLOVAY9: "\index{Solovay, R.} \shortcite{Solovay:1967}" was added to the previous line, and the reference Solovay:1967 was added to the bibliography.
and by Jack Silver\index{Silver, J.} in his 1966 Berkeley Ph.D. thesis (see
%%%%%SOLOVAY10: "by" inserted in the previous line.
\shortcite{Silver:1971}). The existence of $0^{\#}$ cannot be proved in
ZFC, as it serves as a sort of transcendence principle over $L$. For
instance, if $0^{\#}$ exists then every uncountable cardinal of $V$
is a strongly inaccessible cardinal in $L$.\footnote{A cardinal $\kappa$ is \emph{strongly
inaccessible}\index{inaccessible cardinal} if it is uncountable, regular
and a strong limit (i.e., $2^{\gamma} < \kappa$ for all
%%%%%SOLOVAY11: the preceeding line was modified to include uncountability as a hypothesis.
$\gamma < \kappa$). If $\kappa$ is a strongly inaccessible cardinal,
then $V_{\kappa}$ is a model of ZFC. Hence, by G\"{o}del's Second
Incompleteness Theorem, the existence of strongly
inaccessible cardinals cannot be proved in ZFC.
%%%%%SOLOVAY12: The previous two setences were one in the previous version.
See \cite{Jech:settheory} for the definition of
$V_{\alpha}$, for an ordinal $\alpha$.} For any set $X$ there is
an analogous notion of $X^{\#}$
(``$X$ sharp'')\index{sharp of a set} serving as a transcendence principle
over $L(X)$ (see \cite{Kanamori}).






\subsection{The Axiom of Determinacy}\label{adsubsec}

The Axiom of Determinacy, the
statement that all length $\omega$ integer games of perfect information are determined, was
proposed by Mycielski\index{Mycielski, J.} and Steinhaus\index{Steinhaus, H.} \shortcite{MycielskiSteinhaus}.\footnote{We continue to
use the now-standard abbreviation AD for the Axiom of Determinacy;
it was called (A) in \cite{MycielskiSteinhaus}.} In
a passage that anticipated a commonly accepted view of
determinacy, they wrote
\begin{quote}It is not the purpose of this paper to depreciate the
classical mathematics with its fundamental ``absolute'' intuitions on
the universum of sets (to which belongs the axiom of choice), but
only to propose another theory which seems very interesting although
its consistency is problematic. Our axiom can be considered as a
restriction of the classical notion of a set leading to a smaller
universum, say of determined sets, which reflect some physical
intuitions which are not fulfilled by the classical sets $\ldots$
Our axiom could be considered as an axiom added to the classical set
theory claiming the existence of a class of sets satisfying (A) and
the classical axioms (without the axiom of choice).\end{quote}
Mycielski\index{Mycielski, J.} and Steinhaus\index{Steinhaus, H.} summarized the state of knowledge of
determinacy at that time, including the fact that AD implies that all sets of
reals are Lebesgue measurable and have the Baire property,
and they noted that by results of Kurt G\"{o}del\index{Godel, K@G\"odel, K.} and
Addison\index{Addison, J.} \shortcite{Addison:1958}, there is in G\"{o}del's constructible
universe $L$ (and thus consistently with ZFC) a $\Delta^{1}_{2}$ wellordering of the reals,
and thus a $\Delta^{1}_{2}$ set which is not determined.


In his \shortcite{Mycielski:6364}, Mycielski\index{Mycielski, J.} proved several fundamental facts
about determinacy, including the fact that AD implies countable
choice for set of reals (he credits this result to \'{S}wierczkowski, Dana Scott and himself, independently).
Thus, while AD contradicts the Axiom of Choice, it
implies a form of Choice which suffices for many of its most
important applications, including the countable additivity of Lebesgue measure. Via countable choice for
sets of reals, AD implies that $\omega_{1}$ is regular.\footnote{The ordinal
$\omega_{1}$ is the first uncountable ordinal. A
%%%%%SOLOVAY13: "the" inserted in the previous line.
cardinal $\kappa$ is \emph{regular}\index{regular cardinal} if, for
every $\gamma < \kappa$, every function $f \colon \gamma \to \kappa$
has range bounded in $\kappa$. Under ZFC, every successor cardinal
is regular. Feferman\index{Feferman, S.} and L\'{e}vy\index{Levy, A.} \shortcite{FefermanLevy:1963} (see also \cite[pp.~153-154]{HowardRubin})
showed that the singularity of $\omega_{1}$ is
consistent with ZF. Moti Gitik\index{Gitik, M.} \shortcite{Gitik:1980} showed that
it is consistent with ZF (relative to large cardinals) that
$\omega$ is the largest regular cardinal.} Mycielski\index{Mycielski, J.} also showed
that AD implies that there is no uncountable wellordered sequence of
reals. In conjunction with the perfect set property, this implies that under
determinacy, $\omega_{1}^{V}$ is a strongly inaccessible
cardinal in the inner model
$L$ (and even in $L[a]$ for any real number
$a$), a fact which was to be greatly extended by
Solovay\index{Solovay, R.}, Martin\index{Martin, D. A.} and Woodin\index{Woodin, W. H.}. Harrington\index{Harrington, L.} \shortcite{Harrington:1978} would
show that $\Pi^{1}_{1}$-determinacy implies that $0^{\#}$ exists,
and thus that $\Pi^{1}_{1}$-determinacy is not
provable in ZFC.

In the same paper, Mycielski\index{Mycielski, J.} showed that ZF implies the existence of an
undetermined game of perfect information of length $\omega_{1}$
where the players play countable ordinals instead of integers. An
interesting aspect of the proof is that it does not give a specific
undetermined game. As a slight variant on Mycielski's\index{Mycielski, J.} argument, consider the game in which the first
player plays a countable ordinal $\alpha$ (and then makes no other
moves for the rest of the game) and the second player plays a
sequence of integers coding $\alpha$, under some fixed coding of
hereditarily countable sets by reals.\footnote{The \emph{hereditarily countable}\index{hereditarily countable set} sets
are those sets whose transitive closures are countable. Such sets are naturally coded by
sets of integers.} Since the first player cannot
have a winning strategy in this game, determinacy for the game
implies the existence of an injection from $\omega_{1}$ into
$\mathbb{R}$, which contradicts AD but is certainly by itself
consistent with ZF, as it follows from ZFC. Later results of Woodin\index{Woodin, W. H.} would
show that, assuming the consistency of certain large cardinal hypotheses, ZFC is consistent with the
statement that every integer game of length $\omega_{1}$ with payoff set definable from real
and ordinal
parameters is determined (see Section \ref{longgames}, and \cite[p.~298]{Neeman:DLG}). Mycielski\index{Mycielski, J.} noted that under AD there are no nonprincipal ultrafilters\footnote{An
\emph{ultrafilter}\index{ultrafilter} on a nonempty set $X$ is a
collection $U$ of nonempty subsets of $X$ which is closed under
supersets and finite intersections, and which has the property that
for every $A \subseteq X$, exactly one of $A$ and $X \setminus A$ is
in $U$. An ultrafilter is \emph{nonprincipal}\index{ultrafilter!nonprincipal}
if it contains no finite sets. The existence of
nonprincipal ultrafilters on $\omega$ follows from ZFC, but (as this
result shows) requires the Axiom of Choice.} on $\omega$ (this
follows from Lebesgue measurability for all sets of reals plus a
result of Sierpi\'{n}ski\index{Sierpi\'{n}ski, W.} \shortcite{Sierpinski:1938} showing that
nonprincipal ultrafilters on $\omega$ give rise to nonmeasurable
sets of reals), which implies that every ultrafilter is countably
complete (i.e., closed under countable intersections). Finally, in a footnote on the first page of the paper,
Mycielski\index{Mycielski, J.} reiterated a point made in the passage quoted above from his paper with Steinhaus, suggesting
that an inner model
containing the reals could satisfy AD. In a followup paper, Mycielski\index{Mycielski, J.} \shortcite{Mycielski:1966}
presented a number of additional results, including the fact that there is a game
in which the players play real numbers whose determinacy implies
\emph{uniformization} (see Section \ref{scalesec}) for subsets of the plane, another weak form of the Axiom of Choice.



In 1964, a year after Paul Cohen's\index{Cohen, P.} invention of forcing,
Solovay\index{Solovay, R.} \shortcite{Solovay:1970} proved that if there exists a strongly
inaccessible cardinal, then in a forcing extension there exists an
inner model containing the reals in which every set of reals
satisfies the regularity properties from Section \ref{regularitysection}. Shelah\index{Shelah, S.} \shortcite{Shelah:1984}
later showed that a strongly inaccessible cardinal is necessary, in the sense that the Lebesgue
measurability of all sets of reals implies that $\omega_{1}$ is
strongly inaccessible in all models of the form $L[a]$, for $a
\subseteq \omega$. In the introduction to his
paper, Solovay\index{Solovay, R.} conjectured (correctly, as it turned out) that large
cardinals would imply that AD holds in $L(\mathbb{R})$.

%These two facts together imply that under AD $\omega_{1}$ is a
%strongly inaccessible cardinal in $L$.

%Several modifications of $A_2$ are considered, and some implications
%are proved; for instance, $A_2$ and $A_\omega$ are equivalent to
%each other. D. Scott showed that for a certain cardinal $\kappa$,
%$A_{\kappa}$ is inconsistent, and Scott's argument is modified here
%to show that $A_{\omega_1}$ is inconsistent. Using results of Gödel
%[Proc. Nat. Acad. Sci. U.S.A. 24 (1938), 556--557], it is shown that
%some weaker hypotheses than $A_2$ (e.g., "for every analytic set
%$P\subseteq 2^\omega$, one of the players has a winning strategy for
%the game $G_2(P)$") already contradict Gödel's axiom of
%constructibility.

1967 saw two major results in the study of determinacy, one by
Blackwell\index{Blackwell, D.} \shortcite{Blackwell:1967} and the other by Solovay\index{Solovay, R.}.
Reversing chronological order by a few months, we discuss Blackwell's\index{Blackwell, D.} result and its consequences in the next section, and Solovay's\index{Solovay, R.} in
Section \ref{ppdsec}. These two sections correspond roughly to the
work of the so-called Cabal Seminar, whose proceedings were published in four volumes
covering the years 1976 to 1985 \cite{Cabal:7677, Cabal:7779,
Cabal:7981, Cabal:8185} (at this time, one volume of a planned reissue has appeared \cite{KechrisLoeweSteel}).

\section{Reduction and scales}
%\section{Early descriptive set theory}

Blackwell\index{Blackwell, D.} \shortcite{Blackwell:1967} used open determinacy to reprove
a theorem of Kuratowski\index{Kuratowski, K.} \shortcite{Kuratowski:1936} stating
that the intersection of any two analytic sets $A$, $B$ in
a Polish space $Y$ is also the intersection
of two analytic sets $A'$ and $B'$ such that $A \subseteq A'$, $B \subseteq B'$
and $A' \cup B' = Y$.\footnote{Blackwell describes the discovery of his
proof in \cite[p.~26]{AlbersAlexanderson}.}
Briefly, the argument is as follows.
Since $A$ and $B$ are analytic, there exist continuous surjections $f \colon \breals \to A$ and $g \colon \breals \to B$.
For each finite sequence $\langle n_{0}, \ldots, n_{k}\rangle$, let $\Omega(\langle
n_{0},\ldots, n_{k}\rangle)$ be the set of $x \in \breals$ with $\langle n_{0},\ldots,n_{k}\rangle$
as an initial segment; let
$R(\langle n_{0}, \ldots, n_{k}\rangle)$ be the closure (in $Y$) of the $f$-image of
$\Omega(\langle n_{0},\ldots, n_{k}\rangle)$; and let
$S(\langle n_{0}, \ldots, n_{k}\rangle)$ be the closure of the $g$-image of $\Omega(\langle n_{0},\ldots, n_{k}\rangle)$.
Then for each $z \in Y$, let $G(z)$ be the game where players $I$ and $II$ build $x$ and $y$ in $\breals$, with $I$ winning if for some integer $k$, $z \in R(x \restrict k) \setminus S(y \restrict k)$, $II$ winning if for some integer $k$,
$z \in S(y \restrict k) \setminus R(x \restrict (k+1))$, and the run of the game being a draw if neither of these happens.
Roughly, each player is creating a real ($x$ or $y$) to feed into his function, and trying to maintain for as long as possible that
the corresponding output can be made arbitrarily close to the target real $z$; the loser is the first player to fail to maintain this condition.
Let $A'$ be the set of $z$ for which player $I$ has a strategy guaranteeing at least a draw, and let $B'$
be the set of $z$ for which player $II$ has such a strategy. Then the determinacy of open games implies that
$\breals = A' \cup B'$, and $A \subseteq A'$,  $B \subseteq B'$ and $A' \cap B' = A \cap B$ follow from the fact that $A$ is the range of $f$ and $B$ is the range of $g$. The sets $A'$ and $B'$ are analytic, as $A'$ is a projection of the set of pairs $(\phi, z)$
such that $\phi$ is (a code for) a strategy for $I$ in $G(z)$ guaranteeing at least a draw, which is Borel, and similarly for $B'$.\footnote{A \emph{projection}\index{projection of a set of reals} of a set $A
\subseteq(\breals)^{k}$ (for some integer $k \geq 2$) is a set of the form $\{
(x_{0},\ldots,x_{i-i},x_{i+1},\ldots,x_{k-1}) \mid \exists x_i
(x_{0},\ldots,x_{k-1}) \in A\}$, for some $i < k$.}



%defining for each $y \in \breals$ a game $G(y)$, in which players $I$
%and $II$ build infinite sequences of integers $\langle n_{0}, n_{1}, \ldots \rangle$
%trees $S$ and $T$ on $\omega \times \omega$ such that $A = p[S]$ and
%$B = p[T]$, and defining for each real $x$ two games $G_{1}(x)$ and
%$G_{2}(x)$, in which Player $I$ is assigned $A$ and $S$ and player
%$II$ is assigned $B$ and $T$. In each game, each player plays an
%infinite sequence of integers, trying to maintain for as long as
%possible the existence of an infinite sequence extending his finite
%play so far which, when paired with $x$, defines a path through his
%tree. The first player to fail to maintain this condition loses. If
%neither player fails, $I$ wins in the case of game $G_{0}(x)$
%(making this game closed) and $II$ wins in the case of game
%$G_{1}(x)$ (making this game open). The set $A'$ is the set of $x$
%for which $I$ has a winning strategy in $G_{0}(x)$ and the set $B'$
%is the set of $x$ for which $II$ has a winning strategy in
%$G_{1}(x)$.

\subsection{Reduction, separation, norms and prewellorderings}

In his \shortcite{Kuratowski:1936}, Kuratowski\index{Kuratowski, K.} defined the \emph{reduction theorem}
(now called the \emph{reduction property}\index{reduction property})
for a pointclass $\Gamma$ to be the statement that for any $A$, $B$ in $\Gamma$ there exist
disjoint $A', B'$ in $\Gamma$ with $A' \subseteq A$, $B' \subseteq B$ and $A' \cup
B' = A \cup B$.  He showed in this paper that $\uTPi^{1}_{1}$
and $\uTSigma^{1}_{2}$ have the reduction property; Addison\index{Addison, J.} \shortcite{Addison:1958} showed this for
$\Pi^{1}_{1}(a)$ and $\Sigma^{1}_{2}(a)$, for each real number $a$.
Blackwell's argument proves the reduction property for $\uTPi^{1}_{1}$, working with the corresponding $\uTSigma^{1}_{1}$ complements.

%can be stated as saying that for each real $a$, $\Pi^{1}_{1}(a)$ has
%the reduction property. Kuratowksi had shown this also for
%$\Sigma^{1}_{2}(a)$, and this followed as well from Blackwell's
%argument.

Kuratowski\index{Kuratowski, K.} also defined the \emph{first separation theorem} (now called the \emph{separation property}\index{separation
property}) for a pointclass $\Gamma$ to be the statement that for any disjoint $A$, $B$ in $\Gamma$ there exists $C$ in
$\Delta_{\Gamma}$ with $A \subseteq C$ and
$B \cap C = \emptyset$.
This property had been studied by Sierpi\'{n}ski\index{Sierpi\'{n}ski, W.} \shortcite{Sierpinski:1924}
and Luzin\index{Lusin, N.} \shortcite{Luzin:1930fm}
for initial segments of the Borel hierarchy. Kuratowski\index{Kuratowski, K.} also noted that the reduction property for a pointclass
$\Gamma$ implies the separation property for $\neg\Gamma$.
Luzin\index{Lusin, N.} \shortcite[pp.~51-55]{Luzin:1927} proved that the pointclass $\uTSigma^{1}_{1}$ satisfies the
separation property, by showing that disjoint $\uTSigma^{1}_{1}$ sets are contained in disjoint Borel sets. Petr Novikov\index{Novikov, P.} \shortcite{Novikov:1935} showed that $\uTPi^{1}_{2}$ satisfies the separation property
and $\uTSigma^{1}_{2}$ does not.  Novikov\index{Novikov, P.} \shortcite{Novikov:1935}
(in the case of $\uTSigma^{1}_{2}$ sets)  and Addison\index{Addison, J.} \shortcite{Addison:1958} showed that if
$\Gamma$ satisfies the reduction property and has a so-called \emph{doubly
universal} member, and $\Delta_{\Gamma}$ has no universal member, then $\Gamma$ does not have the
separation property, so $\neg\Gamma$ does not have the reduction
property.\footnote{Members $U$,$V$ of a pointclass $\Gamma$ are \emph{doubly
universal}\index{doubly universal set} for $\Gamma$ if for each pair
$A$,$B$ of members of $\Gamma$ there exist an $x \in \breals$ such
that $U_{x} = A$ and $V_{x} = B$.
%The selfdual projective pointclasses are the
%classes $\Delta^{1}_{n}(a)$, for $n \in \omega$ and $a \in \breals$.
The non-selfdual projective pointclasses (e.g., $\Sigma^{1}_{1}(a)$,
$\Pi^{1}_{1}(a)$, $\Sigma^{1}_{2}(a)$, $\Pi^{1}_{2}(a)$, $\ldots$)
all have doubly universal members.}
%Novikov's proof of the failure of separation for $\uTSigma^{1}_{2}$ used the fact
%that $\uTSigma^{1}_{2}$ has a doubly universal member.
 Addison\index{Addison, J.} \shortcite{Addison:1958, Addison:1959} showed
that if all real numbers are constructible, then the reduction property holds for $\uTSigma^{1}_{k}$, for all $k \geq 2$.

Inspired by Blackwell's argument, Addison\index{Addison, J.} and Martin\index{Martin, D. A.} independently
proved that $\uTDelta^{1}_{2}$-determinacy implies that
$\uTPi^{1}_{3}$ has the reduction property. Since the pointclass $\uTSigma^{1}_{3}$ has a doubly universal
member, this shows that $\uTDelta^{1}_{2}$-determinacy implies the existence of a nonconstructible real. This fact also follows
from G\"{o}del's result (discussed in \cite{Addison:1959}) that the Lebesgue measurability of all $\Delta^{1}_{2}$ sets implies
 the existence of a nonconstructible real. Determinacy would soon
be shown to imply stronger structural properties for the projective
pointclasses.



The key technical idea behind the (pre-determinacy) results listed above on separation and reduction for the first two levels of the projective hierarchy was the notion of \emph{sieve} (in French, \emph{crible}). This construction first appeared in a paper of Lebesgue\index{Lebesgue, H.} \shortcite{Lebesgue:1905}, in which he proved the existence of Lebesgue-measurable sets which are not Borel. In Lebesgue's presentation, a sieve is an association of a closed subset $F_{r}$ of the unit interval $[0,1]$ to each rational number $r$ in this interval. The sieve then represents the set of $x \in [0,1]$ such that $\{ r \mid x \in F_{r}\}$ is wellordered, under the usual ordering of the rationals.
Using this approach, Luzin\index{Lusin, N.} and Sierpi\'{n}ski\index{Sierpi\'{n}ski, W.} \shortcite{LuzinSierpinski:1918, LuzinSierpinski:1923} showed that $\uTSigma^{1}_{1}$ sets and $\uTPi^{1}_{1}$ sets are unions of
$\aleph_{1}$ many Borel sets.
%The representation of sets by sieves is a predecessor of the modern method of representing subsets of the Baire space by trees on the ordinals (finite sequences of natural numbers are used in \cite{Suslin:1917, LuzinSierpinski:1923}).

%Their 1923 introduced
%the method of representing sets of reals by so called \emph{sieves} (\emph{cribles}), a predecessor to
%the modern method of representing subsets of the Baire space by trees. Novikov, Kuratowski\index{Kuratowski, K.} and
%Addison\index{Addison, J.} applied the method of wellordered unions to prove the results mentioned above.

%During the years 1964-1974,

Much of the classical work of Luzin,\index{Lusin, N.} Sierpi\'{n}ski,\index{Sierpi\'{n}ski, W.} Kuratowski\index{Kuratowski, K.} and Novikov\index{Novikov, P.} mentioned here was redeveloped in the lightface context by Kleene \shortcite{Kleene:1943, Kleene:1955ajm, Kleene:1955bams, Kleene:1955tams}, who was unaware of their previous work. The two theories were unified primarily by Addison\index{Addison, J.} (for example, \shortcite{Addison:1959}). While Blackwell's argument generalizes throughout the projective hierarchy, Moschovakis\index{Moschovakis, Y.} (\shortcite{Moschovakis:1967, Moschovakis:1969, Moschovakis:1970, Moschovakis:1971}, see also \shortcite[pp.~202-206]{Moschovakis:DST09}) developed via the effective theory a generalization of the Luzin-Sierpi\'{n}ski
approach (decomposing a set of reals into a wellordered sequence of simpler sets) which could be similarly propagated. Moschovakis's goal was to find a uniform approach to the theory of $\Pi^{1}_{1}$ and $\Sigma^{1}_{2}$;
he was unaware of either Kuratowski's work or determinacy (personal communication).
%according to Moschovakis, he first heard of determinacy on November 3, 1967, when Addison presented his proof of reduction for $\uTPi^{1}_{3}$ at the UCLA logic colloquium)
He extracted the following notions, for a
given pointclass $\Gamma$:
%The \emph{projective
%pointclasses} are the classes $\uTSigma^{1}_{n}$ and
%$\uTPi^{1}_{n}$ for each integer $n$, as well as their
%relativized lightface counterparts $\Sigma^{1}_{n}(a)$ and
%$\Pi^{1}_{n}(a)$.}
a $\Gamma$-\emph{norm}\index{norm} for a set $A$ is a function $\rho\colon A \to
\text{On}$ for which there exist relations $R^{+} \in \Gamma$ and
$R^{-} \in \neg\Gamma$ such that for any $y \in A$,
$$x \in A \wedge \rho(x) \leq \rho(y) \leftrightarrow R^{+}(x,y)
\leftrightarrow R^{-}(x,y);$$ a pointclass $\Gamma$ is said to have
the \emph{prewellordering property} if every $A \in \Gamma$ has a
$\Gamma$-norm.\footnote{A \emph{prewellordering}\index{prewellordering} is a binary relation which is wellfounded, transitive and total.
A function $\rho$ from a set $X$ to the ordinals induces a prewellording $\preceq$ on $X$ by setting $a \preceq b$ if and only if $\rho(a) \leq \rho(b)$. Conversely, a prewellordering $\preceq$ on a set $X$ induces a function $\rho$ from $X$ to the ordinals, where for each $a \in X$,
$\rho(a)$ (the $\preceq$-\emph{rank} of $a$) is the least ordinal $\alpha$ such that $\rho(b) < \alpha$ for all $b \in X$ such that $b \preceq a$ and $a \not\preceq b$. The range of $\rho$ is called the \emph{length} of $\preceq$.}
The prewellordering property was first explicitly formulated
by Moschovakis\index{Moschovakis, Y.} in 1964; the definition just given is a reformulation due to
Kechris.\index{Kechris, A.}
% (see \cite{Moschovakis:DST09}, page 270).
Kuratowski\index{Kuratowski, K.} \shortcite{Kuratowski:1936} and
Addison\index{Addison, J.} \shortcite{Addison:1958} had shown that a variant
of the property
implies the reduction property; the same holds for the prewellordering property as defined by Moschovakis\index{Moschovakis, Y.}.
Moschovakis\index{Moschovakis, Y.} applied Novikov's arguments to show that if $\Gamma$ is a projective pointclass
such that $\forall^{1}\Gamma \subseteq \Gamma$, and $\Gamma$ has
the prewellordering property, then so does the pointclass
$\exists^{1}\Gamma$. Martin\index{Martin, D. A.} and Moschovakis\index{Moschovakis, Y.}
independently completed the picture in 1968, proving what is now known as
the First Periodicity Theorem.

%A pointclass is said to be \emph{continuously closed} if it is
%closed under continuous preimages. If $\Gamma$ is a continuously
%closed pointclass, determinacy for all sets in $\Gamma$ is easily
%seen to imply determinacy for all sets in $\neg\Gamma$.}


\begin{theorem}[First Periodicity Theorem\index{First Periodicity Theorem}]
Let $\Gamma$ be an adequate pointclass and suppose that
$\Delta_{\Gamma}$-determinacy holds. Then for all $A \in \Gamma$,
if\/ $A$ admits a $\Gamma$-norm, then $\forall^{1} A$ admits a
$\forall^{1}\exists^{1}\Gamma$-norm.
\end{theorem}

\begin{corollary}[\cite{AddisonMoschovakis:1968,Martin:1968}\index{Martin, D. A.}]
Let $\Gamma$ be an adequate pointclass closed under existential
quantification over reals, and suppose that $\Delta_{\Gamma}$-determinacy holds. If $\Gamma$ satisfies the prewellordering
property, then so does $\forall^{1}\Gamma$.
\end{corollary}

Projective Determinacy (PD)\index{Projective Determinacy (PD)} is the statement
that all projective sets of reals are determined. By the First Periodicity Theorem,
under Projective Determinacy the following pointclasses have the prewellordering
property, for any real $a$:
$$\Pi^{1}_{1}(a), \Sigma^{1}_{2}(a), \Pi^{1}_{3}(a),
\Sigma^{1}_{4}(a), \Pi^{1}_{5}(a), \Sigma^{1}_{6}(a), \ldots$$ By
contrast (see \cite[pp.~409--410]{Kanamori}), in $L$ the pointclasses with the prewellordering property
are
$$\Pi^{1}_{1}, \Sigma^{1}_{2}, \Sigma^{1}_{3},
\Sigma^{1}_{4}, \Sigma^{1}_{5}, \Sigma^{1}_{6}, \ldots$$




%\section{Reduction, Separation and Scales}



\subsection{Scales}\label{scalesec}


As noted above, the Axiom of Determinacy contradicts the Axiom of Choice, but it
is consistent with, and even implies, certain weak forms of Choice. If $X$ and $Y$ are
nonempty sets and
$A$ is a subset of the product $X \times Y$, a function $f$
\emph{uniformizes} $A$ if the domain of $f$ is the set of $x \in X$ such
that there exists a $y \in Y$ with $(x,y) \in A$, and such that for each
$x$ in the domain of $f$, $(x,f(x)) \in A$. A consequence of the Axiom of Choice, \emph{uniformization}\index{uniformization} is the statement
that for every $A \subseteq \mathbb{R} \times \mathbb{R}$ there is a function $f$ which uniformizes $A$.
%such that, for
%all $x \in \mathbb{R}$, if there exists a $y$ such that $(x,y) \in A$, then $x$ is in the domain of $f$, and
%$(x, f(x)) \in A$ (i.e., $f$ and $A$ have the same projection to the $x$-axis).
Uniformization is not implied by AD, as it
fails in $L(\mathbb{R})$ whenever there are no uncountable wellordered sets of reals \index{Solovay, R.} (\cite{Solovay:1978DCAD};
see Section \ref{gameqsec}).

Uniformization was implicitly introduced by Jacques Hadamard\index{Hadamard, J.} \shortcite{Hadamard:1905}, when he pointed out that the Axiom of Choice should imply the
existence of functions on the reals which disagree everywhere with
every algebraic function over the integers. Luzin\index{Lusin, N.} \shortcite{Luzin:1930cr}
explicitly introduced the notion of uniformization and showed that such functions
exist. He also announced several results on uniformization, including the
fact that all Borel sets (but not all $\underTilde{\Sigma}^{1}_{1}$ sets)
can be uniformized by $\underTilde{\Pi}^{1}_{1}$ functions. The result on
Borel sets was proved independently by Sierpi\'{n}ski\index{Sierpi\'{n}ski, W.}.
Novikov\index{Novikov, P.} \shortcite{LuzinNovikov:1935}
showed that every $\uTSigma^{1}_{1}$ set of pairs has a
$\uTSigma^{1}_{2}$ uniformization.

%another weak form of the Axiom of Choice.

A pointclass $\Gamma$ is said to have the \emph{uniformization property}\index{uniformization property}
if every set of pairs in $\Gamma$ is uniformized by a function in $\Gamma$. Motokiti Kondo\index{Kondo, M.} \shortcite{Kondo:1938} showed
that the pointclasses $\uTPi^{1}_{1}$ and $\uTSigma^{1}_{2}$ have the uniformization property.
The effective version of this result (i.e., for $\Pi^{1}_{1}$ and $\Sigma^{1}_{2}$) was proved by Addison\index{Addison, J.}.
In some sense this is as far as one can go in ZFC : L\'{e}vy\index{Levy, A.} \shortcite{Levy:1965def} would show that consistently there exist $\Pi^{1}_{2}$ sets that cannot be uniformized by any projective function.




%A pointclass $\Gamma$ has the \emph{uniformization
%property}\index{uniformization property} if every $A \in \Gamma$ which is a subset of a product is uniformized by an $f$ in $\Gamma$.
After studying Kondo's proof, Moschovakis\index{Moschovakis, Y.} in 1971 isolated a property
for sets of reals which induces uniformizations.
%by coding a tree on
%the ordinals projecting to the set in question.
Given a set $A$ and an ordinal $\gamma$, a
\emph{scale}\index{scale (on a set of reals)}
(or a $\gamma$-scale) on $A$ into $\gamma$ is a sequence of
functions $\rho_{n}\colon A \to \gamma$ $(n \in \omega)$ such that
whenever
\begin{itemize}
\item $\{ x_{i} \mid i \in \omega\} \subseteq A$ and
$\lim_{i\to\omega}x_{i} = x$, and
\item the sequence $\langle \rho_{n}(x_{i}) : i \in \omega \rangle$ is eventually constant for each $n \in \omega$,
\end{itemize}
then $x \in A$ and, for every $n \in \omega$, $\rho_{n}(x)$ is less
than or equal to the eventual value of $\langle \rho_{n}(x_{i}) : i
\in \omega \rangle$. The scale is a
\emph{$\Gamma$-scale}\index{scale (on a set of reals)!$\Gamma$-scale}
if there exist $R^{+} \in \Gamma$ and $R^{-} \in \neg\Gamma$ such that for all $y
\in A$ and all $n \in \omega$,
$$x \in A \wedge \rho_{n}(x) \leq \rho_{n}(y) \leftrightarrow
R^{+}(n,x,y) \leftrightarrow R^{-}(n,x,y).$$ A pointclass $\Gamma$
has the \emph{scale property}\index{scale property} if every $A$ in
$\Gamma$ has a $\Gamma$-scale.
Moschovakis\index{Moschovakis, Y.} \shortcite{Moschovakis:1971} proved the following three theorems about the scale property.

\begin{theorem} If\/ $\Gamma$ is an adequate pointclass, $A \in \Gamma$, and $A$ admits a $\Gamma$-scale, then
$\exists^{1} A$ admits a $\exists^{1}\forall^{1}\Gamma$-scale.
\end{theorem}

\begin{theorem}[Second Periodicity Theorem\index{Second Periodicity Theorem}]
Suppose that $\Gamma$ is an adequate pointclass such that
$\Delta_{\Gamma}$-determinacy holds. Then for all $A \in \Gamma$, if\/
$A$ admits a $\Gamma$-scale, then $\forall^{1}A$ admits a
$\forall^{1}\exists^{1}\Gamma$-scale.
\end{theorem}

\begin{theorem} Suppose that $\Gamma$ is an adequate pointclass which is closed under integer quantification.
Suppose that $\Gamma$ has the scale property, and that $\Delta_{\Gamma}$-determinacy holds. Then $\Gamma$ has
the uniformization property.
\end{theorem}




%If $\Gamma$ is an adequate pointclass such that $\exists^{1}\Gamma \subseteq \Gamma$,
%%closed under existential
%%universal ??
%%real quantification,
%and $\Gamma$ has the uniformization property, then the set of
%subsets of $\omega$ in $\Gamma \cap \neg \Gamma$ forms a \emph{basis} for $\Gamma$, i.e., every
%nonempty set in $\Gamma$ contains subset of $\omega$ in
%$\Gamma$.\index{pointclass!basis for a}



Kondo's proof of uniformization for $\uTPi^{1}_{1}$ shows that $\Pi^{1}_{1}(a)$ has the scale property for every
real $a$ (see \cite[p.~419]{Kanamori}). It follows that under
$\underTilde{\Delta}^{1}_{2n}$-determinacy, $\uTPi^{1}_{2n+1}$ and
$\uTSigma^{1}_{2n+2}$ have the scale property, and every
$\uTPi^{1}_{2n+1}$ relation on the reals can be uniformized by a
$\uTPi^{1}_{2n+1}$ relation (and similarly for $\uTSigma^{1}_{2n+2}$).
Furthermore, under Projective Determinacy, for any real $a$, the
projective pointclasses with the scale property are the same as
those with the prewellordering property: $\Pi^{1}_{1}(a)$,
$\Sigma^{1}_{2}(a)$, $\Pi^{1}_{3}(a)$, $\Sigma^{1}_{4}(a)$,
$\Pi^{1}_{5}(a)$, $\Sigma^{1}_{6}(a)$, etc..





A \emph{tree}\index{tree} on a set $X$ is a collection of finite sequences from $X$ closed under initial segments. Given sets  $X$ and $Z$, a positive integer $k$ and a tree
$T$ on $X^{k} \times Z$, the \emph{projection}\index{projection of a tree} of $T$, $p[T]$\index{$p[T]$}, is the set of
$x \in (X^{\omega})^{k}$ such that for some $z \in Z^{\omega}$, $(x \restrict
n, z \restrict n) \in T$ for all $n \in \omega$ (strictly speaking, this definition involves the identification of finite sequences of $k$-tuples with $k$-tuples of finite sequences). If one substitutes the Baire space $\breals$ for $\mathbb{R}$, Suslin's\index{Suslin, M.} construction for analytic sets (see Section \ref{defsec}) essentially presents them as projections of trees on $\omega \times \omega$, modulo the representation of closed intervals. Many descriptive set theorists, starting perhaps with Luzin and Sierpi\'{n}ski \shortcite{LuzinSierpinski:1923}, used trees to represent sets of reals, except that they converted these trees to linear orders via what is now known as the Kleene-Brouwer ordering (after \cite{Brouwer:1924}\index{Brouwer, L.} and \cite{Kleene:1955ajm}\index{Kleene, S.}). The explicit use of projections of trees as we have presented them here is due to Richard Mansfield\index{Mansfield, R.} \shortcite{Mansfield:1970}. As pointed out in \cite{KechrisMoschovakis:1978}, given an ordinal $\gamma$, a $\gamma$-scale for a subset $A$ of the Baire space naturally gives rise to a tree on $\omega \times \gamma$ such that $p[T] = A$. Given a set $Z$, a subset of the Baire space is said to be
$Z$-\emph{Suslin}\index{Suslin set!$Z$-Suslin} if it is the projection of a tree on $\omega \times Z$. Suslin's representation of analytic sets shows that a set is analytic if and only if it is $\omega$-Suslin. Some authors use ``Suslin'' to mean ``analytic''. We will follow a different usage, however, and say that a subset of the Baire space is \emph{Suslin}\index{Suslin set}
if is $\gamma$-Suslin for some ordinal $\gamma$.




Given a tree $T$ on $\omega \times Z$ and a
wellordering of $Z$, a member of $p[T]$ can be found by following
the so-called \emph{leftmost} infinite branch through $T$ (similar
to the proof of K\H{o}nig's Lemma, one picks a path through the tree
by taking the least next step which is the initial segment of an
infinite path through the tree). In a similar manner, a tree on $(\omega \times \omega) \times \gamma$, for
some ordinal $\gamma$, induces a uniformization of the projection of the tree.






\subsection{The game quantifier}\label{gameqsec}

Given a Polish space $X$ and a set $B \subseteq X \times \breals$, we let $\game B$ denote the set of $x \in X$ such that $I$ has a winning
strategy in $G_{\omega}(B_{x})$. If $\Gamma$ is a pointclass, $\game
\Gamma$ is the class $\{ \game B \mid B \in \Gamma\}$. The following facts
appear in \cite[pp.~245-246]{Moschovakis:DST09}.

\begin{theorem}\label{gameprop} If\/ $\Gamma$ is an adequate pointclass then the following hold.
\begin{itemize}
\item $\game \Gamma$ is adequate and closed under $\exists^{0}$ and $\forall^{0}$.
\item $\exists^{1}\Gamma \subseteq \game \Gamma$ and $\forall^{1}\Gamma \subseteq \game \Gamma$.
\item If {\rm Det}$(\Gamma)$ holds, then $\game \Gamma \subseteq \forall^{1}\exists^{1}\Gamma$.
\end{itemize}
\end{theorem}


The First Periodicity
Theorem can be stated more generally as the fact that if an adequate
pointclass $\Gamma$ has the prewellordering property, then so does
$\game\Gamma$, and the Second Periodicity Theorem can be similarly
stated as saying that if an adequate pointclass $\Gamma$ has the
scale property, then so does $\game\Gamma$ (see \cite[pp.~246,267]{Moschovakis:DST09}).
The propagation of these properties through the
projective pointclasses then follows from Theorem \ref{gameprop}, given that
they hold for $\uTPi^{1}_{1}$ (and its variants).

Modifying the notion of $\Gamma$-scale by dropping the requirement that $\rho_{n}(x)$ is less than or equal to
the eventual value of $\langle \rho_{n}(x_{i}) : i \in \omega
\rangle$, one gets the notion of
$\Gamma$-\emph{semiscale}\index{scale (on a set of reals)!$\Gamma$-semiscale}. Moschovakis's\index{Moschovakis, Y.}
\emph{Third Periodicity Theorem} \shortcite{Moschovakis:1973} concerns the
definability of winning strategies and is stated using the game
quantifier and the notion of semiscale.

\begin{theorem}[Third Periodicity Theorem]
Let $\Gamma$ be an adequate pointclass, and suppose that {\rm Det}$(\Gamma)$ holds.
Fix $A \subseteq \breals$ in $\Gamma$, and suppose that $A$ admits a
$\Gamma$-semiscale and that $I$ has a winning strategy in the game
$G_{\omega}(A)$. Then $I$ has a winning strategy coded by a subset of $\omega$ in $\game \Gamma$.
\end{theorem}

One consequence the Third Periodicity Theorem in conjunction with Theorem \ref{gameprop}
is the following \cite{Moschovakis:1973}: for
any $n \in \omega$, if $\uTSigma^{1}_{2n}$-determinacy
holds, $A \subseteq\omega^{\omega}$ is $\Sigma^{1}_{2n}(a)$ for some
real $a$ and $I$ has a winning strategy in the game with payoff $A$,
then $I$ has a winning strategy coded by a subset of $\omega$ in $\Delta^{1}_{2n+1}(a)$.




Let $\game^{1}$ denote the game quantifier for \emph{real games}, games of length
$\omega$ where the players alternate playing real numbers. Then $\game^{1}\uTSigma^{0}_{1}$ defines the
\emph{inductive}\index{inductive set of reals} sets of reals.\footnote{Formally, this definition requires
a definable association of $\omega$-sequences of reals to individual reals. Alternately,
a set of reals is inductive if
it is in $\Sigma_{1}^{J_{\kappa_{\mathbb{R}}}(\mathbb{R})}$, where $J$
%%\index{$J_{\alpha}$}
refers to Ronald Jensen's\index{Jensen, R.} constructibility hierarchy and
$\kappa_{\mathbb{R}}$ is the least $\kappa$ such that
$J_{\kappa}(\mathbb{R})$ is a model of Kripke-Platek
set theory.}
Moschovakis\index{Moschovakis, Y.} \shortcite{Moschovakis:1978} showed that the
inductive sets have the scale property. Moschovakis
\shortcite{Moschovakis:1983} showed that, assuming the determinacy of all
games with payoff in the class built from the inductive sets by the
operations of projection and complementation, coinductive sets have
scales in this class. Building on this work, Martin\index{Martin, D. A.} and Steel\index{Steel, J.}
\shortcite{MartinSteel:1983} showed that the pointclass $\uTSigma^{2}_{1}$ has the scale property in $L(\mathbb{R})$.
%$\Sigma_{1}^{L(\mathbb{R})}(\mathbb{R} \cup \{ \mathbb{R}\})$
%has the scale property. In
%$L(\mathbb{R})$, the $\Sigma_{1}$ sets of reals are exactly the
%pointclass $\uTSigma^{2}_{1}$.
Kechris and Solovay\index{Solovay, R.} had shown that if
there is no wellordering of the reals in $L(\mathbb{R})$, then there
exists in $L(\mathbb{R})$ a set of reals that cannot be uniformized,
the set of pairs $(x,y)$ such that $y$ is not ordinal definable from
$x$ (i.e., definable from $x$ and some ordinals). This set is
%%%%%SOLOVAY15,16: In the previous line, "i.e.," was inserted, "$y$" was changed to "$x$" and "a finite set of" was changed to "some".
$\Pi^{2}_{1}$ in $L(\mathbb{R})$.

The Solovay\index{Solovay, R.} Basis Theorem says that if $P(A)$ is a $\Sigma^{2}_{1}$
relation on subsets of $\breals$ and there exists a witness to
$P(A)$ in $L(\mathbb{R})$, then there is a $\Delta^{2}_{1}$ witness.
This reflection result, along with the Martin-Steel\index{Martin, D. A.}\index{Steel, J.} theorem on
scales in $L(\mathbb{R})$, compensates in many circumstances for the
fact that not every set of reals has a scale in $L(\mathbb{R})$.

Steel\index{Steel, J.} \shortcite{Steel:1983} applied Jensen's\index{Jensen, R.} fine structure theory
\shortcite{Jensen:1972} to the study of scales in $L(\mathbb{R})$,
refining and unifying a great deal of work on scales and Suslin
cardinals. Extending \cite{MartinSteel:1983}\index{Steel, J.}, he showed that for
each positive ordinal $\alpha$, determinacy for all sets of reals in
$J_{\alpha}(\mathbb{R})$ implies that the pointclass
$\Sigma_{1}^{J_{\alpha}(\mathbb{R})}$ has the scale property.



Martin\index{Martin, D. A.} \shortcite{Martin:1983real} showed how
to propagate the scale property using the game quantifier for integer games
of fixed countable length (this subsumes propagation by the quantifier $\game^{1}$), and Steel\index{Steel, J.}
\shortcite{Steel:1988,Steel:longopen} did the same for certain games of length $\omega_{1}$.





\subsection{Partially playful universes}

The periodicity theorems showed that determinacy axioms imply structural properties for sets of reals
beyond the classical regularity properties. It remained to show that these hypotheses were necessary.
Towards this end, Moschovakis\index{Moschovakis, Y.} (see \cite{Becker:1978}) identified for each integer $n$ (under the
assumption of $\uTDelta^{1}_{k}$-determinacy, where $k$ is the
greatest even integer less than $n$) the smallest transitive
$\Sigma^{1}_{n}$-correct model of ZF + Dependent Choices (DC)
which contains all the ordinals (Joseph Shoenfield\index{Shoenfield, J.} \shortcite{Shoenfield:1961} had shown
that $L$ is $\Sigma^{1}_{2}$-correct).\footnote{The Axiom of
Dependent Choices (DC)\index{Axiom of Dependent Choices}\index{DC}
is the statement that if $R$ is a binary
relation on a nonempty set $X$, and if for each $x \in X$ there is a
$y \in X$ such that $x R y$, then there exists an infinite sequence
$\langle x_{i} : i < \omega \rangle$ such that $x_{i} R x_{i+1}$ for
all $i \in \omega$. This statement is a weakening of the Axiom of
Choice, sufficient to prove K\H{o}nig's Lemma, the regularity of $\omega_{1}$ and the
wellfoundedness of ultrapowers by countably complete ultrafilters. See
\cite{Jech:settheory}.}  This model satisfies AC and
$\uTDelta^{1}_{k}$-determinacy and has a $\Sigma^{1}_{n+1}$
wellordering of the reals. In this model, $\Pi^{1}_{i}$ has the
scale property for all odd $i \leq n$, and $\Sigma^{1}_{i}$ has the
scale property for all other positive integers $i$.

%Shoenfield\index{Shoenfield, J.} \cite{Shoenfield:1961} showed that the Novikov-Kondo\index{Novikov, P.}
%construction yields for each $\Sigma^{1}_{2}$ set an
%$\omega_{1}$-Suslin representation in $L$. Since the wellfoundedness
%of trees is absolute between wellfounded models of ZF, it follows
%that $L$ is $\Sigma^{1}_{2}$ correct.

Kechris\index{Kechris, A.} and Moschovakis\index{Moschovakis, Y.} \shortcite{KechrisMoschovakis:1978}
introduced the models $L[T_{2n+1}]$, where $T_{2n+1}$
denotes the tree for a $\Pi^{1}_{2n+1}$-scale for a complete
$\Pi^{1}_{2n+1}$ set. Moschovakis\index{Moschovakis, Y.} showed that $L[T_{1}] = L$, and conjectured that
$L[T_{2n+1}]$ is independent of
the choice of complete set and scale when for all $n$. This conjecture was proved by Howard Becker\index{Becker, H.} and Kechris in
\shortcite{BeckerKechris:1984}.


Solovay\index{Solovay, R.} \shortcite{Solovay:1969} showed that if $L \cap \mathbb{R}$ is countable, then it is
the largest countable $\Sigma^{1}_{2}$ set of reals (i.e., a countable $\Sigma^{1}_{2}$ set which contains all other such sets). Kechris\index{Kechris, A.} and Moschovakis\index{Moschovakis, Y.} \shortcite{KechrisMoschovakis:1972} showed that for each positive integer $n$,
if Det$(\Delta^{1}_{2n})$ holds then there exists a largest countable $\Sigma^{1}_{2n+2}$ set. The largest countable $\Sigma^{1}_{2n}$ set came to be called $\mathcal{C}_{2n}$. Kechris\index{Kechris, A.} \shortcite{Kechris:1975} showed that under
Projective Determinacy there is for each integer $n$ a largest countable
$\Pi^{1}_{2n+1}$ set, which he also called $\mathcal{C}_{2n+1}$. The case $n=0$ follows from ZF + DC and was shown independently by David Guaspari,
Kechris and Gerald Sacks \shortcite{Sacks:1976}. Kechris also showed that under Projective Determinacy there are no largest countable $\Sigma^{1}_{2n+1}$ or $\Pi^{1}_{2n}$ sets. It follows that under Projective Determinacy the lightface
projective pointclasses with a largest countable set are the same as those in the zig-zag pattern above for the prewellording property and the scale property. Harrington\index{Harrington, L.} and Kechris\index{Kechris, A.}
\shortcite{HarringtonKechris:1981} showed (under the assumption that AD holds in $L(\mathbb{R})$) that the reals of each
$L[T_{2n+1}]$ are exactly $\mathcal{C}_{2n+2}$, for all integers $n$ (the case $n=1$ was due to Kechris and Martin).

Kechris\index{Kechris, A.} showed (assuming Projective Determinacy) that each model $L[\mathcal{C}_{2n}]$ satisfies Det$(\uTDelta^{1}_{2n-1})$ but not Det$(\uTSigma^{1}_{2n-1})$, and has a $\Delta^{1}_{2n}$ wellordering of its reals. Martin\index{Martin, D. A.} would show that Det$(\uTDelta^{1}_{2n})$ implies Det$(\uTSigma^{1}_{2n})$ for each positive integer $n$.


%Kechris and Moschovakis \cite{KechrisMoschovakis:1972, Kechris:1975}
%noted that  These largest countable sets were studied in
%\cite{Martin:1983} and \cite{GuaspariHarrington:1976} as well.
%Kechris and Moschovakis showed that the models $L[C_{2n}]$ are
%absolute for $\Sigma^{1}_{2n}$ relations, satisfy
%$\Delta^{1}_{2n-1}$-determinacy if $V$ does and have
%$\Sigma^{1}_{2n}$-good wellorderings of the reals, and so do not
%satisfy $\Sigma^{1}_{2n-1}$-determinacy.

%Letting $T_{2n+1}$ denote the tree for a $\Pi^{1}_{2n+1}$-scale for
%a complete $\Pi^{1}_{2n+1}$ set, Becker and Kechris
%\cite{BeckerKechris:1984} showed that the model $L[T_{2n+1}]$ is
%independent of the choice of complete set and corresponding scale.
%Kechris and Martin \cite{KechrisMartin:1978} has shown earlier than
%the reals of $T_{2n+1}$ are exactly $C_{2n+2}$.

%In one of the more striking developments, the sets $C_{n}$ were to
%turn out to be the reals of the minimal $n$-Woodin inner models (see

%The models $M^{n}$ and $L[T_{2n+1}]$ appear again in Section
%\ref{emmlt}.

\subsection{Wadge degrees}

In 1968, William Wadge\index{Wadge, W.} considered the following game, given two sets of
reals $A$ and $B$: $I$ builds a real $x$, $II$ builds a real $y$,
and $II$ wins if $x \in A \leftrightarrow y \in B$. Determinacy for
this class of games is known as \emph{Wadge determinacy}. Given two
sets of reals $A$, $B$, we say that $A\leq_{W} B$ ($A$ has
\emph{Wadge rank} less than or equal to $B$, or is \emph{Wadge
reducible} to $B$) if there is a continuous function $f$ such that
for all reals $x$, $x \in A$ if and only if $f(x) \in B$ (i.e., such that $A =
f^{-1}[B]$).  Wadge
determinacy implies that for any two sets of reals $A$, $B$, either
$A \leq_{W} B$ (in the case that $II$ has a winning strategy) or $\omega^{\omega} \setminus B \leq_{W} A$
(in the case that $I$ does), from which it follows that for any two
pointclasses closed under continuous preimages, either the two classes are dual (i.e., a pair of the form
$\Gamma$, $\neg\Gamma$) or one is contained in
the other. Wadge\index{Wadge, W.}
showed that $\leq_{W}$ is wellfounded on the Borel sets, and Martin\index{Martin, D. A.}, using an 
idea of Leonard Monk,\index{Monk, L.} using AD + DC, extended this to all sets of reals under 
AD + CD (see \cite{VanWesep:1978Wadge}).
%%%%%SOLVAY: The previous sentence was revised in light of comments from Martin and Solovay. 

Wadge determinacy and the wellfoundedness of the Wadge hierarchy
divide $\mathcal{P}(\omega^{\omega}))$ into equivalence
%%%%%SOLOVAY18: One "$\mathcal{P}(...)$" was removed from the previous line.
classes by Wadge reducibility and order these classes into a wellfounded
hierarchy, where each level
consists either of one selfdual equivalence class, or two non-selfdual classes,
one consisting of all the complements of the members of the other.
Wadge determinacy also implies that every non-selfdual adequate
pointclass has a universal set (see \cite[p.~162]{VanWesep:1978Wadge}).

The discovery of Wadge determinacy led to further progress on separation
and reduction. Robert Van Wesep\index{Van Wesep, R.}
\shortcite{VanWesep:1978separation} proved that under AD,
if $\Gamma$ is a non-selfdual pointclass which is closed under continuous preimages, then $\Gamma$ and $\neg\Gamma$ cannot both have the
separation property.
%In a paper from the 1977-79 Cal Tech-UCLA Logic Seminar Proceedings,
%\cite{Cabal:7779},
Kechris,\index{Kechris, A.} Solovay\index{Solovay, R.} and Steel\index{Steel, J.} \shortcite{KechrisSolovaySteel:1981} showed that under AD +
DC, if $\Gamma \subseteq L(\mathbb{R})$ is nonselfdual boldface pointclass and $\Gamma$ is
closed under countable intersections and unions and either $\exists^{1}$ or $\forall^{1}$, but not complements,
then either $\Gamma$ or $\neg\Gamma$ has the prewellordering property.
%Examples of such classes include the
%projective pointclasses $\underTilde{\Sigma}^{1}_{n}$ and $\underTilde{\Pi}^{1}_{n}$ for each $n \in \omega$.
In 1981, Steel\index{Steel, J.} \shortcite{Steel:1981} showed that under AD, if $\Gamma$ is a
nonselfdual pointclass closed under continuous preimages, then either $\Gamma$ or
$\neg\Gamma$ has the separation property, and if one assumes in
addition that $\Delta_{\Gamma}$ is closed under finite unions, then
either $\Gamma$ or $\neg\Gamma$ has the reduction property.



\section{Partition properties and the projective ordinals}\label{ppdsec}


A cardinal $\kappa$ is \emph{measurable}\index{measurable cardinal} if there is
a nonprincipal $\kappa$-complete ultrafilter on $\kappa$, where
$\kappa$-\emph{completeness}\index{ultrafilter!$\kappa$-complete}
means closure under intersections of fewer than $\kappa$ many elements. In
ZFC measurable cardinals are strongly inaccessible. In 1967, Solovay\index{Solovay, R.} (see \cite[p.~633]{Jech:settheory} or \cite[p.~348]{Kanamori}) showed that AD implies that the club filter on $\omega_{1}$ is an ultrafilter, which
implies
that $\omega_{1}$ is a measurable cardinal.\footnote{A subset of an ordinal is
\emph{closed unbounded} (or \emph{club})\index{closed unbounded set}
if it is unbounded and closed in the order topology on the ordinals,
and \emph{stationary}\index{stationary set} if it
intersects every club set. The
\emph{club filter}\index{closed unbounded filter}\index{club filter}
on an ordinal $\gamma$ consists of all subsets of $\gamma$
containing a club set.} Ulam had shown that under ZFC there are
stationary, co-stationary subsets of $\omega_{1}$; Solovay's\index{Solovay, R.} result
shows the opposite under AD. Solovay\index{Solovay, R.} also showed that under AD every
subset of $\omega_{1}$ is constructible from a real (i.e., exists in
$L[a]$ for some real number $a$). Since the measurability of
$\omega_{1}$ implies that the sharp of each real exists, this gives
another proof that the club filter on $\omega_{1}$ is an
ultrafilter, since for any real $a$, if $a^{\#}$ exists, then every
subset of $\omega_{1}$ in $L[a]$ either contains or is disjoint from
a tail of the $a$-indiscernibles below $\omega_{1}$, which is a club
set.

A \emph{Turing degree} is a nonempty subset of $\mathcal{P}(\omega)$ closed under
equicomputability. A \emph{cone}\index{cone
of Turing degrees} of Turing degrees is the set of all degrees above
(or computing) a given degree.\footnote{See \cite{Soare:1987, Cooper:2004}
for more on the Turing degrees, including a more precise statement of their definition.}
Martin\index{Martin, D. A.} \shortcite{Martin:1968} showed that under AD the cone measure on
Turing degrees is an ultrafilter, i.e., that every set of Turing degrees either
contains or is disjoint from a cone. This important fact has a
relatively short and simple proof: the two players collaborate to
build a real, with the winner decided by whether the Turing degree
of the real falls inside the payoff set; the cone above the degree
of any real coding a winning strategy must contain or be disjoint
from the payoff set. Martin\index{Martin, D. A.} used this result to find a simpler
proof of the measurability of $\omega_{1}$. Solovay\index{Solovay, R.} followed by showing that
$\omega_{2}$ is measurable as well.
%%%%%SOLOVAY19:The previous sentence was rewritten according to Solovay's recollection. The original follows.
%%%%%Solovay\index{Solovay, R.} used Martin's\index{Martin, D. A.} result to find a simpler
%%%%%proof of the measurability of $\omega_{1}$, and showed that
%%%%%$\omega_{2}$ is measurable as well.
\emph{Turing
determinacy}\index{Turing determinacy} is the restriction of AD to
payoff sets closed under Turing equivalence. This form of
determinacy is easily seen to suffice for Martin's\index{Martin, D. A.} result.
%and its
%corresponding restrictions can be used to prove the First
%Periodicity Theorem.
In the early 1980's, Woodin\index{Woodin, W. H.} would show that,
in $L(\mathbb{R})$, AD and Turing determinacy are equivalent.



Given an ordered set $X$ and an ordinal $\beta$, $[X]^{\beta}$
denotes the set of subsets of $X$ of ordertype $\beta$. Given
ordinals, $\alpha$, $\beta$, $\delta$ and $\gamma$, the expression
$\alpha \to (\beta)^{\gamma}_{\delta}$ denotes the statement that
for every function $f \colon [\alpha]^{\gamma} \to \delta$, there
exists an $X \in [\alpha]^{\beta}$ such that $f$ is constant on
$[X]^{\gamma}$. Frank Ramsey\index{Ramsey, F. P.} \shortcite{Ramsey:1930} proved that $\omega \to (\omega)^{n}_{2}$
holds for each positive $n \in \omega$ (this fact is known as
\emph{Ramsey's Theorem}).  For infinitary partitions, Paul Erd\H{o}s\index{Erd\H{o}s, P.} and
Andr\'{a}s Hajnal\index{Hajnal, A.} \shortcite{ErdosHajnal:1966} showed (in ZFC) that for any infinite
cardinal $\kappa$ there is a function $f \colon [\kappa]^{\omega} \to
\kappa$ such that for every $X \in [\kappa]^{\kappa}$, the range of
$f \restrict X$ is all of $\kappa$.

In 1968, Adrian Mathias\index{Mathias, A.} \shortcite{Mathias:1968, Mathias:Happy} showed that $\omega \to
(\omega)^{\omega}_{2}$ holds in Solovay's\index{Solovay, R.} model from
\shortcite{Solovay:1970}, in which all sets of reals satisfy the regularity properties. A set $Y \subseteq [\omega]^{\omega}$ is
said to be \emph{Ramsey}\index{Ramsey property} if there exists an $X \in
[\omega]^{\omega}$ such that either $[X]^{\omega} \subseteq Y$ or
$[X]^{\omega} \cap Y = \emptyset$. The statement $\omega \to
(\omega)^{\omega}_{2}$ is equivalent to the statement that every
subset of $[\omega]^{\omega}$ is Ramsey.
%every $Y \subseteq[\omega]^{\omega}$ is Ramsey.\footnote{Given an
%ordered set $X$ and an ordinal $\beta$, $[X]^{\beta}$ denotes the
%set of subsets of $X$ of ordertype $\beta$; when $X$ is a set of
%integers, $[X]^{\omega}$ denotes the set of infinite subsets of $X$.
%A set $Y \subseteq[\omega]^{\omega}$ is \emph{Ramsey}\index{Ramsey
%property} if there exists an $X \in [\omega]^{\omega}$ such that
%either $[X]^{\omega} \subseteq Y$ or $[X]^{\omega} \cap Y =
%\emptyset$. The expression $\alpha \to (\beta)^{\gamma}_{\delta}$
%denotes the statement that for every function $f \colon
%[\alpha]^{\gamma} \to \delta$, there exists an $X \in
%[\alpha]^{\beta}$ such that $f$ is constant on $[X]^{\gamma}$. The
%statement that every subset of $[\omega]^{\omega}$ is Ramsey can
%then be expressed as $\omega \to (\omega)^{\omega}_{2}$.} Once
%again, this contradicts the Axiom of Choice.
Prikry \shortcite{Prikry:1976} showed that under AD$_{\mathbb{R}}$
(determinacy for games of perfect information of length $\omega$ for
which the players play real numbers)\index{AD$_{\mathbb{R}}$} every subset of $[\omega]^{\omega}$ is Ramsey.
It follows from the main theorem of \cite{MartinSteel:1983} that AD + $V\mathord{=}L(\mathbb{R})$ implies that every such set is Ramsey.
Whether AD alone suffices is still an open question.

In late 1968, Martin\index{Martin, D. A.} (see \cite[p.~392]{Kanamori}) showed that AD implies $\omega_{1} \to
(\omega_{1})^{\omega}_{2}$ (this implies for instance that the club filter on $\omega_{1}$ is an ultrafilter). Kenneth Kunen\index{Kunen, K.} then showed that AD
implies that $\omega_{1}$ satisfies the weak partition
property, where a cardinal $\kappa$ satisfies the \emph{weak
partition property}\index{weak partition property} if $\kappa \to (\kappa)^{\alpha}_{2}$ holds for
every $\alpha < \kappa$. Martin followed by showing that $\omega_{1} \to (\omega_{1})^{\omega_{1}}_{2}$, again under AD.
The proof actually shows $\omega_{1}
\to (\omega_{1})^{\omega_{1}}_{2^{\omega}}$ and $\omega_{1} \to
(\omega_{1})^{\omega_{1}}_{\alpha}$ for every countable ordinal
$\alpha$.
Martin\index{Martin, D. A.} and Paris (in an unpublished note
\shortcite{MartinParis:1971}, see \cite{Kechris:1978}) showed that under AD + DC,
$\omega_{2}$ has the weak partition property.

%In unpublished work from 1971, Kunen and Martin extended
%these results to the ordinals $\utdelta^{1}_{n}$ (defined in Section \ref{projo}) for $n \in \omega$ (see \cite{Kechris:1978}).

Before continuing with this line of results, we briefly discuss the Coding Lemma and the projective ordinals.


\subsection{$\Theta$, the Coding Lemma and the projective ordinals}\label{thetaprojo}

Following convention, we let $\Theta$\index{$\Theta$} denote the least ordinal that
is not a surjective image of $\mathbb{R}$. Under ZFC, $\Theta = \mathfrak{c}^{+}$, but under AD, $\Theta$ is a
limit cardinal, as noted by Harvey Friedman\index{Friedman, H.} (see \cite[p.~398]{Kanamori}).
This fact follows from a
theorem known as the \emph{Coding Lemma}\index{Coding Lemma}, due to
Moschovakis\index{Moschovakis, Y.} \shortcite{Moschovakis:1970}, extending earlier work of
Friedman\index{Friedman, H.} and Solovay.\index{Solovay, R.}

Given a subset $P$ of some Polish space, let $\uTSigma^{1}_{1}(P)$ denote the
pointclass of sets which are $\Sigma^{1}_{1}$-definable using $P$
and individual reals as parameters.

%\begin{theorem}[Coding Lemma] Assume ZF + AD. Let $\mathfrak{X}$ and $\mathfrak{Y}$ be finite products of spaces from the set $\{ \breals, \omega\}$, let $X$ be a subset of $\mathfrak{X}$, and let $\xi$ be an ordinal. Suppose that $\preceq$ be a prewellordering of $X$ of length $\xi$, and let
%$g \colon \xi \to \mathcal{P}(\mathfrak{Y})$ such that
%\begin{itemize}
%\item $\{ (x,y) \in \mathfrak{X} \times \mathfrak{Y} \mid x \in X \text{ and } y \in g(\phi(x))\}$ is $\uTSigma^{1}_{1}(\preceq)$;
%\item for each $\eta < \xi$, $g(\eta) \subseteq f(\eta)$, and $g(\eta) = \emptyset \Leftrightarrow f(\eta) = \emptyset$.
%\end{itemize}
%\end{theorem}

\begin{theorem}[Coding Lemma] Assume {\rm ZF + AD}. Let $\preceq$ be a prewellordering of a set of reals $X$. Let $\xi$ be the length
of $\preceq$ and let $A$ be a subset of $\xi$. Then there exists a
$Y \subseteq X$ in $\uTSigma^{1}_{1}(\preceq)$ such that $A$ is
the set of $\preceq$-ranks of elements of $Y$.
\end{theorem}



As an immediate consequence, under AD, if $\xi < \Theta$, then there is a surjection from
$\mathbb{R}$ onto $\mathcal{P}(\xi)$ (furthermore, if $\alpha <
\Theta^{M}$ for some wellfounded model $M$ of ZF containing the reals, then such a surjection can be found in
$M$). The proof of the Coding Lemma uses a version of Kleene's Recursion Theorem\index{Kleene, S.}
(first proved in \cite{Kleene:1938} for partial recursive functions on the integers), which can be stated as saying that
given a suitable coding under which each real $x$ codes a continuous partial function $\hat{x}$ (our notation) on the reals,
for each two-variable continuous partial function $g$ on the reals there is a real $x$ such that $\hat{x}(w) = g(x,w)$
for all reals $w$.
%%%%%SOLOVAY20: I inserted "partial" after "two-variable continuous", and changed "partial continuous" to "continuous partial".

If $\Gamma$ is a pointclass,
$\delta_{\Gamma}$  denotes the
supremum of the lengths of the prewellorderings of the reals in
$\Delta_{\Gamma}$.
%In \cite{KechrisSolovaySteel:1981}, Kechris, Solovay and Steel
%showed that whenever $\Gamma$.....
The notation $\undertilde{\delta}^{1}_{n}$ is used to denote
$\delta_{\uTSigma^{1}_{n}}$ (which is the same as
$\delta_{\uTPi^{1}_{n}}$). The \emph{projective
ordinals}\index{projective ordinal}\index{$\delta^1_n$1@$\boldsymbol{\delta}^1_n$ (= $\undertilde{\delta}^1_n$)}
are the ordinals $\utdelta^{1}_{n}$, for $n \in \omega \setminus \{0\}$.
It follows from the results of \cite{LuzinSierpinski:1923} that $\uTSigma^{1}_{1}$ prewellorderings of the
reals have countable length, and therefore that the ordinal $\undertilde{\delta}^{1}_{1}$ is
equal to $\omega_{1}$. Moschovakis\index{Moschovakis, Y.} \shortcite{Moschovakis:1970}
showed (under AD, using the Coding Lemma) that for each $n \in
\omega$, $\undertilde{\delta}^{1}_{n+1}$ is a cardinal, and that
$\undertilde{\delta}^{1}_{2n+1}$ is regular and (using just PD) strictly less
than $\undertilde{\delta}^{1}_{2n+2}$. Martin\index{Martin, D. A.} showed (without AD)
that $\undertilde{\delta}^{1}_{2} \leq \omega_{2}$ (see \cite{KechrisMoschovakis:1978}); together these
results show that under AD, $\undertilde{\delta}^{1}_{2} =
\omega_{2}$.


Kunen\index{Kunen, K.} and Martin\index{Martin, D. A.}
(see \cite{KechrisMoschovakis:1978})
independently established from ZF + DC that every
wellfounded $\kappa$-Suslin prewellordering has length less than
$\kappa^{+}$ (this fact is sometimes called the \emph{Kunen-Martin
Theorem}\index{Kunen-Martin Theorem}). Moschovakis\index{Moschovakis,
Y.} (\shortcite{Moschovakis:1970}; see
\cite[4C.14]{Moschovakis:DST09}) showed (from PD) that any
$\uTPi^{1}_{2n+1}$-norm on a complete $\uTPi^{1}_{2n+1}$ set has
length $\undertilde{\delta}^{1}_{2n+1}$ (this result also uses
Kleene's Recursion Theorem).
%Kechris\index{Kechris, A.} and Moschovakis\index{Moschovakis, Y.}  used the
By the scale property for $\uTPi^{1}_{2n+1}$ sets (under the assumption of
DC + $\uTDelta^{1}_{2n}$-determinacy, given $n \in \omega$ \cite{Moschovakis:1971}),
every $\uTPi^{1}_{2n+1}$ set (and thus every
$\uTSigma^{1}_{2n+2}$ set) is
$\undertilde{\delta}^{1}_{2n+1}$-Suslin, and, since $\utdelta^{1}_{2n+1}$ is regular,
every $\uTSigma^{1}_{2n+1}$ set is $\lambda$-Suslin for some $\lambda <
\undertilde{\delta}^{1}_{2n+1}$.
%(though Kechris\index{Kechris, A.} in
%\cite{Kechris:1978} credits this to Moschovakis\index{Moschovakis, Y.}
%\cite{Moschovakis:1971}).
It follows that under the same hypothesis,
%$\delta^{1}_{2n+1}$ is a successor cardinal and
$\undertilde{\delta}^{1}_{2n+2} \leq
(\undertilde{\delta}^{1}_{2n+1})^{+}$, and under AD that
$\undertilde{\delta}^{1}_{2n+2} =
(\undertilde{\delta}^{1}_{2n+1})^{+}$ for each $n \in \omega$.

%Using
%the existence of a universal set for $\uTSigma^{1}_{2n+1}$ it also
%follows

Kechris\index{Kechris, A.} \shortcite{Kechris:1974} proved (assuming AD) that $\undertilde{\delta}^{1}_{2n+1}$ is a successor
cardinal (its predecessor is called
$\lambda_{2n+1}$).
It follows from his arguments, and those of
the previous paragraph, that the pointclasses
$\uTSigma^{1}_{2n+2}$ and $\uTSigma^{1}_{2n+1}$ are exactly the
$\undertilde{\delta}^{1}_{2n+1}$-Suslin and $\lambda_{2n+1}$-Suslin
sets respectively.

Given an ordinal $\lambda$, the $\lambda$-Borel sets of reals are those in the smallest class containing the
open sets and closed under complements and well-ordered unions of length less than $\lambda$.
Martin\index{Martin, D. A.} showed that if $\kappa$ is a cardinal of uncountable cofinality, then
all $\kappa$-Suslin sets are $\kappa^{+}$-Borel. He also showed (using AD + DC, the
Coding Lemma and Wadge determinacy) the $\utdelta^{1}_{2n+1}$-Borel sets are $\uTDelta^{1}_{2n+1}$, for each $n \in \omega$ (the
reverse inclusion follows from the results of Moschovakis\index{Moschovakis, Y.} \shortcite{Moschovakis:1971} mentioned above).
Using this fact, Kechris\index{Kechris, A.} proved (again, under AD) that
$\lambda_{2n+1}$ has cofinality $\omega$.
%Kunen and Martin showed that (under PD),
%$\undertilde{\delta}^{1}_{2n+2} \leq
%(\undertilde{\delta}^{1}_{2n+1})^{+}$.
%Combined with Moschovakis's result, this shows that
%$\undertilde{\delta}^{1}_{2n+2} =
%(\undertilde{\delta}^{1}_{2n+1})^{+}$ for all $n \in \omega$.
It follows (under AD) that $\undertilde{\delta}^{1}_{2n} <
\undertilde{\delta}^{1}_{2n+1}$ for each $n \in \omega$, so that
under AD the sequence $\langle \undertilde{\delta}^{1}_{n+1} : n
\in \omega \rangle$ is a strictly increasing sequence of successor cardinals. Kunen\index{Kunen, K.} \shortcite{Kunen:1971a} showed that
$\utdelta^{1}_{n}$ is regular for each positive $n \in \omega$.



%A subset of $\omega^{\omega}$ is said to be $\kappa$-Suslin if it is

%the projection of a tree\footnote{Given a set $X$, a \emph{tree on}
%$X$ is a subset of $X^{\less\omega}$ which is closed under finite
%sequences. Given an ordinal $\alpha$ and a tree $T$ on $\omega
%\times \alpha$, the \emph{projection of} $T$, $p[T]$, is the set
%$$\{x \in \omega^{\omega} \exists y \in \alpha^{\omega} \forall n
%\in \omega (x \restrict n, y \restrict n) \in T\}.$$} on $\omega
%\times \kappa$.

%From the  one then gets that all $\uTPi^{1}_{2n+1}$ sets are
%$\undertilde{\delta}^{1}_{2n+1}$-Suslin, and therefore that all
%$\uTSigma^{1}_{2n+2}$ sets are
%$\undertilde{\delta}^{1}_{2n+1}$-Suslin.

%Moschovakis\index{Moschovakis, Y.}  showed (under DC +
%$\uTDelta^{1}_{2}$-determinacy) that all $\uTDelta^{1}_{2n+1}$ sets
%are $\undertilde{\delta}^{1}_{2n+1}$-Borel. Martin\index{Martin, D. A.} (using Wadge
%determinacy) proved the other direction under DC + AD, giving (under
%this hypothesis) that the $\uTDelta^{1}_{2n+1}$ sets of reals are
%exactly the $\undertilde{\delta}^{1}_{2n+1}$-Borel sets.

Solovay\index{Solovay, R.} noted that under AD, $\Theta$ is the
$\Theta$-th cardinal, and that under the further assumption of $V
\mathord{=} L(\mathbb{R})$, $\Theta$ is regular (see
\cite[p.~398]{Kanamori}). He showed \shortcite{Solovay:1978DCAD}
that under DC, $\Theta$ has uncountable cofinality, and also that
ZFC + AD$_{\mathbb{R}}$ + cf$(\Theta) > \omega$ proves the
consistency of ZF + AD$_{\mathbb{R}}$, so that by G\"{o}del's Second
Incompleteness Theorem, if ZF + AD$_{\mathbb{R}}$ is consistent,
then so is ZFC + AD$_{\mathbb{R}}$ + cf$(\Theta) = \omega$.
Kechris\index{Kechris, A.} \shortcite{Kechris:1984}, using the proof
of the Third Periodicity Theorem and work of Martin,\index{Martin, D. A.}
Moschovakis\index{Moschovakis, Y.} and Steel\index{Steel, J.}
on scales \cite{MartinMoschovakisSteel}, showed that DC follows from
AD + $V\mathord{=}L(\mathbb{R})$. Woodin\index{Woodin, W. H.} (see
\cite{Kechris:1984}) strengthened Solovay's\index{Solovay, R.}
result that DC does not follow from AD by showing that, assuming AD
+ $V\mathord{=}L(\mathbb{R})$ there is an inner model of a forcing
extension satisfying  ZF + AD + $\neg$AC$_{\omega}$ (DC
directly implies AC$_{\omega}$). Whether AD implies
DC$(\breals$) (DC for relations on $\breals$) \index{DC$(\breals)$}
is still open.

\subsection{Partition properties and ultrafilters}\label{ppmeas}

Kunen in an unpublished note \index{Kunen, K.} \shortcite{Kunen:1971} proved that
$\undertilde{\delta}^{1}_{2n} \to (\undertilde{\delta}^{1}_{2n})^{\lambda}_{2}$ for all positive $n \in \omega$ and $\lambda < \omega_{1}$,
under AD. He also showed \shortcite{Kunen:1971b} (under the same hypothesis)  that $\undertilde{\delta}^{1}_{2n} \to (\utdelta^{1}_{2n})^{\utdelta^{1}_{2n}}_{2}$ is false.
Martin\index{Martin, D. A.}, in another
unpublished note from 1971, showed that $\undertilde{\delta}^{1}_{2n+1} \to (\undertilde{\delta}^{1}_{2n+1})^{\lambda}_{2}$ for all positive $n \in \omega$ and $\lambda < \omega_{1}$,
under AD.

%Section \ref{projo} discusses the computation of the values of the ordinals $\utdelta^{1}_{n}$.

While Erd\H{o}s\index{Erd\H{o}s, P.} and Hajnal\index{Hajnal, A.} \shortcite{ErdosHajnal:1958} had
shown how to derive partition properties from measurable cardinals,
Eugene Kleinberg\index{Kleinberg, E.} proved the following result in the other direction,
which shows (via $\lambda = \omega$) that $\undertilde{\delta}^{1}_{n}$ is measurable for each
positive $n \in \omega$.\footnote{We let
$C^{\lambda}_{\kappa}$\index{$C^{\lambda}_{\kappa}$} denote the
filter generated by the set of $\lambda$-closed unbounded
subsets of $\kappa$. A filter is \emph{normal}\index{filter!normal}
if every regressive function on a set in the filter is constant on a
set in the filter.}

\begin{theorem}[\cite{Kleinberg:1970}\index{Kleinberg, E.}] If\/ $\lambda < \kappa$,
$\lambda$ is regular, and $\kappa
\to (\kappa)^{\lambda + \lambda}_{2}$ holds, then $C^{\lambda}_{\kappa}$
is a normal ultrafilter over $\kappa$.
\end{theorem}

%\begin{theorem} Suppose that $\kappa$ is not weakly Mahlo, and for
%each regular $\lambda < \kappa$ $C^{\lambda}_{\kappa}$ is a normal
%ultrafilter. Then these are the only normal ultrafilters over
%$\kappa$.
%\end{theorem}



%Martin\index{Martin, D. A.} showed in 1973 that AD implies $\omega_{1} \to
%(\omega_{1})^{\omega_{1}}_{2}$.
%In conjunction with Kleinberg's theorem, these two
%results reprove Solovay's results that $\omega_{1}$ and $\omega_{2}$
%are measurable.
%Martin showed that each $\delta^{1}_{2n+1}$ is
%measurable, and Kunen showed that $\delta^{1}_{2n+2}$ is measurable
%(see \cite{Kechris:1978}).



In 1970, Kunen\index{Kunen, K.} proved, using Martin's result on
the cone measure on the Turing degrees,
that under AD, any $\omega_{1}$-complete filter on an ordinal
$\lambda < \Theta$ can be extended to an $\omega_{1}$-complete
ultrafilter, and that every
ultrafilter on an ordinal less than $\Theta$ is definable from ordinal parameters (see \cite[pp.~399-400]{Kanamori}). Solovay\index{Solovay, R.} \shortcite{Solovay:1978DCAD} proved that under
AD$_{\mathbb{R}}$, there is a normal ultrafilter on
$\mathcal{P}_{\aleph_{1}}\mathbb{R}$: for each $A \subseteq \mathcal{P}_{\aleph_{1}}\mathbb{R}$,
consider the game where $I$ and $II$ collaborate to build a sequence $\langle s_{i} : i < \omega \rangle$ consisting of finite sets of reals, and $I$ wins if and only if
$\bigcup\{ s_{i} : i \in \omega\} \in A$.\footnote{Given a cardinal $\kappa$ and a set $X$,
$\mathcal{P}_{\kappa} X $\index{$\mathcal{P}_\kappa X $}
denotes the collection of subsets of $X$ of cardinality less than $\kappa$.  An ultrafilter $U$ on $\mathcal{P}_{\kappa} X$ is \emph{normal}\index{normal ultrafilter on $\mathcal{P}_{\kappa} X$} if
for each $Y \in U$, if $f$ is a regressive function on $Y$ (i.e., if $\dom(f) = Y$ and $f(A) \in A$ for all nonempty $A \in Y$) then
$f$ is constant on a set in $U$.} This implies (again, under AD$_{\mathbb{R}}$) that for each ordinal $\gamma < \Theta$ there is a normal ultrafilter on
$\mathcal{P}_{\aleph_{1}} \gamma$. It
is not known whether AD suffices for this result, though Harrington and Kechris \shortcite{HarringtonKechris:1981} showed that if AD holds and $\gamma$ is less than a Suslin cardinal,
then there is a normal ultrafilter on $\mathcal{P}_{\aleph_{1}} \gamma$.\footnote{An ordinal (necessarily a cardinal) $\kappa$ is said to be \emph{Suslin} if there is a set of reals which is $\kappa$-Suslin
but not $\lambda$-Suslin for any $\lambda < \kappa$.}
%Becker\index{Becker, H.} \cite{Becker} showed that under AD, if $\lambda$ is a Suslin cardinal, then there is a unique such measure on $\lambda$.
Extending work of Becker\index{Becker, H.}, Woodin\index{Woodin, W. H.}
\shortcite{Woodin:1983ADuniqueness} showed that there is just one such
ultrafilter for each $\gamma < \Theta$, if either AD$_{\mathbb{R}}$ holds or AD holds and $\gamma$ is below a Suslin cardinal.

%A cardinal $\kappa$ is \emph{Suslin} if there is a set of reals
%which is $\kappa$-Suslin but not $\gamma$-Suslin for any $\gamma <
%\kappa$.

%Steel\index{Steel, J.} \cite{Steel:1983} proved that under AD + $V\mathord{=}L(\mathbb{R})$, the Suslin cardinals are closed below their
%supremum. Woodin\index{Woodin, W. H.} later improved the hypothesis to AD. By results of
%Kechris\index{Kechris, A.}, Martin\index{Martin, D. A.} and Moschovakis\index{Moschovakis, Y.}, the Suslin cardinals below the
%supremum of the $\undertilde{\delta}^{1}_{n}$'s are exactly the
%cardinal predecessors of the $\undertilde{\delta}^{1}_{2n+1}$'s.

%Kunen\index{Kunen, K.} (see \cite{Kechris:1978}) showed that under AD,





%The following version of the Coding Lemma is
%taken from \cite{KoellnerWoodin:handbook}.



%For each such $P$ and each
%positive $n \in \omega$, let $U^{(n)}(P) \subseteq(\breals)^{n+1}$
%denote a universal set for subsets of $(\breals)^{n}$ in
%$\uTSigma^{1}_{1}(P)$.

%\begin{theorem}[Coding Lemma] Assume ZF + AD. Fix $X \subseteq\breals$, $\pi \colon X \to On$ and $Z \subseteq X\times \breals$.
%There exists an $e \in \breals$ such that
%\begin{itemize}
%\item $U^{(2)}_{e}(Q) \subseteq Z$
%\item for each $a \in X$, $U^{(2)}_{e}(Q) \cap (Q_{a} \times
%\breals) \neq \emptyset$ if and only if $Z \cap (Q_{a} \times
%\breals) \neq \emptyset$,
%\end{itemize}
%where $Q = \{ \langle a, b \rangle \mid \pi(a) \leq \pi(b) \}$ and,
%for each $a \in X$, $Q_{a} = \{ b \in X \mid \pi(a) = \pi(b)\}$.
%\end{theorem}



%for every
%function $f \colon \breals \to \breals$ in $\uTSigma^{1}_{1}(Q)$
%(where $Q$ is as in the statement of the Coding Lemma) there is an
%$e \in \breals$ such that $U^{(2)}_{e}(Q) = U^{(2)}_{f(e)}(Q)$.

%An
%immediate consequence of the Coding Lemma is that under AD, if
%$\alpha < \Theta$, then there is a surjection from the reals onto
%the powerset of $\alpha$ .






A cardinal $\kappa$ is said to have the \emph{strong partition
property}\index{strong partition property} if $\kappa \to
(\kappa)^{\kappa}_{\mu}$ holds for every $\mu < \kappa$. As mentioned
above, Martin\index{Martin, D. A.} showed that under AD, $\omega_{1}$ has the strong
partition property. In late 1977, Kechris\index{Kechris, A.} adapted
Martin's\index{Martin, D. A.} argument
to show that under AD there exists a cardinal $\kappa$ with the
strong partition property such that the set of $\lambda < \kappa$
with the strong partition property is stationary below $\kappa$ (see \cite[p.~432]{Kanamori}).
Pushing this further, Kechris,\index{Kechris, A.} Kleinberg\index{Kleinberg, E.}, Moschovakis\index{Moschovakis, Y.} and Woodin\index{Woodin, W. H.}
\shortcite{KechrisKleinbergMoschovakisWoodin:1981} showed (using a uniform version of the
Coding Lemma) that AD implies that unboundedly many cardinals below
$\Theta$ have the strong partition property and are stationary limits of cardinals with the strong
partition property. They also showed that whenever $\lambda$ is an ordinal below
a cardinal with the strong partition property, all
$\lambda$-Suslin sets are determined. Using work of Steel\index{Steel, J.}
\shortcite{Steel:1983} and Martin\index{Martin, D. A.} \shortcite{Martin:1983real}, Kechris\index{Kechris, A.} and Woodin\index{Woodin, W. H.}
\shortcite{KechrisWoodin:1983} showed that in $L(\mathbb{R})$, AD is
equivalent to the assertion that $\Theta$ is a limit of cardinals
with the strong partition property, and also to the statement that all Suslin sets are determined.
James Henle,\index{Henle, J.} Mathias\index{Mathias, A.}
and Woodin\index{Woodin, W. H.} \shortcite{HenleMathiasWoodin:1985} later showed that the first
equivalence does not follow from ZF + DC, since the existence of a nonprincipal
ultrafilter on $\omega$ is consistent with $\Theta$ being a limit of
cardinals with the strong partition property.

A key step in the proof of the Kechris-Woodin\index{Kechris, A.}\index{Woodin, W. H.} theorem was
a transfer theorem extending results of Harrington\index{Harrington, L.} and
Martin\index{Martin, D. A.}
(discussed in Section \ref{dhier}). Harrington\index{Harrington, L.} and Martin\index{Martin, D. A.} had shown from ZF + DC that,
%%%%%SOLOVAY21: "They" changed to "Harrington and Martin" on the previous line.
for each real $a$, $\Pi^{1}_{1}(a)$-determinacy is equivalent to determinacy for the larger class $\bigcup_{\beta < \omega^{2}} \beta$-$\Pi^{1}_{1}(a)$.
Kechris\index{Kechris, A.} and Woodin\index{Woodin, W. H.} showed, from the same hypothesis, that for all positive integers $k$, $\uTDelta^{1}_{2k}$-determinacy
is equivalent to $\game^{(2k-1)}\bigcup_{\beta < \omega^{2}} \beta$-$\uTPi^{1}_{1}$-determinacy, where $\game^{(2k-1)}$ indicates
an application of $2k-1$ many instances of the game quantifier $\game$. By Theorem \ref{gameprop}, this means that
$\uTDelta^{1}_{2k}$-determinacy implies $\uTPi^{1}_{2k}$-determinacy. Martin had proved the lightface version in 1973 (see \cite{KechrisSolovay:1985}).
Later results of Woodin\index{Woodin, W. H.} and Itay Neeman\index{Neeman, I.} \shortcite{Neeman:1995}
would show that $\uTPi^{1}_{n+1}$-determinacy is equivalent to $\game^{(n)}\bigcup_{\beta < \omega^{2}} \beta$-$\uTPi^{1}_{1}$-determinacy
for all $n \in \omega$.



\subsection{Cardinals, uniform indiscernibles and the projective ordinals}\label{projo}

%Kunen\index{Kunen, K.} started a project of
%computing the values of the projective ordinals.


A cardinal $\kappa$ is \emph{Ramsey}\index{Ramsey
cardinal} if for every function $f\colon [\kappa]^{\less\omega} \to
\{0,1\}$ (where $[\kappa]^{\less\omega}$
%%\index{$[\kappa]^{\less\omega}$}
denotes the finite subsets of $\kappa$) there exists $A \in [\kappa]^{\kappa}$
such that for each $n \in \omega$, $f \restrict [\kappa]^{n}$ is
constant. Measurable cardinals are Ramsey, and if there exists a
Ramsey cardinal then the sharp of each real number exists. Assuming the existence of a Ramsey cardinal, Martin\index{Martin, D. A.} and
Solovay\index{Solovay, R.} \shortcite{MartinSolovay:1969} showed that nonempty
$\uTSigma^{1}_{3}$ subsets of the plane have $\uTDelta^{1}_{4}$
uniformizations.
%and that $\Delta^{1}_{4}$ is a basis for
%$\uTSigma^{1}_{3}$.
As mentioned above, L\'{e}vy\index{Levy, A.} \shortcite{Levy:1965def} had shown that ZFC does not suffice for this result. Martin\index{Martin, D. A.} and Solovay\index{Solovay, R.} used an analysis of sharps for reals,
and modeled their argument after the proof of the Kondo-Addison theorem.
%used indiscernibles to produce
%a tree projecting to the complement of a complete $\Sigma^{1}_{2}$
%set (i.e., a complete $\Pi^{1}_{2}$ set), which induces a tree
%projecting to a complete $\Sigma^{1}_{3}$ set.
Mansfield\index{Mansfield, R.}
\shortcite{Mansfield:1971} extended the Martin-Solovay analysis to show (using a measurable cardinal) that nonempty $\uTPi^{1}_{2}$ sets are uniformized by $\uTPi^{1}_{3}$ functions.
%and that the class
%of $\Pi^{1}_{3}$ subsets of $\omega$ is a basis for $\uTPi^{1}_{2}$.
%(these results are not optimal ???).


Given a positive ordinal $\alpha$,
$u_{\alpha}$ denotes the $\alpha$th \emph{uniform indiscernible}\index{uniform indiscernible},
the $\alpha$th ordinal which is a Silver indiscernible for each real number. As bijections between $\omega$ and countable
ordinals can be coded by reals, the first uniform indiscernible,
$u_{1}$, is $\omega_{1}$. It follows from the basic analysis of sharps that all uncountable cardinals are uniform indiscernibles, so
$u_{2} \leq \omega_{2}$.
%The arguments of \cite{Shoenfield:1961} can be used to show that $\utSigma^{1}_{2}$ sets are $\omega_{1}$-Suslin, and
By applying the Kunen-Martin theorem inside models of the form $L[a]$, for $a$ a real number, and applying the basic analysis of sharps, Martin\index{Martin, D. A.} showed that $\utdelta^{1}_{2} = u_{2}$ if the sharp of every real exists (see \cite{Kechris:1978}).  Recall that by the results of Section \ref{thetaprojo}, $\utdelta^{1}_{2} = \omega_{2}$, under AD.

%$\undertilde{\delta}^{1}_{2}$ is less than or equal to $u_{2}$ so they are the same under AD.
%Kunen\index{Kunen, K.} and Martin\index{Martin, D. A.} showed that under the same
%hypothesis, $\undertilde{\delta}^{1}_{2}$ is greater than or equal
%to $u_{2}$, and thus equal to it.
Martin\index{Martin, D. A.} showed from ZF plus the
assumption that the sharp of each real exists that every
$\uTSigma^{1}_{3}$ set is $u_{\omega}$-Suslin, and from AD that
$u_{\omega} = \omega_{\omega}$ (see \cite[pp.203-204]{Kanamori}). By the Kunen-Martin Theorem, then,
AD implies that
$\utdelta^{1}_{3} \leq \omega_{\omega + 1}$. Solovay had shown that if the sharp of every real exists, then
$u_{\xi + 1}$ has the same cofinality as $u_{2}$, for every positive ordinal $\xi$ (see \cite{Kechris:1978}). Since $u_{\omega} = \omega_{\omega}$, it follows
that each $\omega_{n}$ $(n \geq 2)$ is of the form $u_{k + 1}$ for some positive integer $k$, and thus that each such $\omega_{n}$ has cofinality $\omega_{2}$. It follows that
under AD + DC, $\utdelta^{1}_{3} = \omega_{\omega + 1}$, since $\utdelta^{1}_{3}$ is a regular cardinal, and therefore that $\utdelta^{1}_{4} = \omega_{\omega + 2}$. Kunen\index{Kunen, K.} and Solovay\index{Solovay, R.}
would then show that $u_{n} = \omega_{n}$ for all $n$ satisfying $1
\leq n \leq \omega$.



%Martin deduced also that under AD, $u_{\omega} = \omega_{\omega}$.



%Martin\index{Martin, D. A.} showed, using the Coding Lemma and his result that
%$\undertilde{\delta}^{1}_{2n+1}$ sets are $\uTDelta^{1}_{2n+1}$,
%that under AD, $u_{\omega} = \omega_{\omega}$. From this it followed
%that each member of $\{\omega_{n} : 3 \leq n < \omega\}$ is a member
%of $\{u_{n} : 3 \leq n < \omega\}$, and Solovay\index{Solovay, R.} had shown that each
%member of this set has cofinality $\omega_{2}$.

%No member of $\{\omega_{n} : 3 \leq n < \omega\}$ is measurable
%under AD, however. By results of Martin and Solovay, they all have
%cofinality $\omega_{2}$.

%Kechris\index{Kechris, A.} \cite{Kechris:1974} showed that for each $n \in \omega$,
%$\utdelta^{1}_{2n+1}$ is the cardinal successor of an cardinal of
%cofinality $\omega$, and $\utdelta^{1}_{2n+2} < \utdelta^{1}_{2n+3}$.



In 1971, Kunen\index{Kunen, K.} reduced the computation of
$\undertilde{\delta}^{1}_{5}$ to the analysis of certain ultrapowers
of $\undertilde{\delta}^{1}_{3}$ (see \cite{Kechris:1978}; as part of his analysis, Kunen showed that $\utdelta^{1}_{3}$
has the weak partition property, see \cite{Solovay:1978}). The
completion of this project was to take another decade. In the early
1980's, Martin\index{Martin, D. A.} proved new results analyzing these ultrapowers,
and Steve Jackson,\index{Jackson, S.} using joint work with Martin,\index{Martin, D. A.} computed
$\undertilde{\delta}^{1}_{5}$. The following theorem
\cite{Jackson:1988, Jackson:1999} completes the calculation of the
$\undertilde{\delta}^{1}_{n}$'s.


\begin{theorem}[Jackson\index{Jackson, S.}] Assume {\rm AD}. Then for $n\geq 1$,
$\undertilde{\delta}^{1}_{2n+1}$ has the strong partition property and is equal to $\omega_{w(2n-1) + 1}$, where $w(1)
= \omega$ and $w(m+1) = \omega^{w(m)}$ in the sense of ordinal
exponentiation.
\end{theorem}

%Jackson\index{Jackson, S.} \shortcite{Jackson:1988, Jackson:1999} also showed that under AD the statement
%$$\undertilde{\delta}^{1}_{2n+1} \to
%(\undertilde{\delta}^{1}_{2n+1})^{\undertilde{\delta}^{1}_{2n+1}}_{2}$$
%holds for all $n \in \omega$.

Jackson's proof of this theorem was over 100 pages long. Elements of his argument (as presented in \cite{Jackson:1999}) include the Kunen-Martin theorem, Kunen's $\uTDelta^{1}_{3}$
coding for subsets of $\omega_{\omega}$ \cite{Solovay:1978}, Martin's theorem that $\uTDelta^{1}_{2n+1}$ is closed under
intersections and unions of sequences of sets indexed by ordinals less than $\utdelta^{1}_{3}$, and so-called homogeneous trees, a notion which traces back to \cite{MartinSolovay:1969} and a result of Martin discussed in the next section.




%Martin\index{Martin, D. A.} showed that, assuming AD, a set of reals is homogeneously
%Suslin if both it and its complement are $\Theta$-Suslin (see
%\cite{MartinSteel:1989}).

%Martin\index{Martin, D. A.} and Woodin\index{Woodin, W. H.} (see
%\cite{MartinWoodin:weaklyhomogeneous}) showed that under AD, if
%$\kappa$ is less than the supremum of the Suslin cardinals, then
%every tree on $\omega \times \kappa$ is weakly homogeneous (Martin\index{Martin, D. A.}
%originally proved the theorem under the assumption of
%AD$_{\mathbb{R}})$.)



%Martin showed in the early 1980's that, under AD$_{\mathbb{R}}$, every tree on a set of the from $\omega \times Z$ is weakly homogeneous (see \cite{MartinWoodin:weaklyhomogeneous}).

%The main construction of \cite{MartinSolovay:1969} can be used to show that if $A$ is a homogeneously Suslin subset of the Baire space, then $\breals \setminus A$ is Suslin.  showed, under ...., that if $A$ is homogeneously Suslin then $A$ and its complement are Suslin.
%Martin and Steeel showed the converse \cite{MartinSteel2008}.



\section{Determinacy and large cardinals}

As discussed above, a strongly inaccessible cardinal is an uncountable regular cardinal which is closed under cardinal exponentiation.
%%%%%SOLOVAY25: "uncountable" inserted in the previous line.
If $\kappa$ is strongly inaccessible, then $V_{\kappa}$ is a model of ZFC, so that the existence of strongly inaccessible cardinals is not a consequence of ZFC. While there is no technical definition of \emph{large cardinal},
%%\index{large cardinal} \
a typical large cardinal notion
(in the context of the Axiom of Choice) specifies a type of strongly inaccessible cardinal. Examples of this type include Ramsey cardinals, measurable cardinals, Woodin cardinals and supercompact cardinals. The \emph{large cardinal hierarchy}
%%\index{large cardinal hierarchy}
orders large cardinals by \emph{consistency strength}.
That is, large cardinal notion $A$ is below large cardinal notion $B$ in the hierarchy if the existence of cardinals of type $B$ implies the consistency of cardinals of type $A$. It is a striking empirical fact that the large cardinal hierarchy is linear, modulo open questions (the examples just given were listed in increasing order, for instance). Even more striking is the fact that many set-theoretic statements having no ostensible relationship to large cardinals are equiconsistent with some large cardinal notion.\footnote{\cite{Kanamori} is the standard reference for the large cardinal hierarchy.}

By results of Mycielski\index{Mycielski, J.} (discussed in Section \ref{adsubsec}), AD implies that $\omega_{1}$ is strongly inaccessible in $L$, which means that AD cannot be proved in ZFC. Moreover,
Solovay's\index{Solovay, R.} result that AD implies the measurability of $\omega_{1}$ implies that under AD, $\omega_{1}$ (as computed in the full universe) is a measurable cardinal in certain inner models of AC, such as HOD.\footnote{The inner model HOD\index{HOD} (a model of ZFC) consists of all sets $x$ such that every member of the transitive closure of $\{x\}$ is ordinal-definable (see \cite[Chapter 13]{Jech:settheory}).} As we shall see in this section, the relationship between large cardinals and determinacy runs in both directions: various forms of determinacy imply the existence of models of ZFC containing large cardinals, and the existence of large cardinals can be used to prove the determinacy of certain definable sets of reals.




\subsection{Measurable cardinals}

Solovay\index{Solovay, R.} \shortcite{Solovay:1969} showed in 1965 that if there exists a measurable
cardinal then every uncountable $\uTSigma^{1}_{2}$ set of reals contains a perfect set. This result was
proved independently by Mansfield\index{Mansfield, R.} (see \cite{Solovay:1969}).
%\footnote{This paper also proves
%the remarkable fact that the perfect set property for $\underTilde{\Pi}^{1}_{1}$ sets
%(also $\underTilde{\Sigma}^{1}_{2}$ sets) is equivalent to the statement
%that each real constructs only countably many other reals.}
Martin\index{Martin, D. A.} \shortcite{Martin:1970} showed that in fact analytic determinacy
follows from the existence of a Ramsey cardinal.

Roughly, the idea behind Martin's\index{Martin, D. A.} proof is that if $A$ is the
projection of a tree $T$ on $\omega \times \omega$ and $\chi$ is a Ramsey cardinal,
one can modify the original game for $A$
to require the second player to play, in addition to his usual moves, a function $G^{*} \colon \omega^{\less\omega} \to \chi$
witnessing (via the wellfoundedness of the ordinal $\chi$) that the fragment of $T$ corresponding to the real
produced by the two players in their moves from the original game has no infinite branches, and thus that this real is not in the projection of $T$.
This modified game is closed, and thus
determined, by Gale-Stewart.\index{Gale, D.}\index{Stewart, F.} If the second player has a winning
strategy in the modified game, then he has a winning strategy in the
original game by ignoring his extra moves. In general there is no
reason that a winning strategy for the first player in the modified game will
induce a winning strategy for the original game. However, if $\chi$ is a Ramsey cardinal, then there is
uncountable $X \subseteq \chi$ such that, as long as the range of $G^{*}$  is contained in $X$,
the first player's strategy does not depend on the extra moves for the second player.
Using this fact, the first player can convert his winning strategy in the modified game into a winning
strategy in the original game. The notion of a determined (often closed) auxiliary game and
a method for transferring strategies from the auxiliary game to the
original game is the basis of many determinacy proofs.

Martin\index{Martin, D. A.}
later proved the following refinement.
\begin{theorem}\label{pi11det}
If the sharp of every real exists, then $\uTPi^{1}_{1}$-determinacy
holds.
\end{theorem}

In the 1970's Kunen\index{Kunen, K.} and Martin\index{Martin, D. A.} independently developed the notion of a \emph{homogeneous} tree, following
a line of ideas deriving from Martin's proof of $\uTPi^{1}_{1}$-determinacy (see \cite{Kechris:1981}).
%Very roughly, a tree is homogeneous membership in the projection of the tree is witnessed by the wellfoundedness of an ultrapower by
%an associated sequence of measures. This idea had its roots in
%\cite{MartinSolovay:1969}, as well as work of Kunen\index{Kunen, K.} (see \cite{Kechris:1981}).
Given a set $Z$ and a cardinal $\kappa$, a tree on $\omega \times Z$ is said to be $\kappa$-\emph{homogeneous}\index{homogeneous tree}\index{tree!homogeneous} if for each
$\sigma \in \omega^{\less\omega}$ there is a $\kappa$-complete
ultrafilter $\mu_{\sigma}$ on $Z^{|\sigma|}$ such that
\begin{itemize}
\item for each $\sigma \in \omega^{\less\omega}$, $\{ z \mid (\sigma, z)
\in T \} \in \mu_{\sigma}$;
\item $p[T]$ is the set of $x \in \omega^{\omega}$ such that the
sequence $\langle \mu_{x \restrict i} : i \in \omega \rangle$ is
countably complete.\footnote{i.e., for each sequence $\langle A_{i}
\mid i \in \omega \rangle$ such that each $A_{i} \in \mu_{x
\restrict i}$ there exists a $t \in Z^{\omega}$ such that $t
\restrict i \in A_{i}$ for each $i$.}
\end{itemize}
A tree is said to be
\emph{homogeneous}\index{homogeneously Suslin tree}\index{tree!homogeneously Suslin}
if it is $\aleph_{1}$-homogeneous. A set of reals is said to be
\emph{homogeneously Suslin}\index{homogeneously Suslin set}\index{Suslin set!homogeneously}
if it is the
projection of a homogeneous tree. There are related notions of
\emph{weakly homogeneous tree} and \emph{weakly homogeneously
Suslin set} of reals, involving a more involved relationship with a
set of ultrafilters. Though it was not the original definition, let us just say that a tree on a set of
the form $\omega \times (\omega \times Z)$ is weakly homogeneous
if and only if the corresponding tree on $(\omega \times
\omega) \times Z$ is homogeneous, and note that a set of
reals is weakly homogeneously Suslin if and only if it is the
projection of a homogeneously Suslin set of
pairs.\index{weakly homogeneously Suslin set}\index{Suslin set!weakly homogeneously}

Martin's proof then shows the following.

\begin{theorem}[Martin\index{Martin, D. A.}] Homogeneously Suslin sets are determined.
\end{theorem}

The unfolding argument mentioned in Section \ref{defsec} then shows that weakly homogeneously
Suslin sets satisfy the regularity properties.



In retrospect, Martin's\index{Martin, D. A.} proof of analytic determinacy can be
broken into two parts, the fact that homogeneously Suslin sets are
determined, and the fact that if there is a Ramsey cardinal then
$\uTPi^{1}_{1}$ sets are homogeneously Suslin.




%Kechris\index{Kechris, A.} \cite{Kechris:1981} and Martin\index{Martin, D. A.} later independently extracted
%from Martin's\index{Martin, D. A.} proof of Theorem \ref{pi11det} the notion of a
%homogeneously Suslin set of reals.
%A tree on $\omega \times Z$, for some set $Z$, is
%$\kappa$-\emph{Suslin}






%Martin \cite{Martin:1970} showed that if there is a measurable
%cardinal $\kappa$, then $\uTPi^{1}_{1}$ sets are
%$\kappa$-homogeneously Suslin.

The results of \cite{MartinSolovay:1969}\index{Martin, D. A.}\index{Solovay, R.} can similarly be
reinterpreted. If $\uTPi^{1}_{1}$ sets are homogeneously Suslin,
then $\uTSigma^{1}_{2}$ sets are weakly homogeneously
Suslin.
%Furthermore, the Shoenfield\index{Shoenfield, J.} tree for $\uTSigma^{1}_{2}$ is
%weakly homogeneous.
The Martin-Solovay\index{Martin, D. A.}\index{Solovay, R.}
construction can be seen as a method for taking a $\gamma$-weakly
homogeneous tree $T$ (for some cardinal $\gamma$) and producing a
tree $S$ on $\omega \times \gamma'$, for some ordinal $\gamma'$, projecting to the complement of the projection of $T$.
%in all forcing extensions by partial orders of cardinality less than
%$\gamma$. The construction of this tree uses the ultrafilters witnessing
%weak homogeneity for $T$.
From this follows that all $\uTPi^{1}_{2}$ sets, and thus all $\uTSigma^{1}_{3}$ sets,
are projections of trees on the product of $\omega$ with some ordinal. More sophisticated arguments can
be carried out from the existence of sharps, using the fact that sharps give ultrafilters over certain inner models.

%Indeed, the construction can be carried
%out using the restriction of these ultrafilters to $L[T,x]$ for each
%real $x$.
%Since the Shoenfield\index{Shoenfield, J.} tree $T$ is in $L$, these restricted
%ultrafilters could be computed from the sharp of each real.

\subsection{Borel determinacy}\label{Borelsub}


In 1968, Friedman\index{Friedman, H.} \shortcite{Friedman:7071} showed that the Replacement axiom is
necessary to prove Borel determinacy, even for sets invariant under Turing degrees (he also showed that
analytic determinacy cannot hold in a forcing extension of $L$).
%Note that Borel determinacy is a $\Pi^{1}_{3}$ statement.
As refined by Martin,\index{Martin, D. A.}
%\shortcite{Martin:determinacy}
his results show (for each $\alpha < \omega_{1}$) that ZFC -
Power Set - Replacement +
``the $\alpha$th iteration of the power set of $\breals$ exists'' does
not prove the determinacy of all $\uTSigma^{0}_{1+ \alpha + 3}$ sets.

James Baumgartner\index{Baumgartner, J.} mixed the method of Martin's $\uTPi^{1}_{1}$-determinacy
proof with Davis's $\uTSigma^{0}_{3}$-determinacy proof to give a new
proof of $\uTSigma^{0}_{3}$-determinacy in ZFC. Using a similar
approach, Martin\index{Martin, D. A.} proved Det($\uTSigma^{0}_{4}$) from the existence of
a weakly compact cardinal,\footnote{A cardinal $\kappa$ is \emph{weakly compact}\index{weakly compact cardinal}
if $\kappa \to (\kappa)^{2}_{2}$. Weakly compact cardinals are below the existence of $0^{\#}$ and above
strongly inaccessible cardinals in the consistency strength hierarchy (see \cite[pp.~76,472]{Kanamori}).} and then Paris\index{Paris, J.} \shortcite{Paris:72} proved it
in ZFC (Paris\index{Paris, J.} noted at the end of his paper that his argument could
be carried out without the power set axiom, assuming instead only
that the ordinal $\omega_{1}$ exists).

Andreas Blass\index{Blass, A.} \shortcite{Blass:1975} and
Mycielski\index{Mycielski, J.} (1967, unpublished) independently
proved that AD$_{\mathbb{R}}$ is equivalent to determinacy for
integer games of length $\omega^{2}$. The key idea in
Blass's\index{Blass, A.} proof was to reduce determinacy in the
given game to determinacy in another, auxiliary, game in such a way
that one player's moves in the auxiliary game correspond to
fragments of his strategy in the original game. Martin\index{Martin,
D. A.} \shortcite{Martin:1975} used this basic idea to prove Borel
determinacy in 1974 (the auxiliary game was in fact an open game).
In his \shortcite{Martin:1985}, Martin\index{Martin, D. A.} gave a
short, inductive, proof of Borel determinacy, and introduced the
notion of \emph{unraveling} a set of reals---roughly, finding an
association of the set to a clopen set in a larger domain with a map
sending strategies in one game to strategies in the other. In his
\shortcite{Martin:1990}, Martin extended this method to games of
length $\omega$ played on any (possibly uncountable) set, with Borel
payoff (in the corresponding sense). Neeman\index{Neeman, I.}
\shortcite{Neeman:2000, Neeman:2006unrav} would unravel
$\uTPi^{1}_{1}$ sets from the assumption of a measurable cardinal
$\kappa$ of Mitchell rank $\kappa^{++}$ (proved to be an optimal
hypothesis by Steel\index{Steel, J.} \shortcite{Steel:1982}; see
\cite[pp.~357-360]{Jech:settheory} for the definition of Mitchell
rank). Complementing Friedman's theorem, Martin\index{Martin, D. A.}
%\shortcite{Martin:determinacy}
proved that for each $\alpha <
\omega_{1}$, the determinacy of each Boolean combination of $\uTSigma^0_{\alpha  +2}$ sets
follows
from ZF - Power Set  - Replacement + $\Sigma_{1}$-Replacement + ``the
$\alpha$th iteration of the power set of $\breals$ exists''.

\subsection{The $\uTPi^{1}_{1}$ difference hierarchy}\label{dhier}

Given a countable ordinal $\alpha$ and a real $a$, a set of reals
$X$ is said to be $\alpha$-$\Pi^{1}_{1}(a)$\index{$\alpha$-$\Pi^{1}_{1}(a)$} if there is wellordering
of $\omega$ of length $\alpha$ recursive in $a$ with corresponding
rank function $R \colon \omega \to \alpha$ and a $\Pi^{1}_{1}(a)$
subset $A$  of $\omega \times {^{\omega}\omega}$ such that
\begin{itemize}
\item for all $n,m \in \omega$, if $R(n) < R(m)$ then
$$\{ x \mid (m,x) \in A\} \subseteq\{ x \mid (n,x) \in A\};$$
\item $X$
is the set of reals $x$ for which the least $\xi$ such that either
$\xi = \alpha$ or $\xi < \alpha$ and $(R^{-1}(\xi), x) \not\in A$ is
odd.
\end{itemize}
This notation has its roots in \cite{Hausdorff:1908}. When $a$ is itself recursive one writes $\alpha$-$\Pi^{1}_{1}$. The
union of the sets $\alpha$-$\Pi^{1}_{1}(a)$ for all reals $a$ is
denoted $\alpha$-$\uTPi^{1}_{1}$. The union of the sets
$\alpha$-$\uTPi^{1}_{1}$ for all $\alpha < \omega_{1}$ is
denoted Diff($\uTPi^{1}_{1}$).
Note that
Diff($\uTPi^{1}_{1}$) is a proper subclass of
$\underTilde{\Delta}^{1}_{2}$.

Friedman \shortcite{Friedman:1971} extended Theorem \ref{pi11det} to
show that Det(3-$\uTPi^{1}_{1}$) follows from the existence of
the sharp of every real. Martin\index{Martin, D. A.} in 1975 then extended this result to show that the
existence of $0^{\#}$ is equivalent to Det($\bigcup_{\beta <
\omega^{2}}$ $\beta$-$\Pi^{1}_{1}$) (see \cite{Dubose:1990}).
%so Det($\bigcup_{\beta <\omega^{2}}$ $\beta$-$\Pi^{1}_{1}$) follows from Det($\Pi^{1}_{1}$).
%In fact, Martin\index{Martin, D. A.} \cite{Martin:1983ctble} showed that $0^{\#}$ is
%recursively equivalent to the complete
%$\game(\less\omega^{2}\thing\Pi^{1}_{1})$ real.
Harrington\index{Harrington, L.}
\shortcite{Harrington:1978} then proved the converse to Theorem
\ref{pi11det} by showing that Det($\Pi^{1}_{1}(a)$) implies the
existence of $a^{\#}$, for each real $a$.

For the purposes of the next theorem, say that a model has $\alpha$
measurable cardinals and indiscernibles if there exists a set of
ordertype $\alpha$ consisting of measurable cardinals of the model,
and there exist uncountably many ordinal indiscernibles of the
model above the supremum of these measurable cardinals. Martin\index{Martin, D. A.}
proved the following theorem after Harrington's result.

\begin{theorem} For any real $a$ and any ordinal $\alpha$ recursive
in $a$, the following are equivalent.
\begin{itemize}
\item {\rm Det}$(\bigcup_{\beta < \omega^{2}}$ $(\omega^{2} \cdot \alpha +
\beta)$-$\Pi^{1}_{1}(a))$.
\item {\rm Det}$((\omega^{2} \cdot \alpha +
1)$-$\Pi^{1}_{1}(a))$.
\item There is an inner model of\/ {\rm ZFC} containing $a$ and having
$\alpha$ many measurable cardinals and indiscernibles.
\end{itemize}
\end{theorem}

Still, a large-cardinal consistency proof of Det$(\underTilde{\Delta}^{1}_{2})$, the hypothesis used by Addison and Martin in their
extension of Blackwell's argument, remained beyond reach. John Green\index{Green, J.} \shortcite{Green:1978} showed that Det($\Delta^{1}_{2}$) implies
the existence of an inner model with a measurable cardinal of
Mitchell rank 1.

%Kechris\index{Kechris, A.} and Woodin\index{Woodin, W. H.} showed that $\Pi^{1}_{n+1}$-determinacy implies
%the determinacy of all $\game^{n} (\less\omega^{2} - \Pi^{1}_{1})$
%sets for all odd integers $n$. Neeman and Woodin\index{Woodin, W. H.} showed it for even
%$n$.

%Hjorth\index{Hjorth, G.} \cite{Hjorth:2001} showed that under AD$^{L(\mathbb{R})}$,
%$\game(\alpha - \Pi^{1}_{1})$ prewellorders (where $\alpha < \omega
%\cdot k$) have length less than $\omega_{k+1}$.

\subsection{Larger cardinals}

In Section \ref{ppdsec} we defined a measurable cardinal to be a cardinal $\kappa$ such that
there exists a nonprincipal $\kappa$-complete ultrafilter on $\kappa$. Equivalently, under
the Axiom of Choice, $\kappa$ is measurable if and only if there is a nontrivial elementary
embedding $j$ from the full universe $V$ into some inner model $M$ whose critical
point\index{critical point} is $\kappa$, i.e., such that $\kappa$ is the least ordinal
not mapped to itself by $j$. Many large cardinal
notions can be expressed both in terms of ultrafilters and in terms of embeddings, though in the Choiceless context (without the corresponding form of {\L}o\'s's Theorem, see \cite[p.~159]{Jech:settheory}) it is the definition in terms of ultrafilters which is relevant. For instance, a
cardinal $\kappa$ is \emph{supercompact}\index{supercompact cardinal}
if for each $\lambda > \kappa$ there exists a normal fine ultrafilter on
$\mathcal{P}_{\kappa} \lambda$.\footnote{Given a cardinal $\kappa$ and a set $X$, a collection $U$ of subsets of $\mathcal{P}_{\kappa} X$
is \emph{fine} if it contains the collection of supersets of each element of $\mathcal{P}_{\kappa} X$.} Under the Axiom of Choice, $\kappa$ is supercompact if and only if for every
$\lambda > \kappa$ there is an elementary embedding $j$ from $V$ into an inner model $M$ such that the critical point of $j$ is $\kappa$ and
$M$ is closed under sequences of length $\lambda$. Every supercompact cardinal is a limit of measurable cardinals. An even
larger large cardinal notion is the huge cardinal, where an uncountable cardinal $\kappa$ is \emph{huge}\index{huge cardinal}
if for some cardinal $\lambda > \kappa$ there is a $\kappa$-complete normal fine ultrafilter on $[\lambda]^{\kappa}$ (where ``normal'' and ``fine'' are defined in analogy with the supercompact case, see \cite[p.~331]{Kanamori}). Under AC,
$\kappa$ is huge if and only if there is an elementary embedding $j \colon V \to M$ with critical point $\kappa$ such that
$M$ is closed under sequences of length $j(\kappa)$. The existence of huge cardinals does not imply the existence
of supercompact cardinals, but it does imply their consistency.

Kunen\index{Kunen, K.} \shortcite{Kunen:1971d} put a limit on the large cardinality hierarchy,
showing in ZFC that there is no nontrivial elementary embedding from
$V$ into itself. A corollary of the proof is that for any elementary
embedding $j$ of $V$ into any inner model $M$, if $\delta$ is the
least ordinal above the critical point of $j$ sent to itself by $j$,
then $V_{\delta + 1} \not\subseteq M$. In 1978, Martin\index{Martin, D. A.}
%%%%%SOLOVAY24: "\delta + 2" was changed to "\delta + 1" in previous line.
\shortcite{Martin:1980} proved $\uTPi^{1}_{2}$-determinacy
from hypothesis I2, which states that for some ordinal $\delta$
there is a nontrivial elementary embedding of $V$ into an inner
model $M$ with critical point less than $\delta$ such that
$V_{\delta} \subseteq M$ and $j(\delta) = \delta$. In 1984, Woodin\index{Woodin, W. H.}
proved AD$^{L(\mathbb{R})}$ from I0, the
statement that for some ordinal $\delta$ there is a nontrivial
elementary embedding from $L(V_{\delta + 1})$ into itself with
critical point below $\delta$, thus verifying Solovay's\index{Solovay, R.}
conjecture that AD$^{L(\mathbb{R})}$ would
follow from large cardinals. I0 is one of the strongest large
cardinal hypotheses not known to be inconsistent. The inner model program
at the time had produced models for
many measurable cardinals, hypotheses far short of I2, and so there
was little hope of showing that I2 and I0 were necessary for these
results.

New large cardinal concepts would prove to be the missing
ingredient. Given an ideal $I$ on a set $X$, forcing with the
Boolean algebra given by the power set of $X$ modulo $I$ gives a
$V$-ultrafilter on the power set of $X$.\footnote{An
\emph{ideal}\index{ideal} is a collection of sets closed under
subsets and finite unions. Given a model $M$ and a set $X$ in $M$,
an $M$-\emph{ultrafilter}\index{M-ultrafilter@$M$-ultrafilter} is a subset of
$\mathcal{P}(X) \cap M$ closed under supersets and finite
intersections such that for every $A \subseteq X$ in $M$, exactly one of $A$
and $X \setminus A$ is in $U$. Note that $U$ does not need to be an
%%%%%SOLOVAY25: The previous sentence was reformulated to imply that an $M$-ultrafilter consists of nonempty sets.
element of $M$.} The ideal $I$ is said to be
\emph{precipitous}\index{precipitous ideal}\index{ideal!precipitous}
if the ultrapower of $V$
by this generic ultrafilter is wellfounded in all generic extensions.
If the underlying set $X$ is a cardinal $\kappa$, the ideal $I$ is
said to be \emph{saturated} if the Boolean algebra
$\mathcal{P}(\kappa)/I$ has no antichains of cardinality
$\kappa^{+}$.\footnote{An \emph{antichain}\index{antichain} in a partial order (or a Boolean algebra) is a set of
 pairwise incompatible elements. In the case of a Boolean algebra of the form $\mathcal{P}(\kappa)/I$, an antichain is a collection
 of subsets of $\kappa$ not in $I$ which pairwise have intersection in $I$.} If $\kappa$ is a regular cardinal, saturation of $I$
implies precipitousness. Huge cardinals were invented by Kunen\index{Kunen, K.} \shortcite{Kunen:1978}, who used them to
produce a saturated ideal on $\omega_{1}$.

%Steel\index{Steel, J.} and Van Wesep\index{Van Wesep, R.} \shortcite{SteelVanWesep:1982} had shown that starting
%from a model of a very strong determinacy hypothesis (weaker than
%AD$_{\mathbb{R}}$ + ``$\Theta$ is regular") one could force to get a
%model of choice in which NS$_{\omega_{1}}$ is saturated. Woodin
%\index{Woodin, W. H.} \shortcite{Woodin:1983ZFCAD} later improved the hypothesis to
%AD$^{L(\mathbb{R})}$.

In early 1984, Matthew Foreman,\index{Foreman, M.} Menachem
Magidor\index{Magidor, M.} and Shelah\index{Shelah, S.}
\shortcite{ForemanMagidorShelah:1988} showed that if there exists a
supercompact cardinal---a hypothesis much weaker than I0 or I2---then
there is an $\omega_{1}$-preserving forcing making the
nonstationary ideal on $\omega_{1}$ (NS$_{\omega_{1}}$) saturated.

Foreman\index{Foreman, M.} (see \shortcite{Foreman:1986}) and Magidor\index{Magidor, M.} \shortcite{Magidor:1980} had
earlier made a connection between generic elementary
embeddings\footnote{A \emph{generic elementary embedding} is an
elementary embedding of the universe $V$ into some class model $M$
which exists in a forcing extension of $V$.} and regularity
properties for reals. Magidor\index{Magidor, M.} \shortcite{Magidor:1980} in particular had
shown that the Lebesgue measurability of $\Sigma^{1}_{3}$ sets
followed from the existence of a generic elementary embedding with
critical point $\omega_{1}$ and wellfounded image model (the existence of such an
embedding follows from the Foreman-Magidor-Shelah\index{Foreman, M.}\index{Magidor, M.}\index{Shelah, S.} result mentioned
above). Woodin\index{Woodin, W. H.} noted that these arguments plus earlier work of his
(see \shortcite{Woodin:1986}) could be used to extend this to Lebesgue
measurability for all projective sets. Woodin\index{Woodin, W. H.} also noted that
arguments from \cite{ForemanMagidorShelah:1988}\index{Foreman, M.}\index{Magidor, M.}\index{Shelah, S.} could be used to
prove the Lebesgue measurability of all sets of reals in
$L(\mathbb{R})$, if one could force to produce a saturated ideal on
$\omega_{1}$ without adding reals. Shelah\index{Shelah, S.} then noted that techniques
from \cite{Shelah:PIF} could be modified to do just that. It followed then that the existence of a supercompact
cardinal implies that all sets of reals in $L(\mathbb{R})$ are Lebesgue measurable.

Woodin\index{Woodin, W. H.} and Shelah\index{Shelah, S.} then addressed the problem of weakening the
hypotheses needed for the Lebesgue measurability of all projective sets of
reals. We follow the account in \cite{Neeman:DLG}. Woodin\index{Woodin, W. H.} noted that
a superstrong cardinal sufficed. Shelah\index{Shelah, S.} then isolated a weaker
notion now known as a \emph{Shelah cardinal}, and showed that
the existence of $n+1$ Shelah cardinals implies that
$\uTSigma^{1}_{n+2}$ sets are Lebesgue measurable.

\begin{definition} A cardinal $\kappa$
is a \emph{Shelah cardinal}\index{Shelah cardinal} if for every $f
\colon \kappa \to \kappa$ there is an elementary embedding $j \colon
V \to N$ with critical point $\kappa$ such that $V_{j(f)(\kappa)} \subseteq N$.
\end{definition}

Woodin\index{Woodin, W. H.}
noted that by switching the quantifiers in Shelah's\index{Shelah, S.} definition one
obtained a weaker, still sufficient, hypothesis, now known as a
Woodin cardinal.

\begin{definition} A cardinal $\delta$ is a \emph{Woodin
cardinal}\index{Woodin cardinal} if for each function $f \colon \delta \to \delta$ there
exists an elementary embedding $j \colon V \to M$ with
critical point $\kappa < \delta$ closed under $f$ such that
$V_{j(f)(\kappa)} \subseteq M$.
\end{definition}

Woodin proved that the existence of $n$ Woodin cardinals below a measurable
cardinal implies the Lebesgue measurability of
$\uTSigma^{1}_{n+2}$ sets, the same amount of
measurability that would follow from
$\uTPi^{1}_{n+1}$-determinacy.
%Shelah\index{Shelah, S.} and Woodin\index{Woodin, W. H.} showed
%that the existence of a supercompact cardinal implies that all sets of reals in $L(\mathbb{R})$ are Lebesgue measurable and have
%the property of Baire.
All of this
work was done within a few weeks of the Foreman-Magidor-Shelah\index{Foreman, M.}\index{Magidor, M.}\index{Shelah, S.} result on the saturation of
NS$_{\omega_{1}}$. In \cite{ShelahWoodin:1990} the hypothesis for the statement that all sets of reals in $L(\mathbb{R})$ are Lebesgue measurable and have the property of Baire was reduced to the existence of ordertype $\omega + 1$ many Woodin cardinals.  The hypothesis was to be reduced even further.

Woodin\index{Woodin, W. H.} extracted from the Foreman-Magidor-Shelah\index{Foreman, M.}\index{Magidor, M.}\index{Shelah, S.} results a one-step
forcing for producing generic elementary embeddings with critical
point $\omega_{1}$, and developed it into a general method, now
known as the \emph{stationary tower}. Using this he showed (by the fall of 1984, see his \shortcite{Woodin:1988})
that if there exists a supercompact cardinal, then every set of
reals in $L(\mathbb{R})$ is weakly homogeneously Suslin.

%After Martin\index{Martin, D. A.} and Steel\index{Steel, J.} proved Theorem \ref{MartinSteeltheorem} below, Woodin\index{Woodin, W. H.} reduced the hypothesis to the existence of infinitely many Woodin cardinals below a measurable cardinal.


%Foreman, Magidor and Shelah showed that if $\kappa$ is a
%supercompact cardinal, the forcing $Col(\omega_{1}, \less\kappa)$
%makes the nonstationary ideal on $\omega_{1}$
%presaturated\footnote{An ideal $I$ on $\kappa$ is
%\emph{presaturated}\index{presaturated ideal} if it is precipitous
%and forcing with $\mathcal{P}(\kappa)/I$ cannot collapse
%$\kappa^{+}$.} (and thus precipitous).



Steel\index{Steel, J.} had been working on the problem of finding inner models for supercompact cardinals.
Inspired by the results of Foreman,\index{Foreman, M.} Magidor,\index{Magidor, M.} Shelah\index{Shelah, S.} and Woodin\index{Woodin, W. H.},
he begin to work on producing models for Woodin cardinals, and had some partial results by the spring of
1985, producing inner models with certain weak variants of Woodin cardinals. These models were generated by sequences of
\emph{extenders}, directed systems of ultrafilters which collectively generate
elementary embeddings whose images contain more of $V$ than possible for
embeddings generated by a single ultrafilter. Special cases of extenders had appeared in Jensen's\index{Jensen, R.} proof
of the Covering Lemma, and in \cite{Mitchell:1979}. Jensen formulated the general notion of extender, which first appeared
in \cite{Dodd:1982}.
Steel\index{Steel, J.} and Martin\index{Martin, D. A.} saw that the problem of building models with Woodin cardinals was linked to the problem of
proving determinacy, and they sets their sight on this problem in the late spring of 1985.

One key combinatorial problem related to elementary embeddings is
whether infinite iterations of these embeddings produce wellfounded
models. Kunen\index{Kunen, K.} \shortcite{Kunen:1970} had shown that the answer was positive for iterations
%%%%%SOLOVAY26: Kunen:1970 was added to the bibliography
derived from a single ultrafilter. With extenders the situation was
more complicated, as the iterations did not need to be linear but
could produce trees of models with no rule for finding a path
through the tree leading to a wellfounded model (indeed, this nonlinearity was essential, since otherwise
the models would have simply definable wellorderings of their reals). The simplest such
tree, a so-called \emph{alternating chain}, is countably infinite and
consists of two infinite branches. Martin\index{Martin, D. A.} and Steel\index{Steel, J.} saw that the
issue of wellfoundedness for the direct limits along the two
branches was linked. This observation led to the following
theorem, proved in August of 1985.

\begin{theorem}[Martin-Steel\index{Steel, J.} \shortcite{MartinSteel:1989}]\label{MartinSteeltheorem}
Suppose that
$\lambda$ is a Woodin cardinal and $A$ is a $\lambda^{+}$-weakly
homogeneously Suslin set of reals. Then for any $\gamma < \lambda$,
${^{\omega}\omega} \setminus A$ is $\gamma$-homogeneously Suslin.
\end{theorem}


It follows from this and the fact that analytic sets are homogeneously Suslin in the presence of a measurable cardinal
that if there exist $n$ Woodin cardinals below a
measurable cardinal, then $\uTPi^{1}_{n+1}$ sets are
determined, and that Projective
Determinacy follows from the existence of infinitely many Woodin
cardinals.

Combined with Woodin's\index{Woodin, W. H.} application of the stationary tower mentioned above, the Martin-Steel\index{Martin, D. A.}\index{Steel, J.} theorem implied that AD$^{L(\mathbb{R})}$
follows from the existence of a supercompact cardinal. By the end of 1985, Woodin\index{Woodin, W. H.} had improved the
%%%%%SOLOVAY27: "follows" inserted in the previous line.
hypothesis to the existence of infinitely many Woodin cardinals
below a measurable cardinal (see \cite{Larson:stationary}).

\begin{theorem}[Woodin]\label{adfromwoodins} If there exist infinitely many Woodin cardinals below a measurable cardinal, then
{\rm AD} holds in $L(\mathbb{R})$.
\end{theorem}


%With Martin\index{Martin, D. A.} he set his sights on the and Steel\index{Steel, J.} (still in 1984,
%see \shortcite{MartinSteel:1994})
%began a project to produce canonical inner models for Woodin
%cardinals.  This project would eventually succeed, but along the way
%they would prove a fundamental fact (Theorem
%\ref{MartinSteeltheorem} below) which would show that

%At this point the concepts involved increase dramatically in
%complexity, and we will have to content ourselves from now on with
%informal descriptions instead of formal definitions.

In the spring of 1986, Martin\index{Martin, D. A.} and Steel\index{Steel, J.}
\shortcite{MartinSteel:1994} produced
extender models with $n$ Woodin cardinals and $\Delta^{1}_{n+2}$
wellorderings of the reals. Such a model necessarily has a
$\Sigma^{1}_{n+2}$ set which is not Lebesgue
measurable, and fails to satisfy $\Pi^{1}_{n+1}$-determinacy.

Skipping ahead for a moment, let $(*)_{n}$ be the statement that for each real $x$ there exists
an iterable model $M$ containing $x$ and $n$ Woodin cardinals plus the sharp of $V_{\delta}^{M}$, for $\delta$ the largest of
these Woodin cardinals. For odd $n$, the equivalence of
$\uTPi^{1}_{n+1}$-determinacy and $(*)_{n}$ was proved by Woodin\index{Woodin, W. H.} in
1989. That $(*)_{n}$ implies
$\uTPi^{1}_{n+1}$-determinacy for all $n$ was proved by
Neeman \shortcite{Neeman:1995} in 1994. Roughly, Neeman's methods work by considering a modified game in
which one player builds an iteration tree and makes moves in the
image of the original game by the embeddings given by the tree. In 1995, Woodin\index{Woodin, W. H.} proved that
$\uTPi^{1}_{n+1}$-determinacy implies $(*)_{n}$ for even
$n > 0$.




As discussed above, homogeneously Suslin and weakly homogeneously Suslin sets of reals
played an important role in applications of large cardinals to regularity properties for sets of reals, as early
as the 1969 results of Martin\index{Martin, D. A.} and Solovay.\index{Solovay, R.} Qi Feng,\index{Feng, Q.} Magidor\index{Magidor, M.} and Woodin\index{Woodin, W. H.} \shortcite{FengMagidorWoodin:1992}
introduced a related tree representation
property for sets of reals. Given a cardinal $\kappa$, a set $A \subseteq\omega^{\omega}$ is $\kappa$-\emph{universally
Baire}\index{universally Baire set}  if
there exist trees $S$, $T$ such that $p[S] = A$ and $S$ and $T$
project to complements in every forcing extension by a partial order
of cardinality less than or equal to $\kappa$.

Woodin\index{Woodin, W. H.} (see \cite{Kanamori, Larson:stationary}) showed that if
$\delta$ is a Woodin cardinal, then $\delta$-universally Baire sets
of reals are $\less\delta$-weakly homogeneously Suslin. It follows
from the arguments of \cite{MartinSolovay:1969} that if $A \subseteq\omega^{\omega}$ is
$\kappa^{+}$-weakly homogeneously Suslin, then it is
$\kappa$-universally Baire.

Feng,\index{Feng, Q.} Magidor\index{Magidor, M.} and Woodin\index{Woodin, W. H.} showed that if there exist two Woodin
cardinals, then every universally Baire set is determined. Neeman\index{Neeman, I.}
later improved the hypothesis to one Woodin cardinal. In addition to the following theorem, Feng,\index{Feng, Q.} Magidor\index{Magidor, M.}
and Woodin\index{Woodin, W. H.} showed that Det($\uTPi^{1}_{1}$) is
equivalent to the statement that every $\uTSigma^{1}_{2}$ set of reals
is universally Baire.

\begin{theorem}[Feng,\index{Feng, Q.} Magidor\index{Magidor, M.}
and Woodin\index{Woodin, W. H.} \shortcite{FengMagidorWoodin:1992}] Assume {\rm AD}$^{L(\mathbb{R})}$. Then the following are
equivalent.
\begin{itemize}
\item {\rm AD}$^{L(\mathbb{R})}$ holds in every
forcing extension.
\item Every set of reals in $L(\mathbb{R})$ is universally Baire.
\end{itemize}
\end{theorem}



Woodin's\index{Woodin, W. H.} \emph{Tree Production Lemma} is a powerful means for
showing that sets of reals are universally Baire (see
\cite{Larson:stationary}). Woodin's\index{Woodin, W. H.} proof of Theorem \ref{adfromwoodins} proceeded by
applying the lemma to the
set $\mathbb{R}^{\#}$. Informally, the lemma can be interpreted as saying that a set of reals $A$ is
$\delta$-universally Baire if for every real $r$ generic for a partial order in $V_{\delta}$, either
$r$ is in the image of $A$ for every $\mathbb{Q}_{\less\delta}$-embedding for which $r$ is in the image model,
or $r$ is in the image of $A$ for no such embedding.

The partial order ${\rm Col}(\omega, \less\delta)$\index{${\rm Col}(\omega, \less\delta)$} consists of all finite partial functions $p$ from
$\omega \times \delta$ to $\delta$, with the requirement that $p(n,\alpha) \in \alpha$ for all $(n,\alpha)$
in the domain of $p$ (${\rm Col}(\omega, \less\delta)$ is ordered by inclusion). If $\delta$ is a regular cardinal, then $\delta$ is the $\omega_{1}$
of any forcing extension by ${\rm Col}(\omega, \less\delta)$.


\begin{theorem}[Tree Production Lemma]
Suppose that $\delta$ is a Woodin cardinal. Let
$\phi$ and $\psi$ be binary formulas, and let $x$ and $y$ be
arbitrary sets, and assume that the empty condition in the
stationary tower $\mathbb{Q}_{\less\delta}$ forces that for each
real $r$,
$$M \models \psi(r,j(y)) \Leftrightarrow V[r] \models \phi(r,x),$$
where $j \colon V \to M$ is the induced elementary embedding. Then
there exist trees $T$,$S$ such that $p[T] = \{ r \mid \psi(r,y)\}$
and $T$ and $S$ project to complements after forcing with
${\rm Col}(\omega, \less\delta)$.
\end{theorem}

Woodin\index{Woodin, W. H.} followed this by determining the exact consistency strength of AD.
The forward direction of the next theorem (proved in \cite{KoellnerWoodin:handbook}) shows from ZF + {\rm AD} that there
exist infinitely many Woodin cardinals in an inner model of a
forcing extension (HOD of the forcing extension with respect to
certain parameters) of $V$. The proof built on a sequence of results, starting with Solovay's\index{Solovay, R.} theorem that AD implies that $\omega_{1}$ is a measurable cardinal, which, as mentioned above, also shows that $\omega_{1}$
(as defined in $V$) is measurable in the inner model HOD.
Becker\index{Becker, H.} (see \cite{BeckerMoschovakis:1981}) had shown that, under AD,
$\omega_{1}^{V}$ is the least measurable in HOD.
Becker,\index{Becker, H.} Martin,\index{Martin, D. A.} Moschovakis\index{Moschovakis, Y.} and Steel\index{Steel, J.} then showed that under AD + $V\mathord{=}L(\mathbb{R})$, $\undertilde{\delta}^{2}_{1}$ is $\beta$-strong in
HOD, where $\beta$ is the least measurable cardinal greater than
$\undertilde{\delta}^{2}_{1}$ in HOD.\footnote{A cardinal
$\kappa$ is $\beta$-\emph{strong}\index{strong cardinal} if there is an elementary
embedding $j \colon V \to M$ with critical point $\kappa$ such that $V_{\beta} \subseteq M$, and
$\less\delta$-\emph{strong} if it is $\beta$-strong for all $\beta < \delta$.}
In the 1980's, Woodin\index{Woodin, W. H.} showed under the same
hypothesis that $\undertilde{\delta}^{2}_{1}$ is $\beta$-strong in
HOD for every $\beta < \Theta$ (and that
$\undertilde{\delta}^{2}_{1}$ is the least ordinal with this
property), and that $\Theta$ is Woodin in HOD.

\begin{theorem}[Woodin\index{Woodin, W. H.}] The following are equiconsistent.
%%%%%SOLOVAY28: extra "the" removed.
\begin{itemize}
\item {\rm ZF + AD}
\item There exist infinitely many Woodin cardinals.
\end{itemize}
\end{theorem}

The following theorem illustrates the reverse direction of the equiconsistency (see \cite{Steel:dmt}).
It can be seen as a special case of the Derived Model Theorem, discussed in Section \ref{adpadr}.


\begin{theorem}[Woodin\index{Woodin, W. H.}] Suppose that $\lambda$ is a limit of Woodin
cardinals, and $G \subseteq {\rm Col}(\omega, \less\lambda)$ is  $V$-generic
filter. Let $\mathbb{R}^{*} = \bigcup \{ \mathbb{R}^{V[G \restrict
\alpha]} \mid \alpha < \lambda \}$. Then {\rm AD} holds in
$L(\mathbb{R}^{*})$.
\end{theorem}

The results of Section \ref{dhier} illustrate the difficulties in proving
the determinacy of $\Pi^{1}_{2}$ sets. Woodin\index{Woodin, W. H.} resolved this problem in 1989.
The forward direction of the following theorem is proved in \cite{KoellnerWoodin:handbook}. The proof was inspired in part by a result of Kechris\index{Kechris, A.} and Solovay\index{Solovay, R.} \shortcite{KechrisSolovay:1985}, saying that
in models of the form $L[a]$ for $a \subseteq \omega$, $\Delta^{1}_{2}$-determinacy implies the determinacy of all ordinal definable sets of
reals. Standard arguments show that if $\Delta^{1}_{2}$ determinacy holds,
then it holds in $L[x]$ for some real $x$. Woodin\index{Woodin, W. H.} showed that if $V$
is $L[x]$ for some real $x$, and $\Delta^{1}_{2}$-determinacy holds,
then $\omega_{2}^{L[x]}$ is a Woodin cardinal in HOD. Recall (from the end of Section \ref{ppmeas}) that $\Delta^{1}_{2}$-determinacy and $\Pi^{1}_{2}$-determinacy are equivalent, by a result of Martin.

\begin{theorem}[Woodin\index{Woodin, W. H.}] The
following are equiconsistent.
\begin{itemize}
\item {\rm ZFC} + {\rm Det}($\Delta^{1}_{2}$).
\item {\rm ZFC} + There exists a Woodin cardinal.
\end{itemize}
\end{theorem}

The following theorem illustrates the reverse direction. Its proof can be found in \cite[p.~1926]{Neeman:handbook}.
The partial order ${\rm Col}(\omega, \delta)$\index{${\rm Col}(\omega, \delta)$} consists of all finite partial functions from $\omega$ to $\delta$, ordered
by inclusion.

\begin{theorem}[Woodin\index{Woodin, W. H.}] If\/ $\delta$ is a Woodin cardinal and $G
\subseteq {\rm Col}(\omega, \delta)$ is a $V$-generic filter, then
$\Delta^{1}_{2}$-determinacy holds in $V[G]$.
\end{theorem}





%From a consistency strength perspective, one can get away with
%slightly less (see \cite{Neeman:handbook}).





%Woodin\index{Woodin, W. H.} and Neeman (see \cite{Neeman:1995}) showed that
%$\Pi^{1}_{n+1}$-determinacy implies
%$\game^{(n)}(\less\omega^{2}\thing\Pi^{1}_{1})$ determinacy for each
%$n \in \omega$. Kechris\index{Kechris, A.} and Woodin\index{Woodin, W. H.} had shown this for odd $n$ by
%descriptive set-theoretic means, not involving models of large
%cardinals.







%Martin\index{Martin, D. A.} and Steel\index{Steel, J.} built a model with a Woodin cardinal and a
%$\Sigma^{1}_{3}$ wellordering of the reals -
%$\uTSigma^{1}_{2}$-determinacy therefore fails in this
%model.







\section{Later developments}

In this final section we briefly review some of the developments that followed the results of
the previous section. As discussed in the introduction to this chapter, the set of topics presented
here is by no means complete. The first two subsections discuss forms of determinacy ostensibly stronger than AD,
in models larger than $L(\mathbb{R})$. The next discusses applications of determinacy to the realm of AC, via
producing models of AC by forcing over models of determinacy. In the last two subsections we present some results
which derive forms of determinacy from their ostensibly weak consequences, or from statements having no obvious relationship
to determinacy. Many of the results of the last two subsections are applications of the study of canonical inner models
for large cardinals.

\subsection{\rm{AD}$^{+}$ and \rm{AD}$_{\mathbb{R}}$}\label{adpadr}

Moschovakis\index{Moschovakis, Y.} \shortcite{Moschovakis:1981} proved that under AD, if
$\lambda$ is less than $\Theta$, $A$ is a set of functions from
$\omega$ to $\lambda$ and $A$ is Suslin and co-Suslin, then the game
$G_{\omega}(A)$ is determined, where here the players play elements of $\lambda$.
Woodin\index{Woodin, W. H.} formulated the following axiom as an attempt to
capture the consequences of being an inner model of a model of AD with the same reals
and Suslin representations for all sets of reals in the inner model. A set of reals $A$ is
said to be $\infty$-Borel if there is a set of ordinals $S$ and binary formula $\phi$ such
that $A = \{ x \in \mathbb{R} \mid L[x,S] \models \phi(x,S)\}$. A Suslin representation
is an example of a witness to a set of reals being $\infty$-Borel.



\begin{definition} AD$^{+}$ is the
conjunction of the following statements.\index{AD$^{+}$}
\begin{itemize}
\item DC$(\breals)$
\item Every set of reals is $\infty$-Borel.
\item If $\lambda < \Theta$ and $\pi \colon \lambda^{\omega}
\to \omega^{\omega}$ is a continuous function, then $\pi^{-1}[A]$ is
determined for every $A \subseteq \omega^{\omega}$.
\end{itemize}
\end{definition}

It is an open question whether AD implies AD$^{+}$, though it is
known that AD$^{+}$ holds in all models of AD of the form $L(A,
\mathbb{R})$, where $A$ is a set of reals (some of the details of
the argument showing this appear in \cite{Jackson:handbook}). It is
not known whether AD$_{\mathbb{R}}$ implies AD$^{+}$, though
AD$^{+}$ does follow from AD$_{\mathbb{R}}$ + DC.

The following consequences of AD$^{+}$ were announced in
\cite{Woodin:1999}.
%If
%a pointclass $\Gamma$ is a Boolean subalgebra of $\mathcal{P}(\mathbb{R})$
%and is cloaed under continuous images and preimages, then
%$N_{\Gamma}$\index{$N_{\Gamma}$} is the collection of all sets
%$X$ such that the triple $\langle \tc (X), X, \in \rangle$, where
%$\tc (X)$ is the transitive closure of $X$, is isomorphic to a triple
%$\langle \mathbb{R}/ \sim, P/\sim, E/\sim \rangle$, where
%\begin{itemize}
%\item $\sim$ is an equivalence relation on $\mathbb{R}$,
%\item $P \subseteq \mathbb{R}$ and $E \subseteq \mathbb{R} \times \mathbb{R}$,
%\item $\sim, P, E$ are all in $\Gamma$.
%\end{itemize}
%For such a $\Gamma$, $M_{\Gamma}$ is the set of $X \in N_{\Gamma}$
%such
%that whenever $\pi_{0}$ and $\pi_{1}$ are functions in $N_{\Gamma}$
%with domain $\mathbb{R}$, the set
%$$\{ (x,y) \in \mathbb{R} \times \mathbb{R} \mid \pi_{0}(x)
%= \pi_{1}(y) \wedge \pi_{0}(x) \in \tc (X) \} \in \Gamma.$$


\begin{theorem}[{\rm ZF + DC}$(\breals)$] If\/ {\rm AD}$^{+}$ holds and
$V = L(\mathcal{P}(\mathbb{R}))$, then
\begin{itemize}
\item the pointclass $\Sigma^{2}_{1}$ has the scale property,
\item every $\Sigma^{2}_{1}$ set of reals is the projection
of a tree in \emph{HOD},
\item every true $\Sigma_{1}$-sentence is witnessed by a
$\undertilde{\Delta}^{2}_{1}$ set of reals.
%$M_{\undertilde{\Delta}^{2}_{1}} \prec_{\Sigma_{1}}
%L(\mathcal{P}(\mathbb{R}))$.
\end{itemize}
\end{theorem}

Woodin's\index{Woodin, W. H.} \emph{Derived Model Theorem}, proved around 1986, gives
a means of producing models of AD${^+}$. The model $L(\mathbb{R}^{*}, Hom^{*})$ in the following
theorem is said to be a \emph{derived model}\index{derived model} (over the ground model).
A tree $T$ is said to be $\less\lambda$-absolutely complemented if there is a tree $S$ such that
$p[T] = \mathbb{R} \setminus p[S]$ in all forcing extensions by partial orders of cardinality less
than $\lambda$.

%AD$^{+}$ is downwards absolute
%to models with the same reals, a fact which is relevant for interpreting the Derived Model Theorem as stated below

\begin{theorem}[Derived Model Theorem; Woodin\index{Woodin, W. H.} (see \cite{Steel:dmt})] Let $\lambda$ be a limit of Woodin cardinals. Let $G
\subseteq {\rm Col}(\omega, \less\lambda)$ be a $V$-generic filter. Let
\begin{itemize}
\item $\mathbb{R}^{*}$ be $\bigcup_{\alpha < \lambda}\mathbb{R}^{V[G
\restrict \alpha]}$;
\item $Hom^{*}$ be the collection of sets of
the form $p[T] \cap \mathbb{R}^{*}$, for $T$ a
$\less\lambda$-absolutely complemented tree in $V$;
\item $\Gamma$ be the collection of
sets of reals $A$ in $V[G]$ such that $L(A, \mathbb{R}^{*}) \models {\rm AD}^{+}$.
\end{itemize}
Then
\begin{itemize}
\item $L(Hom^{*}, \mathbb{R}^{*}) \models {\rm AD}^{+}$.
\item $Hom^{*}$ is the collection of Suslin, co-Suslin sets of reals
in $L(Hom^{*}, \mathbb{R}^{*})$.
\item $L(\Gamma, \mathbb{R}^{*}) \models {\rm AD}^{+}$.
\item $Hom^{*}$ is the collection of Suslin, co-Suslin sets of reals
in $L(\Gamma, \mathbb{R}^{*})$.
\end{itemize}
\end{theorem}

The Derived Model Theorem has a converse, also due to Woodin,\index{Woodin, W. H.} which says that all models
of AD$^{+}$ arise in this fashion.

\begin{theorem}[Woodin\index{Woodin, W. H.}] Let $M$ be a model of {\rm AD}$^{+}$, and let
$\Gamma$ be the collection of sets of reals which are Suslin,
co-Suslin in $M$. Then in a forcing extension of $M$ there is an
inner model $N$ such that $L(\mathbb{R}^{*}, \Gamma)$ is a derived
model over $N$.
\end{theorem}

In unpublished work, Woodin\index{Woodin, W. H.} has shown that over AD, AD$_{\mathbb{R}}$ is equivalent
to some of its ostensibly weak consequences (see \cite{Woodin:1999}). The implication from (\ref{fourfour}) to
(\ref{fourone}) in the following theorem is due independently to Martin. The implication from (\ref{fourone}) to
(\ref{fourfour}) relies heavily on work of Becker \shortcite{Becker:1985}.
%%%%%SOLOVAY: Becker:1985 was added to the bibliography.

\begin{theorem}[Woodin\index{Woodin, W. H.}]\label{adrequiv} Assume {\rm ZF + DC}. Then the following are
equivalent.
\begin{enumerate}
\item\label{fourone} {\rm AD}$_{\mathbb{R}}$
\item\label{fourtwo} {\rm AD} + Uniformization
\item\label{fourthree} {\rm AD} + Every set of reals has a scale
\item\label{fourfour} {\rm AD} + Every set of reals is Suslin
\end{enumerate}
\end{theorem}


%\begin{theorem}[Woodin\index{Woodin, W. H.}] Assume ZF + DC. Then
%the following are equivalent.
%\begin{itemize}
%\item AD$_{\mathbb{R}}$.
%\item AD + Every set of reals has a scale.
%\item AD + Every subset of
%${^{2}(^{\omega}\omega)}$ can be uniformized.
%\end{itemize}
%\end{theorem}

Woodin would also produce models of AD$_{\mathbb{R}}$ from large cardinals.

\begin{theorem}[Woodin\index{Woodin, W. H.}] Suppose that there exists a cardinal $\delta$ of
cofinality $\omega$ which is a limit of Woodin cardinals and $\less\delta$-strong cardinals.
Then there is a forcing extension in which there is an inner model containing the reals and satisfying
{\rm AD}$_{\mathbb{R}}$.
\end{theorem}

\subsection{Long games}\label{longgames}

As mentioned above, Blass\index{Blass, A.} \shortcite{Blass:1975} and Mycielski\index{Mycielski, J.} showed
that determinacy for games of length $\omega^{2}$ is equivalent to
AD$_{\mathbb{R}}$. For each $n \in \omega$, determinacy for games of
length $\omega + n$ is equivalent to AD (think of the game as being
divided in two parts, where in the first part (of length $\omega$)
the players try to obtain a position from which they have a winning
strategy in the second; the winning strategy in the second part can
be coded by an integer, and thus uniformly chosen).


Martin\index{Martin, D. A.} and Woodin\index{Woodin, W. H.} independently showed that AD$_{\mathbb{R}}$ is
equivalent to determinacy for games of length $\alpha$ for each
countable $\alpha \geq \omega^{2}$. Determinacy for games of length $\omega
\cdot 2$ easily gives uniformization. It follows from this and Theorem \ref{adrequiv} that
AD$_{\mathbb{R}}$ is equivalent to determinacy for games of length
$\alpha$ for each $\alpha \geq \omega \cdot 2$.

While AD does not imply uniformization, the Second Periodicity
Theorem shows that PD implies the uniformization of projective sets.
It follows that PD is equivalent to PD for games of length less than
$\omega^{2}$. As noted by Neeman \shortcite{Neeman:2005}, the
techniques from this proof can be used to prove the determinacy of
games of length $\omega^{2}$ with analytic payoff from
AD$^{L(\mathbb{R})}$ plus the existence of $\mathbb{R}^{\#}$.

Steel\index{Steel, J.} \shortcite{Steel:1988} considered \emph{continuously coded games},
games where each stage of the game is associated with an integer,
and the game ends when an associated integer is repeated. Such a
game must end after countably many rounds, but runs of the game can
have any countable length. Steel\index{Steel, J.} proved that ZF + AD + DC + ``every
set of reals has a scale" + ``$\omega_{1}$ is
$\mathcal{P}(\mathbb{R})$-supercompact" implies the determinacy of
all continuously coded games.

Extending this result, Neeman \shortcite{Neeman:DLG, Neeman:2005, Neeman:2006firstadmissible}
proved a number of determinacy results for games of variable countable length. The proofs of
many of these results reduced the determinacy of long games to the iterability of models containing large cardinals.

We give one example. Given $C \subseteq \mathbb{R}^{\less\omega_{1}}$, let
$G_{\rm{local}}(L, C)$ be the game where players $I$ and $II$ alternate playing
natural numbers so as to define elements $z_{\xi}$ of the Baire space. The game ends
at the first $\gamma$ such that $\gamma$ is uncountable
in $L[z_{\xi} : \xi < \gamma]$, with $I$ winning if the sequence $\langle z_{\xi} : \xi < \gamma\rangle$ is
in $C$. It follows from mild large cardinal assumptions (for instance, the existence of the sharp of
every subset of $\omega_{1}$) that $\gamma$ must be countable.


%Given $C \subseteq\mathbb{R}^{\less\omega_{1}}$ and partial $\nu
%\colon \mathbb{R} \to \mathbb{N}$, $G_{{\rm cont}-\nu}(C)$ is played
%in a segments of $\omega$ many round, in the $\alpha$th of which the
%two player collaborate to build a real $y_{\alpha}$. If $y_{\alpha}$
%is not in the domain of $\nu$ the game ends. Otherwise, the game
%ends if $\nu(y_{\alpha}) = \nu(y_{\xi})$ for some $\xi < \alpha$. In
%either case, the winner is determined by whether the sequence of
%$y_{\alpha}$'s is in $C$. Such a game must end in countably many
%moves. The sequence of values $\nu(y_{\alpha})$ induced a code of
%the play so far by a real. For a given pointclass $\Gamma$, then,
%one can talk of the payoff set $C$ being $\Gamma$ \emph{in the
%codes} if there is a set $A$ in $\Gamma$ such that the run of the
%game is in $C$ if and only if the code induced by the sequence of
%$\nu(y_{\alpha})$'s is in $A$.

%The notation $M\cut\kappa$\index{$M\cut\kappa$} is used to denote
%$V_{\kappa}^{M}$.

%\begin{theorem}[Neeman \cite{Neeman:2005}] Suppose that there exists
%an iterable class model $M$ of ZFC and ordinals $\delta < \lambda$
%in $M$ which are Woodin cardinals in $M$, such that $M\cut(\lambda +
%1)$ is countable in $V$ and such that for every $X \in M\cut(\delta
%+ 1)$ there exists an extender $E$ in $M$ overlapping $\delta$ such
%that $X \in Ult(M,E)$. Then all games $G_{{\rm cont}-\nu}(C)$ are
%determined, where $\nu$ is $\Sigma^{0}_{2}$ measurable and $C$ is
%$\Sigma^{1}_{2}$ in the codes.
%\end{theorem}




%In the weak iteration game, one player picks a succession of
%iteration trees of length $\omega$ (starting with the final model of
%the previous tree) and the other player picks branches giving rise
%to wellfounded models. A model $M$ is \emph{weakly iterable} if the
%second player has a winning strategy for the weak iteration game on
%$M$.

%\begin{theorem}[Neeman \cite{Neeman:2006firstadmissible}]
%Suppose that there exists a weakly iterable class model $M$ and an
%ordinal $\kappa$ of $M$ such that $\kappa$ is a limit of Woodin
%cardinals in $M$ and \begin{itemize}
%\item for every $B \subseteq M \cut (\kappa + 1)$ in $M$ there exist a
%ultrafilter $\mu \in M$ on $\kappa$ such that $Z \in Ult(M, \mu)$;
%\item
%$M\cut(\kappa + 2)$ is countable in $V$.
%\end{itemize}
%Then the games $G_{\rm adm}(C)$ are determined for all $C$ in
%$\less\omega^{2}\thing\Pi^{1}_{1}$.
%\end{theorem}

%\begin{theorem}[Neeman \cite{Neeman:2006firstadmissible}]
%Suppose that there exists a countable ordinal $\theta$ and a weakly
%iterable class model $M$ and an ordinal $\kappa$ such that $\kappa$
%is a limit of Woodin cardinals in $M$ and \begin{itemize}
%ultrafilter $\mu \in M$ on $\kappa$ such that $Z \in Ult(M, \mu)$;
%\item there exists a sequence $\langle \delta_{\xi} :\xi <
%\theta\rangle$ of Woodin cardinals in $M$ above $\kappa$;
%\item
%$M\cut(\sup_{\xi < \theta}(\delta_{\xi + 1})$ is countable in $V$.
%\end{itemize}
%Then the games $G(amd + \theta, C)$ are determined for all $C$ in
%$\less\omega^{2}\thing\Pi^{1}_{1}$.
%\end{theorem}

The existence of $M$ satisfying the hypotheses of the following
theorem follows from the existence in $V$ of a sharp for a Woodin
limit of Woodin cardinals \cite{Neeman:2002}. Given a pointclass $\Gamma$, a
set $C$ consisting of countable sequences of reals is said to be $\Gamma$ \emph{in the codes}
if the set of reals coding members of $C$ (under a suitably definable coding) is in $\Gamma$.

\begin{theorem}[Neeman \shortcite{Neeman:DLG}]\label{7E1} Suppose that there exists
an iterable class model $M$ with a cardinal $\theta$ such that
$\theta$ is a Woodin limit of Woodin cardinals in $M$ and countable
in $V$. Then the games $G_{local}(L,C)$ are determined for all $C$
which are $\game_{\omega}(\less\omega^{2}\thing\Pi^{1}_{1})$ in the
codes.
\end{theorem}

Woodin\index{Woodin, W. H.} used techniques from the proof of this theorem and his work on
AD$^{+}$ to prove the following.

\begin{theorem}[Woodin\index{Woodin, W. H.}] Suppose that there is a Woodin limit of
Woodin cardinals and a proper class of inaccessible limits of Woodin
cardinals above it. Let $A$ be universally Baire and let $\Lambda$
be the pointclass of recursive preimages of $A$. Let $C$ be
$\Lambda$ in the codes. Then $G_{\rm local}(L, C)$ is
%%%%%SOLOVAY38: The "L" in the previous line was previously a "$\Lambda$".
determined.
\end{theorem}



Again using some of the techniques from the proof of Theorem
\ref{7E1}, Woodin\index{Woodin, W. H.} proved that if there exists an iterable class
model $M$ with a cardinal $\theta$ which is a Woodin limit of Woodin
cardinals in $M$ and countable in $V$, then it is consistent that
all games on integers of length $\omega_{1}$ with payoff definable
from reals and ordinals are determined (see Exercise 7F.15 of
\cite{Neeman:DLG}). Larson\index{Larson, P.} and Shelah\index{Shelah, S.} \shortcite{LarsonShelah:split}
showed that it is possible to force that some such game is not
determined.

%Many of Neeman's determinacy results have as their hypothesis the
%existence of an iterable model satisfying a certain large cardinal
%property. The existence of such models is the goal of the inner
%model program. In some cases this program has succeeded, but some of
%Neeman's results use hypotheses of this type which have not been
%derived from large cardinal axioms.

We give one more result of Neeman, proving the determinacy of
certain games of length $\omega_{1}$. In Theorem \ref{Neemanlongdet}
below, $\mathcal{L}^{+}$ is the language of set theory with one
additional unary predicate. Given an integer $k$ and a sequence
$\bar{S}$ of stationary sets indexed by $[\omega_{1}]^{\less k}$,
$[\bar{S}]$ is the collection of increasing $k$-tuples $\langle
\alpha_{0}, \ldots, \alpha_{k-1}\rangle$ from $\omega_{1}$ such that
each initial segment of length $j \leq k$ is in $S_{\langle
\alpha_{0},\ldots,\alpha_{j-l}\rangle}$. The game $G_{\omega_{1},
k}(\bar{S}, \phi)$ is a game of length $\omega_{1}$ in which the
players collaborate to build a function $f \colon \omega_{1} \to
\omega_{1}$. Then player $I$ wins if there is a club $C$ such that
$\langle L_{\omega_{1}}, r\rangle \models
\phi(\alpha_{0},\ldots,\alpha_{k-1})$ for all $\langle
\alpha_{0},\ldots,\alpha_{k-1}\rangle \in [\bar{S}] \cap [C]^{k}$,
and player $II$ wins if there is a club $C$ such that $\langle
L_{\omega_{1}}, r\rangle \models
\neg\phi(\alpha_{0},\ldots,\alpha_{k-1})$ for all $\langle
\alpha_{0},\ldots,\alpha_{k-1}\rangle \in [\bar{S}] \cap [C]^{k}$.
Though there can be runs of the game for which neither player wins,
determinacy for this game in the sense of Theorem
\ref{Neemanlongdet} refers to the existence of a strategy for one
player or the other that guarantees victory.



%More specifically, Neeman\index{Neeman, I.} showed that for any recursive enumeration
%of the $\game^{(n)}(\less\omega^{2}\thing\Pi^{1}_{1})$ sets, then
%all $\game^{(n)}(\less\omega^{2}\thing\Pi^{1}_{1})$ sets are
%determined, and the set of $i$ such that $I$ has a winning strategy
%in $G_{\omega}(B_{i})$ is recursively isomorphic to $0^{\#}$.


The model $0^{W}$ is the minimal iterable fine structural inner
model $M$ which has a top extender predicate whose critical point is
Woodin in $M$. The existence of such a model is not known to follow
from large cardinals.

The last part of the conclusion of Theorem \ref{Neemanlongdet} extends a result of
Martin\index{Martin, D. A.}, who showed that for any recursive enumeration
$\langle B_{i} : i < \omega\rangle$
of the $\less\omega^{2}$-$\Pi^{1}_{1}$ sets, the set of $i$ such
that $I$ has a winning strategy in $G_{\omega}(B_{i})$ is
recursively isomorphic to $0^{\#}$.

\begin{theorem}[Neeman\index{Neeman, I.} \shortcite{Neeman:long}]\label{Neemanlongdet} Suppose that $0^{W}$
exists. Let $k < \omega$. Let $\bar{S}$ be a sequence of mutually
disjoint stationary sets indexed by $[\omega_{1}]^{\less k}$. Let
$\phi$ be a $\mathcal{L}^{+}$ formula with $k$ free variables. Then
the game $G_{\omega_{1}, k}(\bar{S}, \phi)$ is determined.
Furthermore, the winner of each such game depends only on $\phi$ and
not on $\bar{S}$, and the set of $\phi$ for which the first player
has a winning strategy is recursively equivalent to $0^{W}$.
\end{theorem}



If one allows the members of $\bar{S}$ all to be $\omega_{1}$, then
there are undetermined games of this type, as observed by Greg Hjorth\index{Hjorth, G.}
(see \cite{Neeman:long}). If one allows the members of $\bar{S}$ all
to be $\omega_{1}$ and changes the winning condition for player $II$
to be simply the negation of the winning condition for $I$, then one
can force from a strongly inaccessible limit of measurable cardinals
that some game of this type is not determined \cite{Larson:2005}\index{Larson, P.}.


\subsection{Forcing over models of determinacy}\label{fomd}

Steel\index{Steel, J.} and Van Wesep\index{Van Wesep, R.}
\shortcite{SteelVanWesep:1982} showed that by forcing over a model
of AD$_{\mathbb{R}}$ + ``$\Theta$ is regular'' (the hypothesis they
used was actually weaker) one can produce a model of ZFC in which
NS$_{\omega_{1}}$ is saturated and $\undertilde{\delta}^{1}_{2}
=\omega_{2}$. This was the first consistency proof of either of
these two statements with ZFC. Martin\index{Martin, D. A.} had
conjectured that ``$\forall n \in \omega\,
\undertilde{\delta}^{1}_{n} = \aleph_{n}$'' is consistent with ZFC,
and this verified the conjecture for the case $n = 2$. Woodin
\index{Woodin, W. H.}\shortcite{Woodin:1983ZFCAD} subsequently
reduced the hypothesis to AD.

Shelah\index{Shelah, S.} \shortcite{Shelah:PIF} later showed that it was possible to force
the saturation of NS$_{\omega_{1}}$ from a Woodin cardinal. Woodin
\index{Woodin, W. H.} \shortcite{Woodin:1999} proved that the saturation of NS$_{\omega_{1}}$
plus the existence of a measurable cardinal implies that
$\undertilde{\delta}^{1}_{2} = \omega_{2}$.  Woodin\index{Woodin, W. H.} then turned his
proof into a general method for producing models of ZFC by forcing
over models of determinacy. The most general form of this method, a
partial order called $\mathbb{P}_{\rm max}$, consists roughly of a
directed system containing all countable models of ZFC with a precipitous
ideal on $\omega_{1}$. In the presence of large cardinals, the resulting
extension is maximal for $\Pi_{2}$ sentences in $H(\omega_{2})$, even
with predicates for NS$_{\omega_{1}}$ and each set of reals in
$L(\mathbb{R})$. In this model, NS$_{\omega_{1}}$ is saturated and
$\undertilde{\delta}^{1}_{2} = \omega_{2}$. There are many variants
of $\mathbb{P}_{\rm max}$. One of these variants, called
$\mathbb{Q}_{\rm max}$, produces a model in which NS$_{\omega_{1}}$
is $\aleph_{1}$-dense (i.e., $\mathcal{P}(\omega_{1})/{\rm NS}_{\omega_{1}}$ has a dense
subset of cardinality $\aleph_{1}$; this implies saturation), from the assumption that AD holds in
$L(\mathbb{R})$. No other method is known for producing a model of
ZFC in which NS$_{\omega_{1}}$ is $\aleph_{1}$-dense.

%Neeman \cite{Neeman:ult} and Woodin independently showed that
%$\undertilde{\delta}^{1}_{n}$ can be $\omega_{2}$ for any odd $n$.

Steel\index{Steel, J.} \shortcite{Steel:1995} showed that under AD, HOD$^{L(\mathbb{R})}$
is a core model below $\Theta$. Woodin\index{Woodin, W. H.} then showed that
the entire model HOD$^{L(\mathbb{R})}$ can be viewed as a core model plus a fragment of its
iteration strategy. Using this approach, Steel\index{Steel, J.} showed that for every
regular $\kappa < \Theta$, the $\omega$-club filter over $\kappa$ is
an ultrafilter in $L(\mathbb{R})$. Woodin\index{Woodin, W. H.} used this
to show that
$\omega_{1}$ is $\less\Theta$-supercompact in $L(\mathbb{R})$, and
showed that $\omega_{1}$ is huge to $\kappa$ for each measurable
$\kappa < \Theta$, improving results of Becker.\index{Becker, H.} Neeman
\shortcite{Neeman:2007} used this approach to prove, for each $\lambda < \Theta$, the uniqueness of the
normal ultrafilter on $\mathcal{P}_{\aleph_{1}} \lambda$ witnessing
the $\lambda$-supercompactness of $\omega_{1}$. Solovay\index{Solovay, R.} had proved this using determinacy
for games on $\lambda$, and Harrington\index{Harrington, L.} and Kechris\index{Kechris, A.}
\shortcite{HarringtonKechris:1981} proved the determinacy of these games
from AD for $\lambda < \undertilde{\delta}^{2}_{1}$. Woodin\index{Woodin, W. H.} proved
the uniqueness of these ultrafilters witnessing supercompactness up to
$\undertilde{\delta}^{2}_{1}$.

Neeman\index{Neeman, I.} and Woodin\index{Woodin, W. H.} independently used
this approach to show that, assuming AD + $V = L(\mathbb{R})$, one
could force without adding reals to obtain ZFC +
$\undertilde{\delta}^{1}_{n} = \omega_{2}$, for any $n \geq 3$.
It is still unknown whether $\undertilde{\delta}^{1}_{n}$ can equal
$\omega_{n}$ for any $n > 2$ (under ZFC).
%%%%%SOLOVAY31: "other" removed from the previous line.

%The Continuum
%Hypothesis is one of the oldest questions in set theory, and this
%can be seen as a effective version of the continuum question.


\subsection{Determinacy from its consequences}

Woodin\index{Woodin, W. H.} \shortcite{Woodin:1982} conjectured that Projective Determinacy follows from the
statement that all projective sets are Lebesgue measurable, have the Baire property and can be uniformized by
projective functions (all consequences of PD). This conjecture was refuted by Steel\index{Steel, J.} in 1997. If
one requires the uniformization property for the scaled projective
pointclasses, then the conjecture is still
open. Woodin\index{Woodin, W. H.} did prove the following version of the conjecture in the late 1990's.
%Woodin proved the following in the early 1990's, after Steel had shown that a
%corresponding version of the question for PD (asked by Woodin) was false.
The reverse direction of the
theorem used work of Steel in inner model theory.


\begin{theorem}[Woodin\index{Woodin, W. H.}] Assume {\rm ZF + DC + }$V\mathord{=}L(\mathbb{R})$. Then
the following are equivalent.
\begin{itemize}
\item {\rm AD}
\item Every set of reals is Lebesgue measurable and has the property
of Baire, and every $\Sigma^{2}_{1}$ subset of\/
${^{2}(^{\omega}\omega)}$ can be uniformized.
\end{itemize}
\end{theorem}

%\begin{theorem}[Woodin\index{Woodin, W. H.}] The statement that AD holds in
%$L(\mathbb{R})$ is equivalent to the statement that for every $A
%\subseteq\breals \times \breals$ in $L(\mathbb{R})$ which is
%$\uTDelta^{2}_{1}$ definable in $L(\mathbb{R})$, $A$ is Lebesgue
%measurable, has the property of Baire and can be uniformized by a
%function in $L(\mathbb{R})$.
%\end{theorem}

Woodin\index{Woodin, W. H.} had proved another equivalence in the early 1980's.

\begin{theorem}[Woodin\index{Woodin, W. H.}] Assume {\rm ZF + DC + }$V\mathord{=}L(\mathbb{R})$. Then
the following are equivalent.
\begin{itemize}
\item {\rm AD}
\item Turing determinacy.
\end{itemize}
\end{theorem}

It is apparently an open question whether AD follows from ZF + DC +
$V\mathord{=}L(\mathbb{R})$ plus either of (a) for every $\alpha< \Theta$
there is a surjection of $\breals$ onto $\mathcal{P}(\alpha)$; (b)
$\Theta$ is inaccessible.

Steel\index{Steel, J.} \shortcite{Steel:1996} showed that the reduction property for
$\uTPi^{1}_{3}$ plus the existence of sharps for reals implies that
there are inner models with Woodin cardinals.







%Martin\index{Martin, D. A.} (see \cite{KechrisSolovay:1985}) proved under ZF + DC that
%$\Delta^{1}_{2n}$-determinacy implies $\Sigma^{1}_{2n}$-determinacy
%for each $n \in \omega$.



Determinacy would turn out to be necessary for some of its earliest applications.
For instance, Steel\index{Steel, J.} \shortcite{Steel:1996} showed that $\uTSigma^{1}_{3}$-separation plus
the existence of sharps for all reals implies $\uTDelta^{1}_{2}$-determinacy. Hjorth\index{Hjorth, G.} \shortcite{Hjorth:1996Wadge}
showed that $\uTPi^{1}_{2}$-determinacy follows from Wadge determinacy for $\uTPi^{1}_{2}$ sets. Earlier, Harrington\index{Harrington, L.}
had shown that, for each real $x$, $\uTPi^{1}_{1}(x)$-Wadge determinacy implies that $x^{\#}$ exists.


%\begin{theorem}[Steel\index{Steel, J.} \shortcite{Steel:1996}] Suppose that for every $x
%\in \breals$, $x^{\#}$ exists and $\Sigma^{1}_{3}(x)$-separation
%holds. Then $\uTDelta^{1}_{2}$-determinacy holds.
%\end{theorem}

%\begin{theorem}[Harrington\index{Harrington, L.}] For all $x \in \breals$, $\uTPi^{1}_{1}(x)$ Wadge determinacy implies
%that $x^{\#}$ exists.
%\end{theorem}

%\begin{theorem}[Hjorth \shortcite{Hjorth:1996Wadge}] $\uTPi^{1}_{2}$ Wadge determinacy implies
%$\uTPi^{1}_{2}$-determinacy.
%\end{theorem}





\subsection{Determinacy from other statements}

Determinacy axioms such as PD and AD$^{L(\mathbb{R})}$ imply the consistency of ZFC (plus certain large cardinal statements)
and so cannot be proved in ZFC. Empirically, however, these statements appear to follow from every natural statement of sufficient
consistency strength. This includes a number of statements ostensibly having little relation to determinacy. In this section we give a few examples
of this phenomenon. Most of these arguments use inner model theory, and our presentation relies heavily on \cite{Schimmerling:handbook}.


The following theorem shows, among other things, that in the presence of large cardinals, even mere forcing-absoluteness for the
theory of $L(\mathbb{R})$ implies AD$^{L(\mathbb{R})}$. The theorem is due to Steel\index{Steel, J.} and Woodin\index{Woodin, W. H.} independently (see
\cite{Steel:2002}).

\begin{theorem} Suppose that $\kappa$ is a measurable cardinal. Then
the following are equivalent.
\begin{itemize}
\item For all partial orders $\mathbb{P} \in V_{\kappa}$, the theory
of $L(\mathbb{R})$ is not changed by forcing with $\mathbb{P}$.
\item For all partial orders $\mathbb{P} \in V_{\kappa}$, {\rm AD} holds in $L(\mathbb{R})$
after forcing with $\mathbb{P}$.
\item For all partial orders
$\mathbb{P} \in V_{\kappa}$, all sets of reals in $L(\mathbb{R})$
are Lebesgue measurable after forcing with $\mathbb{P}$.
\item For all partial orders
$\mathbb{P} \in V_{\kappa}$, there is no $\omega_{1}$-sequence of
reals in $L(\mathbb{R})$ after forcing with $\mathbb{P}$.
\end{itemize}
\end{theorem}



%The forward direction of the following theorem uses Woodin's\index{Woodin, W. H.} $\pmax$
%forcing technology. The reverse direction uses his Core Model
%Induction, and builds on work of Steel\index{Steel, J.} in \shortcite{Steel:1996}.

%\begin{theorem} The following are equiconsistent over ZFC.
%\begin{itemize}
%\item AD holds in $L(\mathbb{R})$.
%\item There is an $\aleph_{1}$-dense ideal on $\omega_{1}$.
%\end{itemize}
%\end{theorem}

As discussed in Section \ref{fomd}, Woodin\index{Woodin, W. H.} showed using a variation of $\mathbb{P}_{\rm{max}}$ that
over a model of AD one can force to produce a model of ZFC in which the nonstationary ideal on $\omega_{1}$
is $\aleph_{1}$-dense. Woodin\index{Woodin, W. H.} developed a technique known as the \emph{core
model induction}, an
application of descriptive set theory and core model theory, and used it to show that the $\aleph_{1}$-density of
NS$_{\omega_{1}}$ implies AD$^{L(\mathbb{R})}$. It follows that the two statements are equiconsistent.
Steel had previously shown, using core models, that Projective Determinacy follows from CH plus the existence
of a homogeneous ideal on $\omega_{1}$ (a weaker assumption that the $\aleph_{1}$-density of NS$_{\omega_{1}}$, which is in turn
inconsistent with CH, by a theorem of Shelah).

%\begin{theorem}[Woodin\index{Woodin, W. H.}] The following are equiconsistent.
%\begin{enumerate}
%\item ZF + AD
%\item ZFC + There are infinitely many Woodin cardinals.
%\item ZFC + The nonstationary ideal on $\omega_{1}$ is $\aleph_{1}$-dense.
%\end{enumerate}
%\end{theorem}

%Let $C_{\Gamma}(x)$ denote the set of reals $y$ such that for some
%countable ordinal $\xi$, $y$ is $\Gamma(x,z)$ for all $z$ coding
%$\xi$.

%A pointclass is \emph{good} if it is $\omega$-parameterized, closed
%under recursive substitution, number quantification and
%$\exists^{1}$, and has the scale property. Kechris\index{Kechris, A.}
%\cite{Kechris:1975} showed that used AD, if $\Gamma$ is good then
%$C_{\Gamma}(x)$ is the largest countable $\Gamma(x)$ set of reals,
%and $C_{\Gamma}(x)$ has a $\Delta(x)$-good wellorder.

%Harrington\index{Harrington, L.} and Kechris\index{Kechris, A.} \cite{HarringtonKechris:1981} showed that if
%$\Gamma$ is good and $T$ is a tree witnessing that $\Gamma$ has the
%scale property and $a$ is countable and transitive, then
%$C_{\Gamma}(a)$ is equal to $\mathcal{P}(a) \cap L(a \cup \{T, a\})$
%and the set of $b \subseteq A$ such that for comeagerly many $x$
%coding $a$, $b_{x} \in C_{\Gamma}(x)$.

%Kechris\index{Kechris, A.} showed that under AD, for every set of ordinals $S$, $L[S,
%x]$ satisfies OD$_{S}$-determinacy for an $S$-cone of reals $x$. It
%follows that $\omega_{1}^{L[S,x]}$ is measurable in
%$HOD^{L[S,x]}_{S}$.

A sequence $C = \langle C_{\alpha} : \alpha < \lambda \rangle$ (for
some ordinal $\lambda$) is called a \emph{coherent sequence} (of
length $\lambda$) if each $C_{\beta}$ is a club subset of $\beta$,
and $C_{\alpha} = \alpha \cap C_{\beta}$ whenever $\alpha$ is a
limit point of $C_{\beta}$. A \emph{thread} of such a coherent
sequence $C$ is a club set $D \subseteq \lambda$ such that
$C_{\alpha} = \alpha \cap D$ for all limit points $\alpha$ of $D$.
The principle $\square(\lambda)$ says that there is a coherent
sequence of length $\lambda$ with no thread. The principle
$\square_{\kappa}$ says that there is a coherent sequence $C$ of
length $\kappa^{+}$ such that the ordertype of $C_{\alpha}$ is at
most $\kappa$, for each limit $\alpha < \lambda$. The principles $\square_{\kappa}$
were isolated in the 1960's by Jensen\index{Jensen, R.} \shortcite{Jensen:1972},
who showed that $\square_{\kappa}$ holds in $L$ for all
infinite cardinals $\kappa$ (see \cite[p.~141]{Devlin}). The principles $\square(\lambda)$
%%%%%SOLOVAY: Devlin was added to the bibliography.
were later extracted from Jensen's work by Stevo Todorcevic\index{Todorcevic, S.}.
%%%%%SOLOVAY:The previous line was rewritten (i.e., "extracted from Jensen's work" was added).
%\cite[pp.~267-268]{Todorcevic:1987}.


Todorcevic\index{Todorcevic, S.} \shortcite{Todorcevic:pfa}
showed that the Proper Forcing Axiom (PFA)\index{Proper Forcing Axiom}\index{PFA} implies that
$\square(\kappa)$ fails at all cardinals $\kappa$ of cofinality at
least $\omega_{2}$, from which it follows that $\square_{\kappa}$
fails for all uncountable cardinals. The failure of these square
principles implies the failure of covering theorems for certain
inner models, from which one can derive inner models with large
cardinals. Using this general approach, Ernest
Schimmerling\index{Schimmerling, E.} \shortcite{Schimmerling:1995} proved that
PFA implies $\uTDelta^{1}_{2}$-determinacy. Woodin\index{Woodin, W. H.}
extended this proof to show that PFA implies PD.
Woodin\index{Woodin, W. H.} also showed that PFA plus the existence
of a strongly inaccessible cardinal implies AD$^{L(\mathbb{R})}$,
using the core model induction.
Roughly, the idea is to inductively work through the Wadge degrees
to build canonical inner models which are correct for each Wadge
class. The induction works through the gap structure highlighted in
\cite{Steel:1983}.\index{Steel, J.}

Alessandro Andretta,\index{Andretta, A.} Neeman\index{Neeman, I.} and Steel\index{Steel, J.} \shortcite{AndrettaNeemanSteel}
showed that PFA plus the existence of a
measurable cardinal implies the existence of a model of
AD$_{\mathbb{R}}$ containing all the reals and ordinals.
Steel\index{Steel, J.} \shortcite{Steel:2005} showed that if $\square_{\kappa}$ fails for
a singular strong limit cardinal $\kappa$, then AD holds in
$L(\mathbb{R})$.


The following theorem is due to Schimmerling\index{Schimmerling, E.} \shortcite{Schimmerling:2007}.
Steel later strengthened the conclusion to AD$^{L(\mathbb{R})}$.

\begin{theorem} If $\kappa \geq \max\{\aleph_{2}, c\}$ and\/
$\square(\kappa)$ and\/ $\square_{\kappa}$ fail, then Projective Determinacy holds.
\end{theorem}

Todorcevic\index{Todorcevic, S.} (see \cite{Bekkali}) and Boban Veli\v{c}kovi\'{c}\index{Veli\v{c}kovi\'{c}, B.} \shortcite{Velickovic:1992} showed
that PFA implies that $2^{\aleph_{0}} = 2^{\aleph_{1}} = \aleph_{2}$. This gives another route towards showing that PFA implies that the AD holds in $L(\mathbb{R})$.

%\begin{theorem} If $\kappa \geq \max\{\aleph_{2}, c\}$ and
%$\square(\kappa)$ and $\square_{\kappa}$ fail, then AD holds in
%$L(\mathbb{R})$.
%\end{theorem}

%\begin{theorem} If $\kappa$ is a singular strong limit and $\square_{\kappa}$ fails, then AD holds in
%$L(\mathbb{R})$.
%\end{theorem}

Schimmerling\index{Schimmerling, E.} and Martin Zeman\index{Zeman, M.} used the core model induction to prove
the following theorem \shortcite{SchimmerlingZeman}. They had previously derived Projective Determinacy from
the failure of a weaker version of $\square_{\kappa}$ at a weakly compact cardinal; Woodin had then derived AD$^{L(\mathbb{R})}$ from
the same hypothesis.

\begin{theorem} If $\kappa$ is a weakly compact cardinal and\/
$\square_{\kappa}$ fails, then {\rm AD} holds in
$L(\mathbb{R})$.
\end{theorem}


%The principle $\square(\lambda)$ says that there is a coherent
%sequence of length $\lambda$ with no thread. The principle
%$\square_{\kappa}$ says that there is a coherent sequence $C$ of
%length $\kappa^{+}$ such that the ordertype of each $C_{\alpha}$ is
%less than or equal to $\kappa$ for all limit $\alpha < \lambda$.



%In the early 1990's, Ernest Schimmerling\index{Schimmerling, E.} proved $\Delta^{1}_{2}$
%determinacy from PFA, and





%Schindler and Steel\index{Steel, J.} used Woodn's core model induction to prove that
%AD + $V\mathord{=}L(\mathbb{R})$ implies that in some forcing extension
%there exists a fine structural inner extender model containing
%infinitely many Woodin cardinals (cofinal in $\omega_{1}^{V})$.



Steel\index{Steel, J.} \shortcite{Steel:1996} showed that if NS$_{\omega_{1}}$ is
saturated and there is a measurable cardinal, then
$\uTDelta^{1}_{2}$-determinacy holds. The hypothesis of the measurable cardinal was
later removed through work of Jensen\index{Jensen, R.} and Steel.\index{Steel, J.}
%Woodin\index{Woodin, W. H.} \cite{Woodin:1999} proved that if SRP$(\omega_{2})$ holds,
%then PD continues to hold after collapsing $\omega_{2}$.
In 2000, Steel\index{Steel, J.} and Zoble \cite{SteelZoble} derived AD$^{L(\mathbb{R})}$ from
another principle of Todorcevic, known as the \emph{Strong Reflection Principle at} $\omega_{2}$.

We conclude with one final example. As mentioned in Section \ref{adsubsec}, Gitik\index{Gitik, M.} showed that if there is a proper class of strongly compact
cardinals, then there is a model of ZF in which all infinite
cardinals have cofinality $\omega$. Using the core model induction, Daniel Busche\index{Busche, D.} and Ralf Schindler\index{Schindler, R.} \shortcite{BuscheSchindler} showed that this
statement implies that AD holds in the $L(\mathbb{R})$ of a forcing
extension of HOD, for some partial order in HOD.

%\subsection{Determinacy and the failure of square}



%Todorcevic had previously shown that the Proper Forcing Axiom
%implies that $\square_{\kappa}$ fails for all uncountable cardinals
%$\kappa$.










%\section{Proving determinacy in weak theories}





%\section{Blackwell determinacy}

%Blackwell introduced a form of game of imperfect information in
%\cite{Blackwell:1969}. Martin\index{Martin, D. A.} \cite{Martin:1998} proved that AD
%implies Blackwell determinacy. Martin,\index{Martin, D. A.} Neeman\index{Neeman, I.} and Vervoort showed
%that if Blackwell determinacy holds in $L(\mathbb{R})$, then AD
%holds in $L(\mathbb{R})$.


%\begin{enumerate}
%\item Check with Steel's history
%\item Steel's comments
%\item Tree Production comments
%\end{enumerate}





\section*{Acknowledgements}

Gunter Fuchs\index{Fuchs, G.} helped with some original sources in German. The author would like to thank
Akihiro Kanamori, Alexander Kechris, Tony Martin, Jan Mycielski, Robert Solovay and John Steel for making many helpful suggestions.
%%%%%SOLOVAY: added "Robert Solovay".

\bibliographystyle{named}
%\bibliography{BiBfile}
%%%%%TEMPORARILY COMMENTED OUT\bibliography{plarson4-14}
\bibliography{determinacy}

%\begin{thebibliography}{}
%etc, etc
%\end{thebibliography}
\printindex
\end{document}
